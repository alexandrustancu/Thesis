\section{Conţinutul tezei de doctorat}

Lucrarea este împărţită în şapte capitole, primul prezentând domeniul abordat în teză, iar ultimul fiind dedicat concluziilor. În continuare va fi prezentat, pe scurt, conţinutul fiecărui dintre celelalte capitole.

Capitolul \ref{ch:introducere_sdn} introduce domeniul reţelelor definite prin software, plecând de la istoria sa și nevoia pentru care această nouă paradigmă a apărut. Apoi va fi ilustrată activitatea de standardizare în acest domeniu, inclusiv demonstraţiile de concept conduse de către \gls{onf}, în particular de către grupul \gls{wt}, care va duce la maturizarea soluţiei și adoptarea acesteia pe scară largă, în toate aspectele unei reţele. Tot în acest capitol se va evidenţia și prezenţa \gls{sdn} în contextul reţelelor actuale.

Cel de-al \ref{ch:sdn_in_contextul_wt}-lea capitol pune accent pe prezenţa \gls{sdn} în reţelele de transport de date fără fir. Sunt prezentate modelele informaţionale dezvoltate în cadrul \gls{onf} în acest context, având rolul de \textit{recomandări tehnice} - \gls{tr}: \textit{modelul informațional pentru microunde} - \textit{TR-532, Microwave Information Model} și \textit{modelul informațional de bază} - \textit{TR-512, Core Information Model}. Ulterior se vor da detalii despre \gls{netconf}, care este protocolul de bază pentru reţelele de transport de date fără fir, în contextul \gls{sdn}. În următoarea secţiune se vor compara câteva soluții software cu sursă deschisă, ce oferă facilitatea unui server \gls{netconf}. Pe baza acestei comparaţii s-a ales unealta software care va face parte din simulatorul propus în lucrare. Capitolul va fi încheiat de o prezentare a arhitecturii demonstraţiilor de concept organizate de grupul \gls{wt} din \gls{onf}, ce va ajuta la înţelegerea necesităţii unui astfel de simulator.

Capitolul \ref{ch:dvm_v01} prezintă prima versiune a simulatorului, numită \textit{Mediatorul cu valori implicite} - \gls{dvm}, folosită în cea de-a doua demonstraţie de concept \gls{wt} din \gls{onf}. Se vor prezenta, pe rând, arhitectura și implementarea, iar apoi se va evidenţia folosirea acestui simulator în contextul demonstraţiilor de concept. Apoi, se va descrie cea de-a doua versiune a \gls{dvm}, abordând aspecte despre arhitectura, implementare și folosire în cadrul celei de-a treia demonstraţii de concept \gls{wt} din \gls{onf}. În plus, se va evidenţia încercarea de a integra acest simulator cu o soluţie de comutator software, LINC, folosit în \gls{sdn}, în special în cadrul reţelelor de transport optic de date, prezentând avantajele și dezavantajele date de această abordare.

Capitolul \ref{ch:wte} descrie ultima și cea mai avansată versiune a \textit{simulatorului reţelelor de transport de date fără fir} - \gls{wte}, prezentând arhitectura, detaliile implementării și folosirea acestuia în cea de-a patra demonstraţie de concept a grupului \gls{wt} din cadrul \gls{onf}.

Capitolul \ref{ch:rezultate_discutii} ilustrează rezultatele obţinute în urma acestei cercetări și propune discuţii pe baza simulatorului implementat. În primul rând, această soluţie este evaluată din punctul de vedere al resurselor consumate și al extensibilităţii pe care o oferă. Apoi, se compară simulatorul cu alte soluţii care există în momentul de faţă în contextul \gls{sdn}. Ulterior, se prezintă câteva cazuri de utilizare, propuse în cadrul grupului \gls{wt} din \gls{onf}, care pot fi demonstrate cu ajutorul simulatorului, eliminând nevoia unor echipamente de transport de date fără fir.