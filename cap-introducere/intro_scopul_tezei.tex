\section{Scopul tezei de doctorat}

Această lucrare îşi propune să prezinte noua paradigmă introdusă anterior, \gls{sdn}, împreună cu avantajele pe care această abordare le poate aduce dacă ar fi aplicată în toate aspectele unei reţele de comunicaţii, punând accent pe reţelele de transport de date fără fir. Autorul îşi va prezenta activitatea de cercetare, constând în unelte software care pot fi folosite ca simulatoare de echipamente de transport de date fără fir, ce expun interfeţe folosite în tehnologia reţelelor definite prin software.

\textbf{Aceste unelte software sunt contribuții originale, placând de la partea de arhitectură și până la cea de implementare.} Ele au fost folosite cu succes în procesul de standardizare al \gls{sdn}, care încă se desfăşoară în cadrul \gls{onf}, facilitând testarea modelelor informaţionale ce se dezvoltă în contextul reţelelor definite prin software și uşurând dezvoltarea și testarea aplicaţiilor \gls{sdn} care fac parte din acest ecosistem. Simulatorul rezultat în urma acestei cercetări, în forma sa finală, poate emula o întreagă reţea de echipamente de transport de date fără fir, care expun interfeţe specifice \gls{sdn}. El poate fi folosit de către dezvoltatorii de produse software \gls{sdn} pentru reţele de transport de date fără fir, eliminând nevoia acestora de a deţine astfel de echipamente scumpe pentru a-și putea testa aplicaţiile. Poate fi folosit și de către operatorii care vor sa lanseze această tehnologie în reţelele de producţie, pentru a simula consecinţele instalării unor noi aplicaţii anterior lansării, fără a afecta reţeaua.