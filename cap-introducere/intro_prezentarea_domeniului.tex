\section{Prezentarea domeniului tezei de doctorat}

În urma dezvoltării tehnologice recente în toate domeniile, în general și în domeniul calculatoarelor și al telecomunicaţiilor, în particular, a apărut nevoia de a redefini arhitectura reţelelor de comunicaţii, din cauza faptului că reţelele tradiţionale au început să îşi arate limitele. În zilele noastre, există o tendinţă de a interconecta toate echipamentele, cu ajutorul unor tehnologii care permit acest lucru, cum ar fi arhitectura \textit{Cloud Computing}, de a crește mobilitatea tuturor aplicațiilor, de a defini concepte și aplicații noi, cum ar fi \textit{Internetul Tuturor Lucrurilor} - \gls{iot} sau sistemele de comunicaţii de generaţia a cincea - 5G. Aceste noi abordări au nevoie, pe lângă o lărgime de bandă crescută, de o reţea mai simplă și agilă, care să faciliteze inovarea. Reţelele definite prin software - \gls{sdn}, reprezintă o nouă paradigmă care a apărut în industria reţelisticii, pentru a contracara dezavantajele pe care reţelele tradiţionale le-au dovedit.

Tehnologia \gls{sdn} nu a ajuns încă la maturitate și nu a pătruns în toate tipurile de reţele de comunicaţii existente. Este prezentă în campusuri universitare, sau în centre de date, însă se încearcă introducerea acesteia în alte reţele de comunicaţii, cum ar fi transportul de date optic - \gls{ot}, transportul de date fără fir - \gls{wt} sau noduri de interconectare ale Internetului. Aceste încercări presupun muncă de cercetare, standardizare și demonstraţii de concept - \gls{poc}, pentru prezentarea avantajelor pe care această nouă soluție le oferă, până când tehnologia se va maturiza și va fi adoptată de toată industria reţelisticii.