\section{Prezentarea domeniului tezei de doctorat}

În urma dezvoltării tehnologice recente în toate domeniile, în general și în domeniul calculatoarelor și al telecomunicaţiilor, în particular, a apărut nevoia de a redefini arhitectura reţelelor de comunicaţii, din cauza faptului că reţelele tradiţionale au început să îşi arate limitele. În zilele noastre, există o tendinţă de a interconecta toate echipamentele, cu ajutorul unor tehnologii care permit acest lucru, cum ar fi arhitectura \textit{Cloud Computing}, mobilitatea, sau idei mai noi, cum ar fi \textit{Internetul Tuturor Lucrurilor} - \gls{iot} sau sistemele de comunicaţii de generaţia a cincea - 5G. Aceste noi abordări au nevoie, pe lângă o lăţime de bandă crescută, de o reţea mai simplă și agilă, unde se facilitează inovarea. Reţelele definite prin software - \gls{sdn}, reprezintă o nouă paradigmă care a apărut în industria reţelisticii, pentru a mitiga dezavantajele pe care reţelele tradiţionale le-au dovedit.

Tehnologia \gls{sdn} nu este încă matură și nu a pătruns în toate tipurile de reţele de comunicaţii. Este prezentă în campusuri universitare, sau în centre de date, însă se încearcă introducerea acesteia în toate aspectele unei reţele de comunicaţii, cum ar fi transportul de date optic - \gls{ot}, transportul de date fără fir - \gls{wt} sau noduri de interconectare ale Internetului. Aceste încercări presupun muncă de standardizare și demonstraţii de concept - \gls{poc}, pentru prezentarea avantajelor pe care această nouă paradigmă de oferă, până când tehnologia se va maturiza și va fi adoptată de toată industria reţelisticii.