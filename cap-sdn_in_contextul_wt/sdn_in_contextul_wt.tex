\chapter{Unelte SDN \^{\i}n contextul reţelelor de transport de date fără fir\label{ch:sdn_in_contextul_wt}}

\graphicspath{ {cap-sdn_in_contextul_wt/figures/} }

Așa cum a fost prezentat în capitolul anterior, o mare parte din cercetarea și implementările \gls{sdn} se bazează pe protocolul OpenFlow. Acesta, însă, nu poate fi utilizat în orice aspect al unei rețele (de exemplu în rețelele de transport de date fără fir). În \cite{onf2015_poc1} s-a demonstrat faptul că se poate extinde protocolul OpenFlow pentru a cuprinde atribute specifice echipamentelor de transport de date fără fir, însă s-a ajuns la concluzia că este totuşi nevoie de un model informațional care să abstractizeze astfel de echipamente pentru a facilita administrarea acestora prin aplicații software. 

Astfel, grupurile de lucru din \gls{onf} au formulat recomandări pentru astfel de modele informaționale care să poată fi aplicate în acest context. În martie 2015 a fost publicată de către \gls{onf} prima versiune (1.0) a modelului informațional de bază, TR-512, \cite{onftr512v1.0}, apoi în noiembrie 2015 versiunea 1.1 \cite{onftr512v1.1}, ca în septembrie 2016 să fie publicată versiunea curentă, 1.2, purtând numele TR-512.1 \cite{onftr512v1.2}. Modelul informațional de bază este doar un schelet, care poate fi folosit în toate tipurile de rețele de transport, indiferent de natura acestora. Pentru rețelele de transport de date fără fir, \gls{onf} a publicat în decembrie 2016 și modelul informațional pentru microunde \cite{onftr532}, pentru abstractizarea echipamentelor din acest tip de rețele, care este integrat cu TR-512.1. 

În următoarele secţiuni aceste modele vor fi detaliate, pentru a putea mai apoi înţelege arhitectura simulatoarelor dezvoltate în această lucrare. Apoi va fi prezentat protocolul \gls{netconf} și modul în care utilizează aceste modele informaționale, precum și alegerea unui cadrul software cu sursă deschisă care să implementeze un server pentru acest protocol. În finalul capitolului se va prezenta arhitectura demonstraţiilor de concept desfăşurate în cadrul \gls{onf}, pentru o înţelegere mai bună a motivaţiei din spatele creării simulatoarelor care fac scopul acestei lucrări.

\section{Modelul informaţional de bază - ONF TR-512}

SDN și reţelele actuale..
\section{ONF TR-532 - Modelul informaţional pentru microunde (\textit{Microwave Information Model})}

Modelul informațional pentru microunde \cite{onftr532} a apărut in decembrie 2016 ca o recomandare formulată de grupul \gls{otwg} din cadrul \gls{onf}. Scopul acestuia este de a modela un echipament de transport de date fără fir, pentru a putea fi folosit de echipamentele de control \gls{sdn}, în încercarea de a asigura o independenţă față de producătorii de echipamente. Chiar dacă este denumit \textit{model informațional pentru microunde}, acesta poate fi aplicat fără probleme nu numai echipamentelor ce funcționează în spectrul microundelor, ci și echipamentelor care funcționează în benzi de frecvenţă mai înalte (lungimi de undă milimetrice), care încep să își facă tot mai mult simţită prezenţa în rețelele actuale de transport.

TR-532 este de fapt o extensie specifică tehnologiei \gls{wt} a modelului informațional de bază, versiunea 1.2 (TR-512.1). Legătura cu acesta se face prin extinderea clasei de obiecte \textit{\gls{lp}}. Astfel, modelul informațional pentru microunde conţine şase pachete condiţionale caracteristice tehnologiilor folosite pentru transport, care au în nume extensia \textit{*\_Pac}: 

\begin{itemize}
	\item \textit{MW\_AirInterface\_Pac};
	\item \textit{MW\_AirInterfaceDiversity\_Pac};
	\item \textit{MW\_PureEthernetStructure\_Pac};
	\item \textit{MW\_HybridMWStructure\_Pac};
	\item \textit{MW\_EthernetContainer\_Pac};
	\item \textit{MW\_TdmContainer\_Pac}
\end{itemize}

O imagine de ansamblu simplificată a acestui model, în limbajul \gls{uml}, care conţine doar obiectele relevante pentru simulatoarele dezvoltate, împreună cu legătura acestuia cu modelul informațional de bază este ilustrată în Figura \ref{fig:microwave_model}.

\begin{figure}[h]
	\centering
	\includegraphics[width=1\textwidth]{microwave_model_overview}
	\caption{Reprezentare UML simplificată a \textit{MicrowaveModel} și legătura acestuia cu \textit{CoreModel} \cite{onftr532}.}
	\label{fig:microwave_model}
\end{figure}

În următoarele paragrafe se vor detalia obiectele acestui model care sunt importante din punctul de vedere al simulatoarelor dezvoltate în această lucrare.

\subsection{Obiectul \textit{MW\_AirInterface\_Pac}}

Obiectul \textit{MW\_AirInterface\_Pac} reprezintă o interfață radio fizică a unui echipament. Este denumit în recomandare ca \textit{punct de terminaţie a traseului secţiunii fizice de microunde} - \gls{mwps-ttp}, astfel că nivelul de transport al obiectului \textit{\gls{lp}} asociat este Secţiunea Fizică de Microunde - \textit{\gls{mwps}} \cite{onftr532}. O reprezentare simplificată în limbajul \gls{uml} a \textit{MW\_AirInterface\_Pac} se poate observa în Figura \ref{fig:airinterface_pac}.

\begin{figure}[h]
	\centering
	\includegraphics[width=1\textwidth]{airinterface_pac}
	\caption{Reprezentare UML simplificată a obiectului \textit{MW\_AirInterface\_Pac} \cite{onftr532}.}
	\label{fig:airinterface_pac}
\end{figure}

Acest obiect conţine alte câteva obiecte care modelează caracteristice unei interfețe radio fizice, cum ar fi: (i) capabilităţi ale modemului și ale transmiţătorului interfeţei radio asociate (de exemplu tipurile de modulaţie suportate pentru transmisie, valorile admisibile ale puterii de transmisie, intervalul de frecvenţe suportate de emiţător sau de receptor, alarmele expuse de interfață, suportul interfeţei pentru modulaţie adaptivă, etc.), (ii) parametrii configurabili ai interfeţei radio (de exemplu numele interfeţei, lărgimea de bandă a canalului de transmisie/de recepţie, frecvenţele folosite pentru transmisie/recepţie, puterea de transmisie, intervalul în care modulaţia poate lua valori, diferite alte caracteristici configurabile ale interfeţei, cum ar fi Anularea Interferenţei dintre Polarizări - \gls{xpic}, Intrări Multiple - Ieşiri multiple - \gls{mimo}, criptarea datelor, etc.), (iii) parametrii care descriu starea interfeţei la un anumit moment de timp (de exemplu frecvenţele actuale de transmisie/recepţie, nivelurile actuale de putere a semnalului de transmisie/recepţie, modulaţia actuală folosită, raportul semnal-zgomot măsurat de către modem, temperatura actuală a unităţii radio, etc.), (iv) problemele actuale ale interfeţei radio (adică alarmele care apar pe interfață la un moment dat), (v) valorile actuale ale parametrilor de performanţă a interfeţei și (vi) valorile istorice ale parametrilor de performanţă a interfeţei \cite{onftr532}.

\subsection{Obiectul \textit{MW\_PureEthernetStructure\_Pac}}

Obiectul \textit{MW\_PureEthernetStructure\_Pac} este o reprezentare logică a unei interfețe radio capabilă să transporte doar trafic Ethernet. Acest obiect este reprezentat într-un mod simplificat, în limbajul \gls{uml}, în Figura \ref{fig:pureethstructure_pac}. Asocierea cu o interfață fizică radio se face la nivelul modelului informațional de bază, printr-o relaţie de tip client-server.

\begin{figure}[h]
	\centering
	\includegraphics[width=1\textwidth]{pureethstructure_pac}
	\caption{Reprezentare UML simplificată a obiectului \textit{MW\_PureEthernetStructure\_Pac} \cite{onftr532}.}
	\label{fig:pureethstructure_pac}
\end{figure}

Acest obiect este denumit în recomandare ca \textit{punct de terminaţie a traseului secţiunii microunde} - \gls{mws-ttp}, astfel că nivelul de transport al obiectului \textit{\gls{lp}} asociat este Secținea de Microunde - \textit{\gls{mws}}. 

Structura obiectelor conţinute de către \textit{MW\_PureEthernetStructure\_Pac} este similară cu cea a obiectului \textit{MW\_AirInterface\_Pac}. Conţine obiecte care reprezintă (i) capabilitățile acestei interfețe logice (de exemplu alarmele aplicabile ei sau identificatorul structurii respective, care poate fi folosit de alte obiecte), (ii) parametrii configurabili ai interfeţei logice (de exemplu gradul de severitate a alarmelor pe care această interfață le expune), (iii) parametrii care descriu starea interfeţei logice la un anumit moment de timp, (iv) problemele actuale ale interfeţei logice, (v) valorile actuale ale parametrilor de performanţă a interfeţei logice și (vi) valorile istorice ale parametrilor de performanţă a interfeţei logice \cite{onftr532}.

\subsection{Obiectul \textit{MW\_EthernetContainer\_Pac}}

Obiectul \textit{MW\_EthernetContainer\_Pac} reprezintă de asemenea o interfață logică și este denumit în recomandare \textit{punct de terminaţie a conexiunii unui client de microunde}, pentru un semnal Ethernet client. Practic, este o interfață logică ce are rol de container pentru traficul Ethernet care este transmis de echipament prin radio. În raport cu obiectul \textit{\gls{lp}} acesta are un nivel de transport denumit Container Ethernet - \textit{\gls{etc}}. O reprezentare grafică simplificată în limbajul \gls{uml} a obiectului \textit{MW\_EthernetContainer\_Pac} poate fi găsită în Figura \ref{fig:ethcontainer_pac}.

\begin{figure}[h]
	\centering
	\includegraphics[width=1\textwidth]{ethcontainer_pac}
	\caption{Reprezentare UML simplificată a obiectului \textit{MW\_EthernetContainer\_Pac} \cite{onftr532}.}
	\label{fig:ethcontainer_pac}
\end{figure}

Și în cazul obiectului \textit{MW\_EthernetContainer\_Pac} se păstrează aceeaşi structura a obiectelor pe care le conţine, ca în cazul celorlalte două obiecte detaliate anterior. Astfel, acesta prezintă obiecte care reprezintă (i) capabilitățile containerului (de exemplu dacă există compresie la diferite niveluri, criptare a datelor sau alarmele pe care această interfață le expune), (ii) parametrii configurabili ai containerului (de exemplu un identificator al containerului, identificatoarele segmentelor folosite pentru a transporta traficul Ethernet asociat acestui container, etc.), (iii) parametrii care descriu starea containerului, (iv) alarmele la momentul actual de timp pe care containerul le raportează, (v) valorile actuale ale parametrilor de performanţă a containerului și (vi) valorile istorice ale parametrilor de performanţă a containerului \cite{onftr532}.
\section{Protocolul NETCONF și limbajul YANG}

Network Configuration Protocol - \gls{netconf} - este un protocol de administrare a echipamentelor de rețea care oferă posibilitatea de a instala, manipula și şterge configuraţii ale echipamentelor de rețea. A fost prima oară definit în RFC 4741 \cite{rfc4741}, în decembrie 2006, ca mai apoi să fie revizuit în RFC 6241 \cite{rfc6241}, în iunie 2011. A apărut ca o nevoie a industriei rețelisticii, care până atunci utiliza alte modalităţi de administrare, precum \gls{cli}, \gls{snmp} sau \gls{corba} \cite{yu2010empirical}.

Chiar dacă tehnicile folosite până acum pentru administrarea echipamentelor sunt utilizate de mai bine de douăzeci de ani, acestea și-au dovedit limitările. De exemplu, prima abordare, cea folosind interfețe cu linie de comandă, implică o foarte mare dependenţă de producătorii de echipamente \cite{elbadawi2011improving}, în timp ce protocolul \gls{snmp} este folosit îndeosebi pentru colectarea de alarme și de valori ale parametrilor de performanţă de la echipamente și mai puţin pentru configurarea acestora. \gls{netconf} încearcă să diminueze aceste limitări propunând o abordare nouă și inovativă. În acest scop a fost dezvoltat și limbajul \gls{yang}, în RFC 6020 \cite{rfc6020}, care să modeleze datele folosite de acest protocol.

În următoarele secţiuni se vor detalia modul de funcţionare a protocolului \gls{netconf}, punând accent pe aspectele relevante din punctul de vedere al simulatoarelor dezvoltate în această lucrare și limbajul de modelare \gls{yang}.

\subsection{NETCONF}

Protocolul \gls{netconf} defineşte un mecanism simplu de administrare a echipamentelor de rețea, prin care parametrii configurabili pot fi ceruţi dispozitivelor, manipulaţi și apoi retrimiși pentru configurarea echipamentelor. Cu ajutorul acestuia dispozitivele pot expune interfețe de programare care să poată fi utilizate de aplicații software pentru administrare.

Mecanismul propus de \gls{netconf} constă în utilizarea paradigmei apelurilor de proceduri la distanţă - \gls{rpc}. Un server rulează pe fiecare dispozitiv, aşteptând conexiuni de la clienţi prin canale securizate și folosind protocoale orientate spre conexiune, precum \gls{ssh}, \gls{tls}, \gls{beep}, \gls{soap} \cite{ji2009challenges}. Clienţii își codează acel apel de procedură la distanţă cu ajutorul limbajului \gls{xml}, iar serverul răspunde printr-un mesaj codat cu același limbaj. În timpul stabilirii conexiunii între un client și un server \gls{netconf}, acesta din urmă își expune capabilitățile pe care le are, astfel încât clientul va fi informat asupra acestora, având posibilitatea de a-și schimba comportamentul pentru a profita de caracteristicile serverului.

\gls{netconf} propune folosirea a două tipuri de date: (i) date care descriu starea unui dispozitiv, adică valorile parametrilor operaţionali sau valori de contorizare a parametrilor de performanţă, toate acestea fiind valori care pot fi doar citite din echipamente și (ii) date configurabile pe dispozitiv, care pot fi atât scrise cât și citite. Pentru manipularea acestora se oferă diferite operații, cele mai importante fiind:

\begin{itemize}
	\item \textit{get} - această operaţie permite unui client să ceară unui dispozitiv atât date de stare, cât și parametri configurabili. Evident, se pot aplica filtre pentru a întoarce doar anumiţi parametri dintre cei pe care îi expune echipamentul;
	\item \textit{get-config} - această operaţie oferă unui client posibilitatea să ceară unui server doar parametri configurabili, excluzându-i pe cei de stare;
	\item \textit{edit-config} - aceasta este o operaţie prin care un client poate schimba valorile unor parametri configurabili pe un dispozitiv.
\end{itemize}

Pentru stocarea acestor date pe echipament, protocolul \gls{netconf} propune la nivel conceptual un loc în care să se stocheze și prin care să se acceseze informaţia, denumit \textit{datastore}. Reprezentarea acestuia este la alegerea fiecărui dispozitiv, putând fi implementat prin fişiere, baze de date, locaţii de memorie flash, etc. Există trei tipuri de astfel de locuri pentru stocarea datelor:

\begin{itemize}
	\item \textit{startup datastore} - acest tip reprezintă un \textit{datastore} ce poate conţine valori implicite ale parametrilor configurabili ai dispozitivului de rețea și poate fi încărcată în echipament în momentul inițializării acestuia. El este prezent doar în dispozitivele care suportă această separare între configurația curentă, care rulează în echipament și configurația care se încarcă în momentul inițializării;
	\item \textit{running datastore} - acest tip reprezintă toţi parametrii configurabili activi la momentul curent, care sunt prezenţi în dispozitiv. Acest tip există întotdeauna pe un echipament;
	\item \textit{candidate datastore} - acest tip reprezintă o copie a tuturor parametrilor configurabili ai unui dispozitiv. Modificarea acestora nu influenţează configurația curentă a dispozitivului, însă aceasta se poate aplica, prin copierea cu ajutorul unei tranzacţii, înlocuind \textit{running datastore}. Nu toate tipurile de echipamente suportă această capabilitate.
\end{itemize}

\gls{netconf} propune și un mecanism prin care să protejeze accesul concurent la scrierea parametrilor configurabili ai dispozitivelor. Astfel, un client poate bloca o parte sau chiar toată \textit{running datastore} cât timp execută operații prin care schimbă valorile acestor parametri, oferind astfel protecție datelor pe care le modifică. Dispozitivul însă trebuie să se asigure că aceste valori nu pot fi modificate în același timp prin alte căi, cum ar fi \gls{snmp} sau \gls{cli}.

Acest protocol oferă și posibilitatea serverelor \gls{netconf} să trimită informări către clienţi cu privire la anumite evenimente care se petrec în dispozitiv. Spre deosebire de celelalte mesaje, aceste notificări sunt iniţiate de către server și sunt trimise tuturor clienților și-au exprimat dorinţa de a le primi (prin trimiterea către server a unui mesaj de abonare). 

\subsection{YANG} 

Yet Another Next Generation - \gls{yang} este un limbaj de modelare a informaţiei dezvoltat specific pentru protocolul \gls{netconf}. Acesta descrie atât datele de configuraţie și de stare pe care un dispozitiv le poate expune pentru a fi folosite de către protocol, cât și apelurile de proceduri la distanţă sau notificările. A apărut în octombrie 2010 ca RFC 6020, fiind dezvoltat de \gls{ietf}.

În momentul apariției \gls{netconf}, dat fiind faptul că se baza pe limbajul \gls{xml}, soluţia naturală pentru definirea modelului de date folosit de protocol era utilizarea soluţiilor \gls{xml} existente pentru modelarea informaţiei, precum schemele \gls{xml} sau Relax NG \cite{ji2009challenges}. Aceste soluții aveau însă dezavantajele de a fi greu de folosit și de a avea o lizibilitate redusă.

Astfel, un nou limbaj a fost dezvoltat: \gls{yang}. Au fost considerate mai multe aspecte în dezvoltarea acestuia: lizibilitatea, o abordare orientată pe obiecte și o oarecare similaritate cu limbajele de programare. \gls{yang} îndeplineşte aceste condiţii, fiind folosit pentru a descrie ierarhii de noduri, care pot reprezenta notificări, apeluri de proceduri la distanţă sau parametri de stare sau de configuraţie și pot fi folosite de operaţiile \gls{netconf}. Informaţiile sunt stocate în modele \gls{yang} și precum în limbajele de programare, un model poate include date din alt model, oferind astfel posibilitatea de a crea seturi de modele de date generice și reutilizabile\cite{nataf2010end}. Acestea descriu atât nodurile într-un mod concis și clar, cât și interacţiunile dintre ele \cite{cui2009contrast}.

\gls{yang} descrie informaţia într-un mod ierarhic, astfel că un fiecare nod are, pe de o parte, un nume și pe de altă parte o valoare sau un set de noduri copil. Se pot descrie și constrângeri ce pot fi aplicate asupra apariției sau valorii unor noduri, bazându-se pe prezenţa sau valoarea altor noduri ale ierarhiei.

Există mai multe tipuri de noduri definite în limbajul \gls{yang}. Cele relevante din punctul de vedere al simulatoarelor implementate în această lucrare sunt:

\begin{itemize}
	\item \textit{grouping} - acesta este, așa cum o sugerează și numele, un nod care reprezintă o grupare de noduri. După ce este definit el poate fi utilizat în același sau în alte module sau sub-module;
	\item \textit{list} - acest tip de nod \gls{yang} defineşte o listă de noduri, iar intrările în această listă sunt distinse prin noduri care reprezintă cheia intrării respective;
	\item \textit{typedef} - este folosit pentru definirea unui tip de date care poate fi utilizat ulterior de alte noduri;
	\item \textit{rpc} - este folosit pentru modelarea apelurilor de proceduri la distanţă, prin definirea numelui procedurii și a parametrilor de intrare și de ieșire;
	\item \textit{notification} - acest tip de nod se folosește pentru descrierea unei notificări \gls{netconf} pe care un server o poate genera, prin modelarea conținutului acesteia;
	\item \textit{leaf} - nodurile frunză reprezintă nivelul cel mai jos al ierarhiei si descriu un parametru al dispozitivului, care poate fi de stare (poate fi doar citit) sau de configurare (poate fi și scris și citit).
\end{itemize}

Așa cum a fost prezentat anterior, \gls{onf} dezvoltă modelele informaționale pe care le recomandă cu ajutorul limbajului \gls{uml}, care este mult mai general si mai puţin specializat decât \gls{yang}. Însă, pentru a putea fi folosite într-un mod facil, \gls{onf} a dezvoltat și o unealtă software care să transforme modelele din limbajul \gls{uml} în limbajul \gls{yang}, împreună cu o recomandare despre cum această transformare ar trebui făcută \cite{onftr531}.
\section{Alegerea unui cadru software pentru serverul NETCONF}

Există numeroase soluții care lucrează cu protocolul \gls{netconf}, atât pe partea de client, cât și pe cea de server. O listă a acestora este menţinută de către grupul de lucru \gls{netconf} și poate fi găsită online \cite{netconfwiki}. Conţine și soluții software proprietare, dar și soluții cu sursă deschisă. Pentru implementarea simulatoarelor prezentate în această lucrare au fost considerate trei opţiuni de implementare a unui server \gls{netconf}, cu sursă deschisă: \textit{Netopeer}, \textit{OpenYuma} și \textit{\gls{netconf} Test Tool} (unealtă oferită de proiectul \gls{odl}). O comparaţie între acestea va fi prezentată în continuare, justificând astfel alegerea de a folosi una dintre ele în simulatoare \cite{stancu2016comparison}.

\subsection{Netopeer}

\textit{Netopeer} este o soluție ce se bazează pe librăria \textit{libnetconf}, oferind atât o implementare pentru server, cât și una pentru client. Această librărie este una cu sursă deschisă, implementată în limbajul C, ce oferă o implementare a protocolului \gls{netconf} \cite{krejci2013building}. Este o soluție care poate fi personalizată, oferind numeroase posibilităţi pentru implementările de server și de client și suportă toate caracteristicile protocolului \gls{netconf}.

\textit{Netopeer} oferă câteva unelte, de exemplu pentru a facilita integrarea modelelor informaționale \gls{yang} în module ale serverului, denumită \textit{netopeer-manager}, sau pentru a configura caracteristicile serverului \gls{netconf}, \textit{netopeer-configurator}. Orice model de date \gls{yang} poate fi adăugat ca un modul al serverului, însă acesta trebuie prelucrat înainte. Astfel, fişierul \textit{*.yang} este transformat de către această soluție în fişiere pe care serverul le poate recunoaşte, inclusiv un fişier \textit{*.c}, care conţine un schelet de cod C, reprezentând așa-numite funcții de apel invers (\textit{callback functions}) ce pot fi implementate pentru ca serverul să ofere comportamentul dorit, în raport cu modelul \gls{yang} folosit. Apoi, codul C este compilat, rezultând o bibliotecă partajată (\textit{shared library}) care poate fi utilizată de către codul de bază al serverului.

\subsection{OpenYuma}

\textit{OpenYuma} este o soluție software care se bazează pe proiectul \textit{Yuma}, care a devenit proprietar în anul 2011. Propune de asemenea implementări pentru server și client \gls{netconf}, scrise în limbajul C, oferind chiar posibilitatea de a încorpora acest cod în dispozitive al căror software folosește tot limbajul C.

\textit{OpenYuma} are o filosofie asemănătoare cu \textit{Netopeer}, oferind unelte pentru transformarea modelelor \gls{yang} în cod C schelet, care să fie apoi implementat pentru ca serverul \gls{netconf} să ofere facilităţile propuse. Unealta propusă de acest cadru software se numeşte \textit{yangudmp} și transformă fişierele \textit{*.yang} în fişiere \textit{*.h} și \textit{*.c}, conţinând, la fel ca în cazul \textit{Netopeer}, funcții de apel invers ce trebuie rescrise.

Codul C obţinut după transformarea modelelor \gls{yang} se compilează, rezultând tot o bibliotecă partajată care să poată fi folosită de către codul de cază al serverului. Aceasta poate fi încărcată în momentul inițializării serverului sau chiar în mod dinamic, în timp ce acesta rulează.

\subsection{Netconf Test Tool}

\textit{Netconf Test Tool} este un cadru software oferit în cadrul proiectului OpenDaylight. Este o soluție simplă, care nu poate fi personalizată foarte mult, oferind doar o implementare Java pentru un server \gls{netconf}. Aceasta este folosită de proiectul \gls{odl} pentru a-și testa interfața de tip \textit{southbound} care implementează protocolul \gls{netconf}.

Scopul acestei soluții este puțin diferit de al celorlalte, deoarece \textit{Netconf Test Tool} nu își propune oferirea unui cadru software pentru un server \gls{netconf} care apoi să poată fi integrat cu echipamentele de rețea, ci oferirea unei soluții simple și rapide care să încarce un model \gls{yang} specific, cu scopul de a-l testa. Cu toate acestea, acest software a fost considerat pentru comparaţie, deoarece și scopul simulatoarelor este de a testa modelele \gls{yang} și de a crea topologii specifice rețelelor de transport de date fără fir, expunând modelele informaționale descrise anterior.

\subsection{Comparaţie între soluţiile care oferă server NETCONF}

Soluţiile descrise anterior au fost evaluate atât prin compararea documentaţiei relevante pe care acestea o pun la dispoziţie, cât și prin experimente practice care consideră diferite scenarii. Primul criteriu care poate fi considerat este limbajul pe programare în care aceste cadre software sunt implementate: \textit{Netopeer} și \textit{OpenYuma} sunt scrise în limbajul C, pe când \textit{Netconf Test Tool} este o implementare Java.

Un alt criteriu pentru evaluare constă în abilitatea serverului de a încărca în mod direct (dinamic sau în momentul inițializării) modele \gls{yang}. Această posibilitate este oferită doar de implementarea Java. Celelalte două soluții au nevoie de o fază premergătoare de procesare,transformând fişierele \textit{*.yang} în \textit{*.c}. \textit{Netconf Test Tool} poate încărca modelul \gls{yang} doar în momentul inițializării serverului, dintr-un director anume, pe când celelalte cadre software pot încărca acest model, după ce a fost procesat, în mod dinamic.

Un alt subiect pentru comparaţie este dat de tipurile de locuri de stocare a datelor (\textit{datastore}) propuse de protocolul \gls{netconf} suportate de implementările serverelor. \textit{Netopeer} și \textit{OpenYuma} folosesc fişiere \gls{xml} în care stochează informaţiile și suportă toate cele trei tipuri de \textit{datastore} propuse de \gls{netconf}: \textit{startup}, \textit{candidate} și \textit{runnning}. Cealaltă soluție software, \textit{Netconf Test Tool} oferă posibilitatea de a utiliza doar \textit{running datastore} stocând valorile parametrilor în variabile de execuţie, acestea pierzându-se în momentul în care serverul este oprit. O diferenţă importantă apare în acest context, între cele două soluții implementate în C, \textit{OpenYuma} oferind o flexibilitate mai mare. În cazul \textit{Netopeer}, atunci când serverul se iniţializează, încarcă valorile parametrilor din \textit{startup datastore} și \textit{running datastore} în memorie. Apoi, începe să analizeze valorile din \textit{startup datastore}, comparându-le cu valorile corespunzătoare atributelor din \textit{running datastore}. Dacă valorile nu sunt egale, sau valoarea din \textit{running datastore} nu există (însemnând prima utilizare a serverului), atunci serverul copiază valoarea din \textit{startup datastore} și apelează funcţia de apel invers asociată parametrului de configurare. Prin această abordare severul se asigură ca nu există inconsistenţe între \textit{startup datastore} și \textit{running datastore} și, mai mult, dacă acest server este conectat la un echipament real de rețea, prin apelarea funcţiei asociate parametrului, dispozitivul va fi configurat astfel încât valorile din server să reflecte valorile de pe echipament. \textit{OpenYuma} are o abordare mai flexibilă, permiţând dezvoltatorilor să altereze \textit{running datastore} în timpul inițializării modulului, fără a implica \textit{startup datastore}.

O altă caracteristică importantă a serverelor care poate fi comparată este abilitatea de a genera notificări \gls{netconf}. Toate cele trei soluții oferă notificări. În cazul \textit{Netconf Test Tool} este ceva simplist, acestea fiind declanşate printr-o comandă \gls{netconf} și trimise dintr-un fişier \gls{xml} care le conţine. \textit{OpenYuma} oferă câte o funcție cu apel invers pentru fiecare notificare pe care o găseşte în momentul procesării modelului \gls{yang}. \textit{Netopeer} nu are această abilitate în mod implicit, însă suportul este oferit de \textit{libnetconf}.

Din perspectiva funcţiilor cu apel invers generate în momentul procesării modelului \gls{yang} putem compara doar cele două soluții implementate în limbajul C, deoarece \textit{Netconf Test Tool} nu oferă astfel de caracteristici. Soluția \textit{Netopeer} oferă posibilitatea de a alege care dintre parametrii configurabili vor avea o funcție cu apel invers asociată, în timp ce \textit{OpenYuma} va genera o astfel de funcție pentru fiecare din parametrii configurabili pe care îi găseşte în modelul \gls{yang}. Pentru informaţiile de stare \textit{Netopeer} generează o singură funcție cu apel invers care este apelată pentru orice operaţie \gls{netconf} de tip \textit{get} care ajunge la server, iar \textit{OpenYuma} generează câte o astfel de funcție pentru fiecare parametru de stare.

Din punctul de vedere al abilităţii de a configura portul pe care serverul ascultă, toate soluțiile oferă această posibilitate. \textit{Netconf Test Tool} are nevoie de un port de plecare, incrementând apoi numărul portului pentru celelalte instanţe ale serverului. Și celelalte soluții oferă posibilitatea de a schimba portul pe care serverul ascultă. Din perspectiva rulării mai multor instanţe de server pe aceeaşi mașină, atât \textit{Netconf Test Tool} cât și \textit{OpenYuma} oferă această posibilitate, diferitele instanţe folosind porturi diferite. \textit{Netopeer}, pe de altă parte, nu permite rularea mai multor instanţe pe aceeaşi mașină.

Un alt criteriu pentru compararea acestor cadre software este dat de suportul pentru mai multe fire de execuție, adică abilitatea serverului de a permite conectarea mai multor clienţi în același timp. \textit{Netconf Test Tool} oferă această posibilitate. \textit{OpenYuma} oferă de asemenea acest suport, cu menţiunea că accesul concurent la resursele comune trebuie rezolvat în funcțiile cu apel invers care vor fi implementate de către dezvoltatori, nefiind rezolvat de către cadrul software. În cazul \textit{Netopeer}, acest acces concurent este rezolvat din faza de proiectare a serverului și este oferit de \textit{libnetconf}.

Și capabilitățile de depanare oferite ar putea constitui un criteriu de comparaţie, însă toate soluţiile se bazează pe fişiere de tip jurnal, oferind mai multe niveluri de jurnalizare. 

Comparaţia bazată pe lucrurile descrise în documentaţie este rezumată în Tabelul \ref{tab:Table_1}, în timp ce un sumar al comparaţiei bazată pe experimentare se găseşte în Tabelul \ref{tab:Table_2}.

\begin{table}[tp]
	\caption{Comparaţie a caracteristicilor oferite de cadrele software considerate.\label{tab:Table_1}}
	
	\begin{tabular}{|M{0.35\textwidth}|M{0.17\textwidth}|M{0.17\textwidth}|M{0.16\textwidth}|}
			\hline 
			\textbf{Criteriile} & \multicolumn{3}{c|}{\textbf{Soluții servere NETCONF}} \tabularnewline
			\cline{2-4} 
			\textbf{de comparație} & \textbf{\emph{Netopeer}} & \textbf{\emph{OpenYuma}} & \textbf{\emph{Testtool}}\tabularnewline
			\hline 
			Limbajul de programare & C & C & Java\tabularnewline
			\hline 
			Încărcarea modelelor YANG brute & Nu & Nu & Da\tabularnewline
			\hline 
			Încărcarea dinamică a modulelor în server & Da & Da & Nu\tabularnewline
			\hline 
			NETCONF \textit{datastore} & toate & toate & \textit{running} \tabularnewline
			\hline 
			Suport pentru notificări & da & da & da\tabularnewline
			\hline 
			Port configurabil & da & da & da\tabularnewline
			\hline 
			Mai multe instanțe de server & nu & da & da \tabularnewline
			\hline 
			Mai multe conexiuni în același timp & da & da & da\tabularnewline
			\hline 
			Capabilități pentru depanare & jurnalizare & jurnalizare & jurnalizare\tabularnewline
			\hline
		\end{tabular}
\end{table}

\begin{table}[tp]

	\caption{Compararea practică a cadrelor software considerate\label{tab:Table_2}}
	\begin{tabular}{|M{0.35\textwidth}|M{0.17\textwidth}|M{0.17\textwidth}|M{0.16\textwidth}|}
		\hline
		\textbf{Scenariul experimentat} & \textbf{\emph{Netopeer}} & \textbf{\emph{OpenYuma}} & \textbf{\emph{Testtool}} \tabularnewline
		\hline 
		Procesarea \textit{*.yang} în \textit{*.c} & lnctool & yangdump & N/A\tabularnewline
		\hline 
		Reprezentarea \textit{datastore} în cadrul serverului & fişier XML & fișier XML & variabile de execuție \tabularnewline
		\hline 
		Încărcarea datelor în faza de iniţializare a serverului & \textit{startup} & flexibilă, orice se poate suprascrie & N/A \tabularnewline
		\hline 
		Implementarea notificărilor NETCONF & \textit{libnetconf} & funcție cu apel invers oferită & fișier XML \tabularnewline
		\hline 
		Funcții pentru parametrii configurabili & câte una per atribut & câte una per atribut & N/A\tabularnewline
		\hline 
		Funcții pentru parametrii de stare & doar una, globală & câte una per atribut & N/A\tabularnewline
		\hline 
		Conexiuni concurente de la clienți & da & suportă, trebuie implementat & da\tabularnewline
		\hline \end{tabular}
\end{table}

\section{Arhitectura demonstraţiilor de concept WT SDN}

Așa cum a fost amintit anterior, proiectul \textit{Wireless Transport} din grupul Open Transport Working Grup din cadrul \gls{onf} se ocupă cu cercetare legată de \gls{sdn} în contextul rețelelor de transport de date fără fir. După cum este prezentat și în \cite{bercovich2015software, bernardos2014architecture}, acest tip de rețele sunt o parte importantă atât din perspectiva rețelelor curente, cât și din cea a rețelelor viitorului, precum cele 5G.

După efectuarea unei demonstraţii de concept \cite{onf2015_poc1} care folosea protocolul OpenFlow pentru a arăta capabilitățile \gls{sdn} în rețelele de transport de date fără fir, observând dificultăţile pe care le presupune lucrul cu OpenFlow în acest context, grupul a trecut la dezvoltarea unui model informațional care să abstractizeze dispozitivele ce fac parte din aceste rețele, model care a fost descris anterior.

Apoi, alte trei demonstraţii de concept au avut loc, descrise în \cite{onf2016_poc2, onf2016_poc3, onf2017_poc4}, care au urmărit evoluția acestui model informațional. Chiar dacă în cel de-al doilea \gls{poc} s-a folosit un model de date mult simplificat, iar în cel de-al patrulea a fost folosită cea mai bună și matură variantă a sa, arhitectura acestor demonstraţii a fost similară.

\begin{figure}[h]
	\centering
	\includegraphics{poc_architecture}
	\caption{Arhitectura demonstraţiilor de concept WT.}
	\label{fig:poc_architecture}
\end{figure}

După cum se poate observa în Figura \ref{fig:poc_architecture}, în arhitectura acestor demonstraţii apare nevoia unui nivel intermediar, de mediator. Acest mediator este reprezentat de o aplicație software care expune o interfață \gls{netconf} către nord, unde se conectează echipamentul de control \gls{sdn}, folosind modelul informațional descris de TR-532. Rolul mediatorului este de a traduce comenzile \gls{netconf} care vin de la controler în comenzi pe care dispozitivul de rețea le poate înţelege (de exemplu \gls{snmp}, \gls{cli} sau \gls{rest}). Această nevoie apare din cauza faptului că echipamentele din rețelele actuale, care au fost folosite și pentru demonstraţiile de concept, nu suportă încă noul model informațional în mod nativ, deoarece abia a fost dezvoltat. Cel mai probabil, în viitor această nevoie va dispărea, echipamentele putând să încadreze \textit{Microwave Model} în aplicaţia software proprie de control.

\begin{figure}[t]
	\centering
	\includegraphics{poc_architecture_simulator}
	\caption{Poziţionarea simulatoarelor în arhitectura demonstraţiilor de concept WT.}
	\label{fig:poc_architecture_simulator}
\end{figure}


Poziţionarea simulatoarelor în arhitectura acestor demonstraţii de concept este ilustrată în Figura \ref{fig:poc_architecture_simulator}. Astfel, ele pot fi folosite de către dezvoltatorii de aplicații \gls{sdn} pentru testarea aplicațiilor care utilizează interfaţa \gls{netconf} ce implementează modelul informațional TR-532, eliminând nevoia acestora de a deţine dispozitive pentru transportul de date fără fir și mediatorul asociat acestora.
