\section{Protocolul NETCONF și limbajul YANG}

Protocolul de Configurare a Rețelei - \textit{Network Configuration Protocol (\gls{netconf})} - este un protocol de administrare a echipamentelor de rețea care oferă posibilitatea de a instala, manipula și şterge configuraţii ale echipamentelor de rețea. A fost prima oară definit în RFC 4741 \cite{rfc4741}, în decembrie 2006, ca mai apoi să fie revizuit în RFC 6241 \cite{rfc6241}, în iunie 2011. A apărut ca o nevoie a industriei rețelisticii, care până atunci utiliza alte modalităţi de administrare, precum Interfața prin Linie de Comandă - \gls{cli}, Protocolul Simplu pentru Administrarea Rețelelor - \gls{snmp} sau Arhitectura Agentului care Solicită Obiecte Comune - \gls{corba} \cite{yu2010empirical}.

Chiar dacă tehnicile folosite până acum pentru administrarea echipamentelor sunt utilizate de mai bine de douăzeci de ani, acestea și-au dovedit limitările. De exemplu, prima abordare, cea folosind interfețe cu linie de comandă, implică o foarte mare dependenţă de producătorii de echipamente \cite{elbadawi2011improving}, în timp ce protocolul \gls{snmp} este folosit îndeosebi pentru colectarea de alarme și de valori ale parametrilor de performanţă de la echipamente și mai puţin pentru configurarea acestora. \gls{netconf} încearcă să diminueze aceste limitări propunând o abordare nouă și inovativă. În acest scop a fost dezvoltat și limbajul \gls{yang}, în RFC 6020 \cite{rfc6020}, care să modeleze datele folosite de acest protocol.

În următoarele secţiuni se vor detalia modul de funcţionare a protocolului \gls{netconf}, punând accent pe aspectele relevante din punctul de vedere al simulatoarelor dezvoltate în această lucrare și limbajul de modelare \gls{yang}.

\subsection{NETCONF}

Protocolul \gls{netconf} defineşte un mecanism simplu de administrare a echipamentelor de rețea, prin care parametrii configurabili pot fi ceruţi dispozitivelor, manipulaţi și apoi retrimiși pentru configurarea echipamentelor. Cu ajutorul acestuia dispozitivele pot expune interfețe diferite de programare care să poată fi utilizate de aplicații software pentru administrare.

Mecanismul propus de \gls{netconf} constă în utilizarea paradigmei Apelurilor de Proceduri la Distanţă - \gls{rpc}. Un server rulează pe fiecare dispozitiv, aşteptând conexiuni de la clienţi prin canale securizate și folosind protocoale orientate spre conexiune, precum Terminalul Securizat - \gls{ssh}, Securitatea Nivelului Transport - \gls{tls}, Protocolul de Interschimbare a Blocurilor Extensibile - \gls{beep}, Protocolul de Acces cu Obiecte Simple - \gls{soap} \cite{ji2009challenges}. Clienţii își codează acel apel de procedură la distanţă cu ajutorul Limbajului de Marcare Extensibilă - \gls{xml}, iar serverul răspunde printr-un mesaj codat cu același limbaj. În timpul stabilirii conexiunii între un client și un server \gls{netconf}, acesta din urmă își expune capabilitățile pe care le are, astfel încât clientul va fi informat asupra acestora, având posibilitatea de a-și schimba comportamentul pentru a profita de caracteristicile serverului.

\gls{netconf} propune folosirea a două tipuri de date: (i) date care descriu starea unui dispozitiv, adică valorile parametrilor operaţionali sau valori de contorizare a parametrilor de performanţă, toate acestea fiind valori care pot fi doar citite din echipamente și (ii) date configurabile pe dispozitiv, care pot fi atât scrise cât și citite. Pentru operarea acestora se oferă diferite comenzi, cele mai importante fiind:

\begin{itemize}
	\item \textit{get} - această operaţie permite unui client să ceară unui dispozitiv atât date de stare, cât și parametri configurabili. Evident, se pot aplica filtre pentru a întoarce doar anumiţi parametri dintre cei pe care îi expune echipamentul;
	\item \textit{get-config} - această operaţie oferă unui client posibilitatea să ceară unui server doar parametri configurabili, excluzându-i pe cei de stare;
	\item \textit{edit-config} - aceasta este o operaţie prin care un client poate schimba valorile unor parametri configurabili pe un dispozitiv.
\end{itemize}

Pentru stocarea acestor date pe echipament, protocolul \gls{netconf} propune la nivel conceptual un loc în care să se stocheze și prin care să se acceseze informaţia, denumit \textit{bază de stocare a datelor - datastore}. Reprezentarea acestuia este la alegerea fiecărui dispozitiv, putând fi implementat prin fişiere, baze de date, locaţii de memorie flash, etc. Există trei tipuri de astfel de locuri pentru stocarea datelor:

\begin{itemize}
	\item \textit{baza de stocare a datelor de iniţializare} - acest tip reprezintă o bază de stocare a datelor ce poate conţine valori implicite ale parametrilor configurabili ai dispozitivului de rețea și poate fi încărcată în echipament în momentul inițializării acestuia. El este prezent doar în dispozitivele care suportă această separare între configurația curentă, care rulează în echipament și configurația care se încarcă în momentul inițializării;
	\item \textit{baza de stocare a datelor de operare} - acest tip reprezintă toţi parametrii configurabili activi la momentul curent, care sunt prezenţi în dispozitiv. Acest tip există întotdeauna pe un echipament;
	\item \textit{baza de stocare a datelor de rezervă} - acest tip reprezintă o copie a tuturor parametrilor configurabili ai unui dispozitiv. Modificarea acestora nu influenţează configurația curentă a dispozitivului, însă aceasta se poate aplica, prin copierea cu ajutorul unei tranzacţii, înlocuind baza de stocare a datelor de operare. Nu toate tipurile de echipamente suportă această capabilitate.
\end{itemize}

\gls{netconf} propune și un mecanism prin care să protejeze accesul concurent la scrierea parametrilor configurabili ai dispozitivelor. Astfel, un client poate bloca o parte sau chiar toată baza de stocare a datelor de operare cât timp execută operații prin care schimbă valorile acestor parametri, oferind astfel protecție datelor pe care le modifică. Dispozitivul însă trebuie să se asigure că aceste valori nu pot fi modificate în același timp prin alte căi, cum ar fi \gls{snmp} sau \gls{cli}.

Acest protocol oferă și posibilitatea serverelor \gls{netconf} să trimită informări către utilizatori cu privire la anumite evenimente care se petrec în dispozitiv. Spre deosebire de celelalte mesaje, aceste notificări sunt iniţiate de către server și sunt trimise tuturor utilizatorilor și-au exprimat dorinţa de a le primi (prin trimiterea către server a unui mesaj de abonare). 

\subsection{YANG} 

Încă o Generație Următoare - \textit{Yet Another Next Generation - \gls{yang}} este un limbaj de modelare a informaţiei dezvoltat specific pentru protocolul \gls{netconf}. Acesta descrie atât datele de configuraţie și de stare pe care un dispozitiv le poate expune pentru a fi folosite de către protocol, cât și apelurile de proceduri la distanţă sau notificările. A apărut în octombrie 2010 ca RFC 6020, fiind dezvoltat de \gls{ietf}.

În momentul apariției \gls{netconf}, dat fiind faptul că se baza pe limbajul \gls{xml}, soluţia naturală pentru definirea modelului de date folosit de protocol era utilizarea soluţiilor \gls{xml} existente pentru modelarea informaţiei, precum schemele \gls{xml} sau Relax NG \cite{ji2009challenges}. Aceste soluții aveau însă dezavantajele de a fi greu de folosit și de a avea o lizibilitate redusă.

Astfel, un nou limbaj a fost dezvoltat: \gls{yang}. Au fost considerate mai multe aspecte în dezvoltarea acestuia: lizibilitatea, o abordare orientată pe obiecte și o oarecare similaritate cu limbajele de programare. \gls{yang} îndeplineşte aceste condiţii, fiind folosit pentru a descrie ierarhii de noduri, care pot reprezenta notificări, apeluri de proceduri la distanţă sau parametri de stare sau de configuraţie și pot fi folosite de operaţiile \gls{netconf}. Informaţiile sunt stocate în modele \gls{yang} și precum în limbajele de programare, un model poate include date din alt model, oferind astfel posibilitatea de a crea seturi de modele de date generice și reutilizabile\cite{nataf2010end}. Acestea descriu atât nodurile într-un mod concis și clar, cât și interacţiunile dintre ele \cite{cui2009contrast}.

\gls{yang} descrie informaţia într-un mod ierarhic, astfel că un fiecare nod are, pe de o parte, un nume și pe de altă parte o valoare sau un set de noduri copil. Se pot descrie și constrângeri ce pot fi aplicate asupra apariției sau valorii unor noduri, bazându-se pe prezenţa sau valoarea altor noduri ale ierarhiei.

Există mai multe tipuri de noduri definite în limbajul \gls{yang}. Cele relevante din punctul de vedere al simulatoarelor implementate în această lucrare sunt:

\begin{itemize}
	\item \textit{grouping} - acesta este, așa cum o sugerează și numele, un nod care reprezintă o grupare de noduri. După ce este definit el poate fi utilizat în același sau în alte module sau sub-module;
	\item \textit{list} - acest tip de nod \gls{yang} defineşte o listă de noduri, iar intrările în această listă sunt identificate prin noduri care reprezintă cheia intrării respective;
	\item \textit{typedef} - este folosit pentru definirea unui tip de date care poate fi utilizat ulterior de alte noduri;
	\item \textit{rpc} - este folosit pentru modelarea apelurilor de proceduri la distanţă, prin definirea numelui procedurii și a parametrilor de intrare și de ieșire;
	\item \textit{notification} - acest tip de nod se folosește pentru descrierea unei notificări \gls{netconf} pe care un server o poate genera, prin modelarea conținutului acesteia;
	\item \textit{leaf} - nodurile frunză reprezintă nivelul cel mai jos al ierarhiei si descriu un parametru al dispozitivului, care poate fi de stare (poate fi doar citit) sau de configurare (poate fi și scris și citit).
\end{itemize}

Așa cum a fost prezentat anterior, \gls{onf} dezvoltă modelele informaționale pe care le recomandă cu ajutorul limbajului \gls{uml}, care este mult mai general si mai puţin specializat decât \gls{yang}. Însă, pentru a putea fi folosite într-un mod facil, \gls{onf} a dezvoltat și o unealtă software care să transforme modelele din limbajul \gls{uml} în limbajul \gls{yang}, împreună cu o recomandare despre cum această transformare ar trebui făcută \cite{onftr531}.