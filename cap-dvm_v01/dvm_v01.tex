\chapter{Mediatorul cu valori implicite (DVM)\label{ch:dvm_v01}}

\graphicspath{ {cap-dvm_v01/figures/} }

Nivelul mediator a apărut în arhitectura demonstraţiilor de concept, cum a fost prezentat anterior, deoarece dispozitivele de rețea încă nu au suport pentru modelul informațional pentru microunde TR-532. Astfel, această aplicație software trebuie să traducă operaţiile \gls{netconf} care vin de la echipamentul de control într-un limbaj care să fie recunoscut de echipamente. Fiecare producător de dispozitive care a participat la cea de-a doua demonstraţie de concept \gls{wt} \gls{sdn} a fost nevoit să implementeze un astfel de mediator.

Am dezvoltat mediatorul cu valori implicite (\textit{Default Values Mediator - \gls{dvm}}) special pentru cea de-a doua demonstrație de concept \gls{wt} a \gls{onf}, pentru a oferi posibilitatea dezvoltatorilor de aplicații \gls{sdn} care nu au acces la echipament de transport de date fără fir și la mediatoarele asociate acestora posibilitatea de a implementa și testa astfel de aplicații, care vor putea apoi interacţiona cu mediatoare reale, pentru că folosesc aceeaşi interfață \gls{netconf} care expune modelul informațional de bază și modelul informațional pentru microunde. Apoi, această implementare a evoluat în cea de-a doua versiune a \gls{dvm}, care a fost folosită pentru ce-a de-a doua demonstrație de concept.

\gls{dvm} este o implementare software cu sursă deschisă a unui server \gls{netconf}, bazându-se pe soluţia software \textit{OpenYuma}. Spre deosebire de mediatoarele reale, \textit{\gls{dvm}} are o singură interfață (cea prin care comunică cu controlerul \gls{sdn}), neavând nevoie de o interfață care să se lege la dispozitive reale de rețea. Acest simulator va returna valori implicite pentru atributele interogate de către controler.

\section{Arhitectura DVM versiunea 1}

Prima versiune a mediatorului cu valori implicite expune către echipamentul de control \gls{sdn} modelul informațional simplificat pentru microunde, care a fost folosit în cea de-a doua demonstraţie de concept efectuată de proiectul \gls{wt} din \gls{onf} \cite{stancu2016default}.

Implementarea acestuia este una simplă și modulară, bazată pe soluţia \textit{OpenYuma}, oferind câte o funcție cu apel invers pentru fiecare atribut din modelele \gls{yang} expuse, indiferent de natura lor (parametri de configurare sau de stare). În aceste funcții se stabileşte valoare atributului respectiv, care este returnată unui client \gls{netconf}. Excepţia de la această regulă este dată de câteva atribute ale căror valori sunt definite într-un fişier de configurare folosit de către mediator. Dezavantajul acestei abordări este că dacă un dezvoltator de aplicații are nevoie de o altă valoare a unui atribut care nu face parte din acest fişier de configurare, va trebui să o modifice în funcţia asociată parametrului respectiv și apoi să recompileze codul asociat modului de server care conţine acel parametru și apoi să încarce din nou acel modul în server.

O imagine de ansamblu a arhitecturii primii versiuni a \gls{dvm} este ilustrată în Figura \ref{fig:dvm_v01_architecture}. Acesta se bazează pe un fişier de configurare ce conţine câțiva parametri importanţi din punctul de vedere al aplicațiilor \gls{sdn}. Aceștia sunt: (i) numele echipamentului de rețea - \textit{Network Element Name} și (ii) un identificator unic folosit pentru fiecare legătură radio - \textit{Radio Signal ID}.

\begin{figure}[h]
	\centering
	\includegraphics{dvm_v01_architecture}
	\caption{Arhitectura primei versiuni a DVM \cite{stancu2016default}.}
	\label{fig:dvm_v01_architecture}
\end{figure}

Aceşti parametri sunt importanţi pentru aplicații deoarece cu ajutorul lor se pot identifica în mod unic dispozitivele din rețea. Prin pregătirea mai multor fişiere de configurare se pot simula mai multe elemente de rețea, fără a fi nevoie de recompilarea simulatorului. Astfel, vor rula mai multe instanţe ale \gls{dvm}, fiecare având propriul fişier de configurare ce conţine un alt nume pentru dispozitiv și alte identificatoare pentru legăturile radio.

În această arhitectură, \gls{dvm} este configurat să trimită și notificări \gls{netconf} fictive către utilizatorii care s-au abonat la primirea acestora. Fişierul de configurare conţine astfel și o valoare numerică reprezentând numărul de secunde dintre două notificări fictive consecutive. Dacă această valoare este mai mare decât zero, se vor trimite notificări fictive la intervalul de timp specificat. Altfel, dacă valoarea este zero, \gls{dvm} nu va trimite astfel de mesaje \gls{netconf}.
\section{Implementarea DVM versiunea 1}

Primul pas al implementării a fost reprezentat de alegerea modelelor \gls{yang} care să fie expuse de serverul \gls{netconf}. În cea de-a doua demonstraţie de concept \gls{wt} a fost agreată folosirea unor modele informaționale reduse, care să permită demonstrarea cazurilor de utilizare alese \cite{onf2016_poc2}. Acestea conţineau aproximativ şaizeci de atribute care făceau parte atât din modelul informațional de bază, cât și din modelul informațional pentru microunde. Astfel, au fost alese trei modele \gls{yang} pentru implementarea în cadrul primei versiuni a \gls{dvm}: \textit{CoreModel-CoreNetworkModule-ObjectClasses}, \textit{MicrowaveModel-ObjectClasses-MwConnection} și \textit{MicrowaveModel-Notifications}. Asta a însemnat, practic, generarea a trei biblioteci partajate reprezentând module ale serverului \gls{netconf}, care să poată fi încărcate în soluţia \textit{OpenYuma} și să ofere capabilitățile dorite.

O organigramă a fazei de dezvoltare și implementare a \gls{dvm} este ilustrată în Figura \ref{fig:dvm_v01_workflow}.

\begin{figure}[h]
	\centering
	\includegraphics{dvm_v01_workflow}
	\caption{Organigramă a dezvoltării și implementării DVM \cite{stancu2016default}.}
	\label{fig:dvm_v01_workflow}
\end{figure}

Cel de-al doilea pas al implementării a constat în procesarea modelelor \gls{yang} alese și generarea codului C schelet al modulelor asociate acestora. Acest lucru a fost realizat cu utilitarul \textit{yangdump} oferit de soluţia \textit{OpenYuma}. Pentru a îmbunătăţi flexibilitatea \gls{dvm} și pentru a avea o mai bună separare între codul folosit de server și codul utilizatorului, care trebuie rescris, utilitarul a fost folosit cu opţiunea \textit{--split}. Astfel, pentru fiecare modul au fost generate patru fişiere: câte unul \textit{.c} și \textit{.h} pentru codul de server, respectiv pentru cel de utilizator.

Următorul pas a fost reprezentat de implementarea funcţiilor cu apel invers generate pentru fiecare atribut al modelului \gls{yang}. Pentru a asocia câte o valoare implicită fiecărui parametru, funcțiile au fost modificate astfel încât să întoarcă valoarea respectivă în momentul apelării.

Cea mai importantă parte a implementării \gls{dvm} a constat în construirea bazei de stocare a datelor de operare, reprezentând de fapt arborele atributelor \gls{yang} pe care serverul \gls{netconf} le va utiliza atunci când va fi interogat de către echipamentul de control \gls{sdn}. Nu a fost posibilă construirea automată a acestui arbore de parametri, astfel că atributele au fost implementate manual, de la rădăcină către frunze. Pentru acest lucru a fost nevoie să se altereze funcţia de iniţializare \textit{init2()} generată automat pentru fiecare model \gls{yang}. A fost necesară adăugarea câte unui nod \textit{OpenYuma} pentru fiecare parametru \gls{yang}, având asociată o funcție cu apel invers în care se setează valoarea implicită a acelui atribut. În cazul parametrilor anterior menţionaţi, care fac parte din fişierul de configurare, implementarea funcţiei constă în citirea valorii respective din acel fişier. Această abordare a permis utilizatorilor \gls{dvm} un acces facil la valorile implicite asociate fiecărui atribut. Dacă un utilizator avea nevoie de schimbarea valorii unui parametru \gls{yang}, o putea face uşor prin funcţia cu apel invers asociată sau prin fişierul de configurare, fără să fie nevoit să ştie detaliile de implementare referitoare la construirea arborelui de valori.

Următorii paşi ai implementării sunt simpli și direcţi: codul rezultat se compilează, rezultând bibliotecile partajate care apoi sunt încărcate în serverul \gls{netconf} (mai exact în procesul \textit{netconfd} asociat acestuia).

Etapele anterior menţionate se aplică în cazul modelului informațional de bază și în cazul modelului informațional pentru microunde, excluzând cazul notificărilor \gls{netconf}, unde abordarea este puţin diferită.

Deoarece modelul asociat notificărilor \gls{netconf}, \textit{MicrowaveModel-Notifications}, are o structură diferită, conţinând obiecte \gls{yang} ce reprezintă notificări în locul atributelor obişnuite, comportamentul soluţiei \textit{OpenYuma} este diferit în acest caz. În loc să se genereze funcții cu apel invers pentru obţinerea și setarea valorilor atributelor, în cazul notificărilor soluţia \textit{OpenYuma} va genera funcții cu apel invers folosite pentru declanşarea acestora (trimiterea lor de către server tuturor utilizatorilor care s-au abonat).

Pentru implementarea notificărilor \gls{netconf} în \gls{dvm}, un nou fir de execuţie s-a creat în funcţia de iniţializare a modulului, \textit{init2()}. Acesta rulează o singură funcție care implementează o buclă infinită în care se folosește funcţia cu apel invers asociată pentru a trimite o notificare fictivă, la un interval de secunde definit în fişierul de configurare. În cazul în care valoarea intervalului este zero, declanşarea notificărilor nu va fi activată. Detaliile conţinute în notificarea \gls{netconf} fictivă se găsesc în interiorul funcţiei care implementează generarea și modificarea acestora nu este banală pentru un utilizator neexperimentat.

Structura fişierului de configurare este foarte simplă și nu oferă prea multă flexibilitate utilizatorilor \gls{dvm}. Aceasta este fixă și poate fi observată în Figura \ref{fig:dvm_v01_config}, într-un exemplu în care un dispozitiv conţine două interfețe radio. Conţine doar trei tipuri de parametri, așa cum a fost menţionat anterior: numele echipamentului de rețea (\textit{NeName}), identificatoarele legăturilor radio (\textit{radioSignalId} - există câte un identificator pentru fiecare interfață radio a dispozitivului) și intervalul de timp, exprimat în secunde, dintre două notificări fictive consecutive (\textit{eventFrequency}).

\begin{figure}[h]
	\centering
	\includegraphics{dvm_v01_config}
	\caption{Structura fişierului de configurare al DVM. Exemplu pentru un echipament cu două interfețe radio.}
	\label{fig:dvm_v01_config}
\end{figure}
\section{Folosirea în contextul demonstraţiilor de concept a DVM versiunea 1}

Prima versiune a mediatorului cu valori implicite a fost o unealtă foarte importantă în contextul celei de-a doua demonstraţii de concept a proiectului rețelelor de transport de date fără fir din cadrul \gls{onf}. Acesta a ajutat la accelerarea implementării aplicațiilor \gls{sdn} care au fost dezvoltate pentru cazurile de utilizare propuse în \cite{onf2016_poc2}.

\gls{dvm} a oferit interfaţa de Sud \gls{netconf} care expune modelele informaționale dezvoltate de \gls{onf}, în același mod în care ar expune-o un mediator real ce se conectează la echipamente de transport de date fără fir. În acest mod, dezvoltatorii aplicațiilor \gls{sdn} au putut utiliza acest simulator pentru implementarea și testarea acestora, fără a avea nevoie să deţină dispozitive de rețea, care au un preţ foarte ridicat.

Această primă versiune de simulator a accelerat activitățile de pregătire a celei de-a doua demonstraţii de concept, permiţând lucrul în paralel la aplicațiile \gls{sdn} și la dezvoltarea mediatoarelor. Interfaţa \gls{netconf} comună a putut fi testată înainte ca producătorii de echipamente să își implementeze mediatoarele, oferind astfel mai mult timp dezvoltatorilor aplicațiilor \gls{sdn} pentru depanarea programelor. De exemplu, generarea unei notificări \gls{netconf} se poate face mult mai facil cu ajutorul simulatorului. Pentru un mediator real, trebuie ca dispozitivul să fie făcut să genereze o notificare către mediator, prin interfaţa proprietară echipamentului respectiv, apoi mediatorul să traducă acel mesaj într-o notificare \gls{netconf}.

Figura \ref{fig:dvmv01_poc_usage} reprezintă elementele de bază ale demonstraţiei de concept, așa cum sunt prezentate în lucrarea apărută după desfăşurarea acestuia \cite{onf2016_poc2}, în care se poate vedea \gls{dvm}.

\begin{figure}[h]
	\centering
	\includegraphics[width=1\textwidth]{dvmv01_poc_usage}
	\caption{Configurarea rețelei de test SDN utilizând mașini virtuale \cite{onf2016_poc2}.}
	\label{fig:dvmv01_poc_usage}
\end{figure}

Înregistrarea la controlerul \gls{sdn} a primei versiuni a simulatoarelor \gls{dvm}, se face la fel ca pentru un mediator real. Astfel, echipamentul de control oferă o interfață de programare a aplicaţiei - \gls{api} - prin care o aplicație \gls{sdn} poate înregistra un astfel de mediator în controlerul \gls{sdn}. Înregistrarea nu este una automată, utilizatorul fiind nevoit să facă această înregistrare manual. În cea de-a doua demonstraţie de concept, acest lucru a fost făcut prin interfaţa grafică a echipamentului de control folosit (\gls{odl}). După înregistrare, controlerul stabileşte conexiunea \gls{netconf} cu mediatorul.

Codul asociat primei versiuni a \gls{dvm} este oferit cu sursă deschisă și se poate găsi în repertoriul asociat \gls{onf} de pe platforma GitHub, denumit CENTENNIAL \cite{dvmv01github}.
\section{Arhitectura DVM versiunea 2}

Cea de-a doua versiune a \gls{dvm} a fost implementată pentru a treia demonstraţie de concept a rețelelor de transport de date fără fir desfăşurată în cadrul \gls{onf}. A fost dezvoltată pe scheletul primei versiuni, păstrând aceeaşi abordare: un server \gls{netconf} care se bazează pe soluţia software \textit{OpenYuma} și un fişier de configurare din care serverul poate citi valori pentru atributele \gls{yang} pe care le expune. Din acest punct de vedere, \gls{dvm} versiunea 2 folosește întregul model informațional pentru microunde TR-532 și o parte semnificativă din modelul informațional de bază, TR-512.1, cumulând aproximativ 300 de astfel de parametri.

Arhitectura celei de-a doua versiuni a \gls{dvm} este influenţată de natura atributelor \gls{yang} ce alcătuiesc modelul informațional pentru microunde \cite{stancu2017enabling}: parametri de configurare, de stare sau care prezintă capabilitățile dispozitivului. Atributele de configurare pot fi citite sau scrise și prin intermediul acestora un utilizator poate influenţa comportamentul dispozitivului. Parametrii de stare pot fi doar citiţi și reprezintă situația curentă a echipamentului, în timp ce capabilitățile sunt atribute care pot fi doar citite și descriu abilităţile dispozitivului.

Mediatorul cu valori implicite a fost gândit astfel încât să fie flexibil și să ofere posibilitatea înlocuirii citirii valorilor din fişierul de configurare cu o citire dintr-un echipament real, cu ajutorul unui protocol la alegere. Deoarece mediatorul este o aplicație software externă, care nu face parte din echipamentele de transport de date fără fir, în momentul inițializării ar trebui să reflecte configurația echipamentului la care se conectează. Din cauza faptului că dispozitivul nu are o configuraţie statică și aceasta poate fi modificată înaintea inițializării mediatorului, nu poate fi utilizată baza de stocare de date de iniţializare oferită de serverul \gls{netconf}. Asta înseamnă ca în momentul inițializării mediatorul trebuie să își construiască baza de stocare de date de operare într-un mod arborescent, interogând dispozitivul de rețea asupra valorilor atributelor. Acest lucru se aplică doar în cazul parametrilor configurabili sau în cazul celor care au un caracter static, precum capabilitățile echipamentului. Deoarece aceste valori se salvează în baza de stocare de date de operare, înseamnă că acestea nu sunt persistente și vor fi citite la fiecare iniţializare din dispozitiv. Fiindcă simulatorul \gls{dvm} a fost proiectat ca un mediator real, toate aceste aspecte se reflectă și în arhitectura acestuia, diferenţa fiind că valorile atributelor nu se citesc dintr-un dispozitiv, ci dintr-un fişier de configurare descris în limbajul \gls{xml}.

Natura atributelor \gls{yang}, descrisă anterior, a dus la arhitectura celei de-a doua versiuni a \gls{dvm} care se poate vedea în Figura \ref{fig:dvm_v02_architecture}. Atributele \gls{yang} au fost împărţite în trei categorii: de iniţializare, de execuţie și de configurare.

\begin{figure}[h]
	\centering
	\includegraphics{dvm_v02_architecture}
	\caption{Arhitectura simulatorului DVM versiunea 2.}
	\label{fig:dvm_v02_architecture}
\end{figure}

Parametrii de iniţializare reprezintă acele atribute statice, care sunt citite din dispozitiv o singură dată, în momentul inițializării simulatorului. În cadrul serverului \gls{netconf} pot fi doar citite, în cazul parametrilor care reprezintă capabilitățile echipamentului (adică informaţie statică, prezentată de către dispozitiv, care nu se poate schimba) sau pot fi și scrise și citite, ca în cazul parametrilor de configurare. În momentul inițializării \gls{dvm} aceștia sunt consideraţi parametri de iniţializare, ca apoi să devină de configurare, având câte o funcție cu apel invers asociată, care permite modificarea acestei valori în fişierul \gls{xml} de configurare prin intermediul serverului \gls{netconf}. Nu este nevoie ca aceste atribute de configurare să fie citite din dispozitiv decât în momentul inițializării, deoarece apoi se presupune că orice configurare asupra dispozitivului se va face prin echipamentul de control \gls{sdn}, deci prin intermediul serverului \gls{netconf} care va fi astfel informat asupra noii valori a parametrilor respectivi.

Atributele de execuţie reprezintă parametrii dinamici, care pot fi doar citiţi ai dispozitivului, precum alarme, informații de stare sau de monitorizare a performanţelor. Acestea trebuie citite din echipament de fiecare dată când controlerul \gls{sdn} le cere, din cauza naturii lor dinamice. Soluția acestei abordări constă în nodurile virtuale oferite de soluția \textit{OpenYuma}. În loc de a avea o valoare stocată pentru un atribut \gls{yang}, sau valori pentru grupuri de atribute, \textit{OpenYuma} oferă posibilitatea de a asocia o funcție cu apel invers pentru astfel de parametri. Aceasta va fi apelată de fiecare dată când valoarea atributului asociat este cerută serverului \gls{netconf}, iar implementarea acesteia va prelua valoarea din fişierul \gls{xml} de configurare, sau va construi arborele asociat răspunsului, în cazul unui grup de atribute.

Cea de-a doua versiune a simulatorului \gls{dvm} se bazează pe abordarea \textit{OpenYuma} de a folosi funcții cu apel invers pentru implementarea funcţionalităţii de citire sau scriere asociată atributelor modelelor \gls{yang}. Proiectarea \gls{dvm} urmăreşte separarea descrisă anterior a parametrilor, oferind diferite funcții cu apel invers pentru diferitele categorii. Fiecare atribut din modelele \gls{yang} va fi modelat ca un nod \textit{OpenYuma}, care reprezintă un tip de date pus la dispoziţie de această soluție software, conţinând diverse informații rezultate din analizarea modelului \gls{yang}, precum numele, constrângeri asupra valorilor pe care atributul le poate lua și funcţia cu apel invers asociată, de citire sau de scriere. Astfel, au fost considerate trei funcții cu apel invers generice, care să fie folosite în cele trei cazuri date de împărțirea atributelor \gls{yang}: pentru parametrii de iniţializare, de execuţie și de configurare. Acestea sunt considerate generice, deoarece o singură astfel de funcție cu apel invers este utilizată pentru fiecare tip de atribut, diferenţierea pentru fiecare atribut în parte făcându-se în implementarea acesteia, în funcție de numele atributului. Această abordare oferă flexibilitate și uşurinţă în implementarea simulatorului \gls{dvm}.


\section{Implementarea DVM versiunea 2}

Implementarea simulatorului \gls{dvm} versiunea 2 folosește memoria internă și un fişier aflat în sistemul local de fişiere pentru stocarea și întoarcerea valorilor atributelor modelelor \gls{yang} pe care le expune. Fişierul de configurare conţine valori pentru toţi aceşti parametri și are o structură asemănătoare unui răspuns la operaţia \textit{get} al unui server \gls{netconf}. Această abordare este foarte avantajoasă, din două motive. În primul rând, dacă un dezvoltator de aplicații software \gls{sdn} are nevoie de alte valori pentru orice atribut \gls{yang}, este suficient să modifice acea valoare în fişier și să repornească simulatorul \gls{dvm}, fără sa fie nevoie să cunoască detaliile de implementare sau să recompileze serverul \gls{netconf}. În cel de-al doilea rând, \gls{dvm} poate oferi răspunsuri similare cu un mediator real dacă în fişierul de configurare se va introduce răspunsul \gls{xml} venit de la un astfel de mediator la o operaţie \textit{get}.

Fluxul de lucru pentru operaţiile de aducere a atributelor este ilustrat în Figura \ref{fig:dvm_v02_workflow}.

\begin{figure}[h]
	\centering
	\includegraphics[width=1\textwidth]{dvm_v02_workflow}
	\caption{Fluxul de lucru pentru aducerea atributelor: a) de execuţie; b) de iniţializare \cite{stancu2017enabling}.}
	\label{fig:dvm_v02_workflow}
\end{figure}

Echipamentul de control \gls{sdn} care se conectează la simulatorul \gls{dvm} poate efectua diferite operații asupra parametrilor de execuţie sau de iniţializare.
\section{Folosirea în contextul demonstraţiilor de concept a DVM versiunea 2}

Așa cum prima versiune a \gls{dvm} a fost foarte importantă în contextul celei de-a doua demonstraţii de concept a rețelelor de transport de date fără fir și a doua versiune a acestui simulator a fost o unealtă critică pentru pregătirea celei de-a treia demonstraţii de concept. A permis dezvoltatorilor de aplicații \gls{sdn} să le testeze și să le depaneze într-o manieră facilă, eliminând necesitatea deţinerii unor echipamente de rețea și a mediatorului asociat.

\gls{sdn} nu influenţează doar modul în care rețelele sunt controlate și administrate, ci și modul în care sunt organizate proiectele și modul în care anumite cerinţe și capabilităţi sunt dezvoltate și implementate. Dacă înainte aplicațiile și dispozitivele de rețea erau oferite împreună, de către producătorii de echipamente, interfeţele standard dintre dispozitive și echipamentele de control \gls{sdn}, sau dintre echipamentele de control și aplicații, oferă posibilitatea companiilor de a oferi doar anumite secţiuni din întreaga soluție. Aceste companii pot profita de simulatorul \gls{dvm}, deoarece le oferă acces la aceste interfețe fără a avea nevoie să deţină echipamente de rețea foarte scumpe.

Avantajul pe care \gls{dvm} îl oferă și care a fost folosit în cea de-a treia demonstraţie de concept \gls{onf} este reprezentat de faptul că orice modificare în modelele \gls{yang} dezvoltate poate fi testată cu ajutorul simulatorului, deoarece acesta oferă o modalitate rapidă de a o implementa, simulând astfel, din punctul de vedere al echipamentului de control \gls{sdn}, interacţiunea cu un dispozitiv de rețea. După dezvoltarea simulatorului și a aplicațiilor \gls{sdn}, acestea din urmă pot fi testate, chiar înainte ca software-ul din elementele de rețea să fie implementat. Cu toate acestea, este nevoie ca aplicațiile să efectueze teste de integrare cu dispozitivele de rețea, deoarece comportamentul simulatorului nu poate fi identic cu cel al echipamentelor. Simulatorul \gls{dvm} folosit în cea de-a treia demonstraţie de concept nu a simulat dispozitivele de rețea, ci nivelul mediator care oferă interfaţa \gls{netconf} ce expune modelele \gls{yang} dorite. A fost posibilă simularea diferitor topologii de rețea, având dispozitive cu diferite configurări și interfețe de transport de date. Simulatorul \gls{dvm} a fost și în cazul celor de-a doua și de-a treia demonstraţii de concept \gls{onf} unealta principală pentru a putea crea o topologie de rețea simulată \cite{onf2016_poc2, onf2016_poc3}. 

Simulatorul a fost folosit și pentru teste de extensibilitate și de performanţă ale aplicațiilor \gls{sdn} dezvoltate. Fişierul \gls{xml} de configurare poate fi manipulat foarte uşor, adăugând interfețe de transport de date echipamentelor sau intrări pentru valorile indicatorilor de performanţă. De exemplu, un dispozitiv de rețea are nevoie de 24 de ore pentru a genera 96 de intrări ale indicatorilor de performanţă de 15 minute și 30 de zile pentru a genera 30 de intrări ale indicatorilor de performanţă de 24 de ore. O altă funcție importantă a \gls{dvm} care a fost folosită în pregătirea demonstraţiei de concept a fost cea de generare a notificărilor \gls{netconf}. Au putut fi simulate diferite tipuri de notificări care să apară la un interval de timp configurabil.

În mod evident, timpii de răspuns ai simulatorului sunt foarte reduşi, deoarece nu există o comunicaţie cu un echipament de rețea real, deci nici nevoia de a procesa valorile atributelor care sunt citite din dispozitivele de rețea. Nevoia de a procesa valorile unor atribute apare în cazul mediatoarelor reale, în momentul în care atributele definite în modelul informațional pentru microunde nu se potrivesc în totalitate cu atributele dispozitivului și o anume procesare a acelei valori este necesară pentru a transforma-o în ce se doreşte în modelul \gls{yang}.

Tabelul \ref{tab:Table_3} ilustrează diferenţele între \gls{dvm} și un mediator real, în diferite situaţii care au apărut în pregătirea celei de-a treia demonstraţii de concept \gls{onf} \cite{stancu2017enabling}.

\begin{table}[hp]
	
	\caption{Comparaţie între comportamentele unui mediator real și al simulatorului în diferite situații \cite{stancu2017enabling}.\label{tab:Table_3}}
	\begin{tabular}{|M{0.5\textwidth}|M{0.2\textwidth}|M{0.2\textwidth}|}
		\hline
		\textbf{Situația} & \textbf{\emph{Mediator real}} & \textbf{\emph{Simulatorul DVM}} \tabularnewline
		\hline 
		Dezvoltarea aplicațiilor \gls{sdn} & costisitoare & eficientă \tabularnewline
		\hline 
		Timpul de implementare & greoi & rapid \tabularnewline
		\hline 
		Topologia de rețea & complex de schimbat & flexibilă, uşor de schimbat \tabularnewline
		\hline 
		Generarea notificărilor \gls{netconf} & greu de controlat & simplă \tabularnewline
		\hline 
		Timpi de răspuns & reali & nerealişti \tabularnewline
		\hline 
		Procesare a valorilor atributelor & în unele cazuri & nu este nevoie \tabularnewline
		\hline \end{tabular}
\end{table}

Pentru a oferi un suport mai bun testelor de extensibilitate și de performanţă ale aplicațiilor, simulatorul \gls{dvm} ar fi putut implementa timpi de răspuns configurabili la operaţiile \gls{netconf} sau generarea de notificări care să urmărească anumite pofile temporale, în funcție de nevoile aplicațiilor \gls{sdn}.

La fel ca în cazul versiunii anterioare a \gls{dvm}, înregistrarea simulatoarelor la echipamentul de control \gls{sdn} se face manual, cu ajutorul interfeţei grafice a controlerului, care stabileşte apoi conexiunea securizată și iniţiază o sesiune \gls{netconf}.
\section{LINC-WE. Integrarea cu \textit{mininet}}

Așa cum a fost descris anterior, simulatorul \gls{dvm} permite emularea nivelului mediator care apare în arhitectura \gls{sdn} a rețelelor de transport de date fără fir. Se pot simula diferite topologii de rețea prin manipularea fişierelor \gls{xml} de configurare ale diferitelor instanţe ale \gls{dvm}, făcând simulatoarele să prezinte diferite configuraţii. Această abordare oferă un bun prim pas pentru simularea rețelelor de transport de date fără fir în contextul \gls{sdn}, însă nu este suficient, deoarece instanţele \gls{dvm} sunt independente și rețeaua simulată nu este completă, deoarece nu există legături între dispozitivele simulate. Din acest motiv, această secţiune prezintă încercarea de a integra simulatorul \gls{dvm} cu un comutator software deja existent, care a fost folosit anterior pentru simulări de rețele optice în contextul \gls{sdn}: \textit{Legătura Nu Este Închisă} - \gls{linc} și cu simulatorul de rețele definite prin software, \textit{mininet}. Comutatorul software folosit anterior în simularea rețelelor optice, bazat pe \gls{linc} se numește \textit{Legătura Nu Este Închisă - Extensii Optice} - \gls{linc-oe}, astfel că am denumit comutatorul rezultat în urma integrării \gls{dvm} cu \gls{linc}: \textit{Legătura Nu Este Închisă - Extensii Fără Fir} - \gls{linc-we} \cite{stancu2017wireless}.

\subsection{LINC}

\gls{linc} este un comutator software ce suportă protocolul OpenFlow, scris în limbajul de programare Erlang \cite{lincsw}. A fost proiectat să fie modular, oferind o metodă rapidă de a face prototipuri \gls{sdn} și de a testa noi caracteristici ale protocolului OpenFlow. Este oferit ca o implementare cu sursă deschisă de către FlowForwarding.org și are ca scop oferirea unei soluții prin care utilizatorii pot evalua rapid protocoalele OpenFlow și OF-Config.

Un singur comutator este implementat ca un nod Erlang și cuprinde mai multe comutatoare logice. Acestea, porturile și legăturile dintre porturi sunt descrise de către utilizator prin intermediul unui fișier de configurare, așa cum este ilustrat și în Figura \ref{fig:linc_architecture}. Datorită acestei arhitecturi modulare, comutatorul \gls{linc} este capabil să ofere mai multe implementări pe care utilizatorul le poate alege, fiecare având asociată o versiune diferită a protocolului OpenFlow (versiunile 1.2, 1.3 sau 1.4). Aceste implementări sunt responsabile de comutarea pachetelor și alterarea tabelelor de fluxuri de date.

\begin{figure}[h]
	\centering
	\includegraphics{linc_architecture}
	\caption{Arhitectura comutatorului software LINC \cite{linc2014qsg}.}
	\label{fig:linc_architecture}
\end{figure}

După cum este prezentat în \cite{linc2014qsg}, \gls{linc} prezintă mai multe blocuri software: comutatorul capabil OpenFlow, modului protocolului OpenFlow, modului protocolului OF-Config. Acestea sunt dezvoltate ca aplicații Erlang separate, respectând principiile \textit{Platforma Telecom Deschisă} - \gls{otp} ale limbajului Erlang. Există și o componentă separată care se ocupă de conexiunea la un echipament de control \gls{sdn}. Tabelele de fluxuri, porturile sau tabelele de grup sunt administrate de implementările separate reprezentând diferitele versiuni ale protocolului OpenFlow.

\subsubsection{LINC-OE}

Datorită naturii modulare a comutatorului \gls{linc}, a apărut o nouă implementare care să acopere cazurile de utilizare ale \gls{sdn} din domeniul optic: \gls{linc-oe}. Acesta reprezintă un simulator de comutatoare optice care suportă extensiile optice ale protocolului OpenFlow. Diferenţa dintre \gls{linc} și \gls{linc-oe} este că, în cazul celui din urmă, comutatoarele logice care sunt simulate reprezintă comutatoare optice, astfel că spre deosebire de interfeţele electrice, porturile optice oferă mai multe canale de comunicaţie independente, diferenţiate prin lungimea de undă. În implementarea \gls{linc-oe}, mesajele ce se transmit printr-un astfel de port optic nu mai sunt pachete Ethernet, ci mesaje Erlang, ce conţin informații adiţionale, pe lângă pachetul Ethernet propriu-zis. O altă diferenţă este dată de faptul ca \gls{linc-oe} oferă interfețe pentru simularea defectării legăturilor dintre porturi.

\subsubsection{Interfaţa NETCONF}

Comutatorul software \gls{linc} expune de asemenea și o interfață \gls{netconf}. Este folosită de către protocolul OF-Config și este disponibilă pentru un comutator doar dacă acel protocol este folosit. În cazul în care este permisă utilizarea acestuia, o nouă aplicație Erlang este pornită în cadrul nodului Erlang: \textit{enetconf}, care pornește un server \gls{netconf} ce așteaptă conexiuni pe IP-ul și portul selectate de utilizator.

Această abordare prezintă un dezavantaj major: un singur server \gls{netconf} este pornit pentru comutatorul \gls{linc}, astfel că nu se pot accesa comutatoarele logice care alcătuiesc comutatorul \gls{linc} individual prin interfața \gls{netconf}. Acestea pot fi configurate doar prin intermediul protocolului OpenFlow.

De asemenea, modelul informațional \gls{yang} care este expus de către server nu este configurabil, astfel încât utilizatorul nu poate alege un model \gls{yang} pe care comutatorul software să îl poată folosi. Dacă acest lucru ar fi fost oferit, ar fi fost banală transformarea \gls{linc-oe} în \gls{linc-we}, prin înlocuirea modelului \gls{yang} expus cu modelul informațional pentru microunde și apoi corelarea atributelor \gls{yang} cu parametrii comutatorului \gls{linc}.

\subsubsection{\textit{mininet}}

\subsection{LINC-WE}