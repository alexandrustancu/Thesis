% Modified by Cristian Damian for SD-ETTI-B
% ****************************************************************************************** % Dissertation template and document class for Princeton University
% Author  : Jeffrey Scott Dwoskin <jdwoskin@princeton.edu>
% Adapted from: http://www.math.princeton.edu/graduate/tex/puthesis.html
% ****************************************************************************************** %


%%% For print copies
%% set 'singlespace' option to set entire thesis to single space, and define "\printmode" to remove all hyperlinks for printed copies of the thesis. Delete all output files before changing this mode -- it will turn hyperref package on and off
\documentclass[12pt,lot, lof, singlespace, twoside, openright]{puthesis}
\newcommand{\printmode}{}

%%% For the electronic copy, use doublespacing, define "\proquestmode" to use outlined links, instead of colored links. 
%\documentclass[12pt,twoside]{puthesis}
%\newcommand{\proquestmode}{}
% I prefer proquestmode to be off for electronic copies for normal use, since the colored links are less distracting. However when printed in black and white, the colored links are difficult to read. 

%%% For early drafts without some of the frontmatter
% Also see the "ifodd" command below to disable more frontmatter
%\documentclass[12pt]{puthesis}

%%%%%%%%%%%%%%%%%%%%%%%%%%%%%%%%%%%%%%%%%%%%%%%%%%%%%%%%%%%%%\
%%%% Author & title page info

\title{Contributions to the development and implementation of Software-Defined Networks}
\titleRo{Contribuții la dezvoltarea și implementarea rețelelor definite prin programe soft}

\submitted{București 2018}  % Location and year
\decision{..... din .......... }
\copyrightyear{2018}  % year in which the copyright is secured by publication of the dissertation.
\author{Ing. Liviu-Alexandru STANCU}

% Doctoral comission table
\adviser{
	\hline
	\centering Președinte & \centering Prof. Dr. Ing. \mbox{Gheorghe BREZEANU} & \centering de la & Univ. ``Politehnica'' din Bucureşti \\ \hline
   	\centering Conducător de doctorat & \centering Prof. Dr. Ing. \mbox{Simona HALUNGA} & \centering de la & Univ. ``Politehnica'' din Bucureşti \\ \hline
	\centering Referent &  & \centering de la & Univ. ``Politehnica'' din Bucureşti \\ \hline
	\centering Referent &  & \centering de la & Univ. ``Politehnica'' din Bucureşti \\ \hline
	\centering Referent &  & \centering de la & Univ. ``Politehnica'' din Bucureşti \\ \hline
}  
%\departmentprefix{Program in}  % defaults to "Department of", but programs need to change this.
%\department{Electrical Engineering}

%%%%%%%%%%%%%%%%%%%%%%%%%%%%%%%%%%%%%%%%%%%%%%%%%%%%%%%%%%%%%\
%%%% Tweak float placements
% From: http://mintaka.sdsu.edu/GF/bibliog/latex/floats.html "Controlling LaTeX Floats"
% and based on: http://www.tex.ac.uk/cgi-bin/texfaq2html?label=floats
% LaTeX defaults listed at: http://people.cs.uu.nl/piet/floats/node1.html

% Alter some LaTeX defaults for better treatment of figures:
    % See p.105 of "TeX Unbound" for suggested values.
    % See pp. 199-200 of Lamport's "LaTeX" book for details.
    %   General parameters, for ALL pages:
    \renewcommand{\topfraction}{0.85}	% max fraction of floats at top
    \renewcommand{\bottomfraction}{0.6}	% max fraction of floats at bottom
    %   Parameters for TEXT pages (not float pages):
    \setcounter{topnumber}{2}
    \setcounter{bottomnumber}{2}
    \setcounter{totalnumber}{4}     % 2 may work better
    \setcounter{dbltopnumber}{2}    % for 2-column pages
    \renewcommand{\dbltopfraction}{0.66}	% fit big float above 2-col. text
    \renewcommand{\textfraction}{0.15}	% allow minimal text w. figs
    %   Parameters for FLOAT pages (not text pages):
    \renewcommand{\floatpagefraction}{0.66}	% require fuller float pages
	% N.B.: floatpagefraction MUST be less than topfraction !!
    \renewcommand{\dblfloatpagefraction}{0.66}	% require fuller float pages
    
    \renewcommand*{\chaptername}{Capitolul}
    \renewcommand\bibname{Bibliografie}
    \renewcommand{\figurename}{Fig.}
    \renewcommand{\tablename}{Tabelul}
    
    \hyphenation{par-ti-ci-pat}
    \hyphenation{ca-rac-te-ris-ti-ci-le}
    \hyphenation{ca-rac-te-ris-tici}
    \hyphenation{cer-ce-ta-re}
    \hyphenation{con-si-de-ra-te}
    \hyphenation{con-tex-tul}
    \hyphenation{mo-de-lul}

% The documentclass already sets parameters to make a high penalty for widows and orphans. 

%%%%%%%%%%%%%%%%%%%%%%%%%%%%%%%%%%%%%%%%%%%%%%%%%%%%%%%%%%%%%\
%%%% Use packages

%\usepackage{amsfonts}
\usepackage[utf8x]{inputenc}

%%% For figures
\usepackage{graphicx, color, rotating}

\DeclareGraphicsExtensions{.png, .pdf}
%\usepackage{subfig,rotate}

%%% for comments
\usepackage{verbatim}

%%% For tables
\usepackage{multirow}
% Longtable lets you have tables that span multiple pages.
\usepackage{longtable}

% Booktabs produces far nicer tables than the standard LaTeX tables.
%   see: http://en.wikibooks.org/wiki/LaTeX/Tables
\usepackage{booktabs}

\usepackage{emptypage}
\usepackage{lipsum}


\usepackage[acronym,nomain,nonumberlist,nopostdot,nogroupskip]{glossaries}
%\usepackage[nonumberlist,nopostdot,nogroupskip]{glossaries}
\makeglossaries
\PrerenderUnicode{ș}
\PrerenderUnicode{ț}
\PrerenderUnicode{ă}
\PrerenderUnicode{î}
\PrerenderUnicode{â}

\newglossarystyle{symbunitlong}{%
	\setglossarystyle{long3col}% base this style on the list style
	\renewcommand{\arraystretch}{1.2}
	\renewenvironment{theglossary}{% Change the table type --> 3 columns
		\begin{longtable}[c]{l p{0.4\textwidth} p{0.4\textwidth}}}%%
		{\end{longtable}}%
	%
	\renewcommand*{\glossaryheader}{}
	\renewcommand*{\glossentry}[2]{%  Change the displayed items
		\glstarget{##1}{\glossentryname{##1}} %
		& \glossentrydesc{##1}% Description
		& \glsentryuseri{##1}  \tabularnewline
	}
}
%\usepackage{glossary-mcols}
%\usepackage{glossaries-extra}
%\renewcommand{\glsnamefont}[1]{\textbf{#1}\hspace{0.5cm}-\hspace{0.25cm}}
\newacronym{iot}{IoT}{Internet of Things}
\newacronym{sdn}{SDN}{Software-Defined Networking}
\newacronym{onf}{ONF}{Open Networking Foundation}
\newacronym{tr}{TR}{Technical Recommendation}
\newacronym{ts}{TS}{Technical Specification}
\newacronym{netconf}{NETCONF}{Network Configuration Protocol}
\newacronym{ot}{OT}{Optical Transport}
\newacronym{wt}{WT}{Wireless Transport}
\newacronym{dvm}{DVM}{Default Values Mediator}
\newacronym{poc}{PoC}{Proof of Concept}
\newacronym{wte}{WTE}{Wireless Transport Emulator}
\newacronym{ncp}{NCP}{Network Control Point}
\newacronym{forces}{ForCES}{Forwarding and Control Element Separation}
\newacronym{pof}{POF}{Protocol-Oblivious Forwarding}
\newacronym{ovsdb}{OVSDB}{Open vSwitch Database Management}
\newacronym{rofl}{ROFL}{Revised Open Flow Library}
\newacronym{hal}{HAL}{Hardware Abstraction Layer}
\newacronym{vlan}{VLAN}{Virtual Local Area Network}
\newacronym{mpls}{MPLS}{Multi-Protocol Label Switching}
\newacronym{nat}{NAT}{Network Address Translation}
\newacronym{nvp}{NVP}{Network Virtualization Platform}
\newacronym{xdpd}{xDPd}{eXtensible Datapath Daemon}
\newacronym{odl}{ODL}{OpenDaylight}
\newacronym{onos}{ONOS}{Open Network Operating System}
\newacronym{posix}{POSIX}{Portable Operating System Interface}
\newacronym{wlan}{WLAN}{Wireless Local Area Network}
\newacronym{sdo}{SDO}{Standards Developing Organization}
\newacronym{ietf}{IETF}{Internet Engineering Task Force}
\newacronym{etsi}{ETSI}{European Telecommunications Standards Institute}
\newacronym{itu-t}{ITU-T}{International Telecommunications Union - Telecommunications Standardization Sector}
\newacronym{ieee}{IEEE}{Institute of Electrical and Electronic Engineers}
\newacronym{bbf}{BBF}{BroadBand Forum}
\newacronym{mef}{MEF}{Metro Ethernet Forum}
\newacronym{oif}{OIF}{Optical Interface Forum}
\newacronym{otwg}{OTWG}{Open Transport Working Group}
%\input{abrevieriroen}

\usepackage{array}
\newcolumntype{P}[1]{>{\centering\arraybackslash}p{#1}}
\newcolumntype{M}[1]{>{\centering\arraybackslash}m{#1}}


\usepackage{fancyhdr}
\fancypagestyle{mypagestyle}{%
	\fancyhf{}% Clear header/footer
	\fancyhead[OC]{\fontsize{10}{12} \selectfont \nouppercase{\leftmark}}% Author on Odd page, Centred
	\fancyhead[EC]{\fontsize{10}{12} \selectfont Contribuții la dezvoltarea și implementarea rețelelor definite prin programe soft}% Title on Even page, Centred
	\fancyfoot[C]{\thepage}%
	%\renewcommand{\headrulewidth}{.4pt}% Header rule of .4pt
	\renewcommand{\headrulewidth}{0pt}
}

\usepackage{notoccite}
\usepackage{cite}
\usepackage{url}
%set parameters for longtable:
% default caption width is 4in for longtable, but wider for normal tables
\setlength{\LTcapwidth}{\textwidth}

%%%%%%%%%%%%%%%%%%%%%%%%%%%%%%%%%%%%%%%%%%%%%%%%%%%%%%%%%%
%%% Printed vs. online formatting
\ifdefined\printmode

% Printed copy
% url package understands urls (with proper line-breaks) without hyperlinking them
\usepackage{url}


\else

\ifdefined\proquestmode
%ProQuest copy -- http://www.princeton.edu/~mudd/thesis/Submissionguide.pdf

% ProQuest requires a double spaced version (set previously). They will take an electronic copy, so we want links in the pdf, but also copies may be printed or made into microfilm in black and white, so we want outlined links instead of colored links.
\usepackage{hyperref}
\hypersetup{bookmarksnumbered}

% copy the already-set title and author to use in the pdf properties
\makeatletter
\hypersetup{pdftitle=\@title,pdfauthor=\@author}
\makeatother

\else
% Online copy

% adds internal linked references, pdf bookmarks, etc

% turn all references and citations into hyperlinks:
%  -- not for printed copies
% -- automatically includes url package
% options:
%   colorlinks makes links by coloring the text instead of putting a rectangle around the text.
\usepackage{hyperref}
\hypersetup{colorlinks,bookmarksnumbered}

% copy the already-set title and author to use in the pdf properties
\makeatletter
\hypersetup{pdftitle=\@title,pdfauthor=\@author}
\makeatother

% make the page number rather than the text be the link for ToC entries
%\hypersetup{linktocpage}
\fi % proquest or online formatting
\fi % printed or online formatting


%%%%%%%%%%%%%%%%%%%%%%%%%%%%%%%%%%%%%%%%%%%%%%%%%%%%%%%%%%%%%\
%%%% Define commands

% Define any custom commands that you want to use.
% For example, highlight notes for future edits to the thesis
%\newcommand{\todo}[1]{\textbf{\emph{TODO:}#1}}


% create an environment that will indent text
% see: http://latex.computersci.org/Reference/ListEnvironments
% 	\raggedright makes them left aligned instead of justified
\newenvironment{indenttext}{
    \begin{list}{}{ \itemsep 0in \itemindent 0in
    \labelsep 0in \labelwidth 0in
    \listparindent 0in
    \topsep 0in \partopsep 0in \parskip 0in \parsep 0in
    \leftmargin 1em \rightmargin 0in
    \raggedright
    }
    \item
  }
  {\end{list}}

% another environment that's an indented list, with no spaces between items -- if we want multiple items/lines. Useful in tables. Use \item inside the environment.
% 	\raggedright makes them left aligned instead of justified
\newenvironment{indentlist}{
    \begin{list}{}{ \itemsep 0in \itemindent 0in
    \labelsep 0in \labelwidth 0in
    \listparindent 0in
    \topsep 0in \partopsep 0in \parskip 0in \parsep 0in
    \leftmargin 1em \rightmargin 0in
    \raggedright
    }

  }
  {\end{list}}



%%%%%%%%%%%%%%%%%%%%%%%%%%%%%%%%%%%%%%%%%%%%%%%%%%%%%%%%%%%%%\
%%%% Front-matter

% For early drafts, you may want to disable some of the frontmatter. Simply change this to "\ifodd 1" to do so.
% front-matter disabled while writing chapters

\renewcommand*{\makeabstract}{}
\renewcommand*{\makecopyrightpage}{}

% you can just skip the \acknowledgements and \dedication commands to leave out these sections.

\acknowledgements{
%I would like to thank...
Țin să mulțumesc familiei mele, în special Veronicăi, soția mea, pentru că mi-a fost alături și m-a încurajat pe toată perioada studiilor (care a părut interminabilă), ajutându-ma să trec peste toate momentele dificile cu zâmbetul pe buze, dar și părinților, pentru toate eforturile pe care le-au făcut ca să ajung până în acest punct și pentru încrederea pe care mi-au acordat-o întotdeauna.

De asemenea, aș vrea să le mulțumesc profesorilor mei, sunt convis că fiecare și-a pus amprenta într-un fel sau altul asupra mea și implicit asupra acestei lucrări, cu atât mai mult celor din comisia de îndrumare, care au avut răbdare cu mine și mi-au asigurat condițiile necesare performanței. 

Nu în ultimul rând, aș vrea să îmi exprim aprecierea față de prietenii mei, care m-au tolerat în toată această (lungă) perioadă solicitantă, și în special față de bunul meu prieten Martin Skorupski, care a contribuit din plin la această activitate, prin discuții, sugestii, idei și bună dispoziție.

}

%\dedication{To my parents.}


%%%%%%%%%%%%%%%%%%%%%%%%%%%%%%%%%%%%%%%%%%%%%%%%%%%%%%%%%%%%%\
%%%% Hide some chapters

%%% If you want to produce a pdf that includes only certain chapters, specify them with includeonly, in addition to including all chapters below.
%\includeonly{ch-intro/chapter-intro}
%%% You can also specify multiple chapters.
%\includeonly{ch-intro/chapter-intro,ch-usage/chapter-usage}
%\includeonly{chap1,chap2,chap3}


%%%%%%%%%%%%%%%%%%%%%%%%%%%%%%%%%%%%%%%%%%%%%%%%%%%%%%%%%%%%%
%%%% Notes:

% Footnotes should be placed after punctuation.\footnote{place here.}
% Generally, place citations before the period~\cite{anotherauthor}.
% The proper usage for i.e., and e.g., include commas ``(e.g., option A, option B)''

%%%%%%%%%%%%%%%%%%%%%%%%%%%%%%%%%%%%%%%%%%%%%%%%%%%%%%%%%%%%%
%%%% Import chapters

\begin{document}

\makefrontmatter


% If you've disabled frontmatter, you can insert the toc manually
%\tableofcontents\clearpage

% \include lets us split up the document (and each include starts a new page):

\chapter{Introducere\label{ch:introducere}}

\gls{utc} is used here.

\lipsum

\section{Prezentarea domeniului tezei de doctorat}

În urma dezvoltării tehnologice recente în toate domeniile, în general și în domeniul calculatoarelor și al telecomunicaţiilor, în particular, a apărut nevoia de a redefini arhitectura reţelelor de comunicaţii, din cauza faptului că reţelele tradiţionale au început să îşi arate limitele. În zilele noastre, există o tendinţă de a interconecta toate echipamentele, cu ajutorul unor tehnologii care permit acest lucru, cum ar fi arhitectura \textit{Cloud Computing}, mobilitatea, sau idei mai noi, cum ar fi \textit{Internetul Tuturor Lucrurilor} - \gls{iot} sau sistemele de comunicaţii de generaţia a cincea - 5G. Aceste noi abordări au nevoie, pe lângă o lăţime de bandă crescută, de o reţea mai simplă și agilă, unde se facilitează inovarea. Reţelele definite prin software - \gls{sdn}, reprezintă o nouă paradigmă care a apărut în industria reţelisticii, pentru a mitiga dezavantajele pe care reţelele tradiţionale le-au dovedit.

Tehnologia \gls{sdn} nu este încă matură și nu a pătruns în toate tipurile de reţele de comunicaţii. Este prezentă în campusuri universitare, sau în centre de date, însă se încearcă introducerea acesteia în toate aspectele unei reţele de comunicaţii, cum ar fi transportul de date optic - \gls{ot}, transportul de date fără fir - \gls{wt} sau noduri de interconectare ale Internetului. Aceste încercări presupun muncă de standardizare și demonstraţii de concept - \gls{poc}, pentru prezentarea avantajelor pe care această nouă paradigmă de oferă, până când tehnologia se va maturiza și va fi adoptată de toată industria reţelisticii.
\section{Scopul tezei de doctorat}

Această lucrare îşi propune să prezinte noua paradigmă introdusă anterior, \gls{sdn}, împreună cu avantajele pe care această abordare le poate aduce dacă ar fi aplicată în toate aspectele unei reţele de comunicaţii, punând accent pe reţelele de transport de date fără fir. Autorul îşi va prezenta activitatea de cercetare, constând în unelte software care pot fi folosite ca simulatoare de echipamente de transport de date fără fir, ce expun interfeţe folosite în tehnologia reţelelor definite prin software.

\textbf{Aceste unelte software sunt contribuții originale, placând de la partea de arhitectură și până la cea de implementare.} Ele au fost folosite cu succes în procesul de standardizare al \gls{sdn}, care încă se desfăşoară în cadrul \gls{onf}, facilitând testarea modelelor informaţionale ce se dezvoltă în contextul reţelelor definite prin software și uşurând dezvoltarea și testarea aplicaţiilor \gls{sdn} care fac parte din acest ecosistem. Simulatorul rezultat în urma acestei cercetări, în forma sa finală, poate emula o întreagă reţea de echipamente de transport de date fără fir, care expun interfeţe specifice \gls{sdn}. El poate fi folosit de către dezvoltatorii de produse software \gls{sdn} pentru reţele de transport de date fără fir, eliminând nevoia acestora de a deţine astfel de echipamente scumpe pentru a-și putea testa aplicaţiile. Poate fi folosit și de către operatorii care vor sa lanseze această tehnologie în reţelele de producţie, pentru a simula consecinţele instalării unor noi aplicaţii anterior lansării, fără a afecta reţeaua.
\section{Conţinutul tezei de doctorat}

Lucrarea este împărţită în opt capitole, primul prezentând domeniul abordat în teză, iar ultimul fiind dedicat concluziilor. În continuare va fi prezentat, pe scurt, conţinutul fiecărui dintre celelalte capitole.

Capitolul \ref{ch:introducere_sdn} introduce domeniul reţelelor definite prin software, plecând de la istoria sa și nevoia pentru care această nouă paradigmă a apărut. Apoi va fi ilustrată activitatea de standardizare în acest domeniu, inclusiv demonstraţiile de concept conduse de către \gls{onf}, în particular de către grupul \gls{wt}, care va duce la maturizarea soluţiei și adoptarea acesteia pe scară largă, în toate aspectele unei reţele. Tot în acest capitol se va evidenţia și prezenţa \gls{sdn} în contextul reţelelor actuale.

Cel de-al \ref{ch:sdn_in_contextul_wt}-lea capitol pune accent pe prezenţa \gls{sdn} în reţelele de transport de date fără fir. Sunt prezentate modelele informaţionale dezvoltate în cadrul \gls{onf} în acest context, având rolul de recomandări tehnice - \gls{tr}: \textit{TR-532, Microwave Information Model} și \textit{TR-512, Core Information Model}. Ulterior se vor da detalii despre \gls{netconf}, care este protocolul de bază pentru reţelele de transport de date fără fir, în contextul \gls{sdn}. În următoarea secţiune se vor compara câteva cadre software cu sursă deschisă, ce oferă facilitatea unui server \gls{netconf}. Pe baza acestei comparaţii s-a ales unealta software care va face parte din simulatorul propus în lucrare. Capitolul va fi încheiat de o prezentare a arhitecturii demonstraţiilor de concept organizate de grupul \gls{wt} din \gls{onf}, ce va ajuta la înţelegerea necesităţii unui astfel de simulator.

Capitolul \ref{ch:dvm_v01} prezintă prima versiune a simulatorului, numită \textit{Mediatorul cu valori implicite} - \gls{dvm}, folosită în cel de-al doilea \gls{poc}. Se vor prezenta, pe rând, arhitectura și implementarea, iar apoi se va evidenţia folosirea acestui simulator în contextul demonstraţiilor de concept.

Următorul capitol, \ref{ch:dvm_v02}, descrie cea de-a doua versiune a \gls{dvm}, abordând aspecte despre arhitectura, implementare și folosire in cadrul celui de-al treilea \gls{poc}. În plus, se va evidenţia încercarea de a integra acest simulator cu o soluţie de comutator software, LINC, folosit în \gls{sdn}, în special în cadrul reţelelor de transport optic de date, prezentând avantajele și dezavantajele date de această abordare.

Capitolul \ref{ch:wte} descrie ultima și cea mai avansată versiune a simulatorului reţelelor de transport de date fără fir - \gls{wte}, prezentând arhitectura, detaliile implementării și folosirea acestuia în cea de-a patra demonstraţie de concept a grupului \gls{wt} din cadrul \gls{onf}.

Capitolul \ref{ch:rezultate_discutii} ilustrează rezultatele obţinute în urma acestei cercetări și propune discuţii pe baza simulatorului implementat. În primul rând, această soluţie este evaluată din punctul de vedere al resurselor consumate și al extensibilităţii pe care o oferă. Apoi, se compară simulatorul cu alte soluţii care există în momentul de faţă în contextul \gls{sdn}. Ulterior, se prezintă câteva cazuri de utilizare, propuse în cadrul grupului \gls{wt} din \gls{onf}, care pot fi demonstrate cu ajutorul simulatorului, eliminând nevoia unor echipamente de transport de date fără fir.



\chapter{Introducere \^{\i}n reţelele definite prin software\label{ch:introducere_sdn}}

\graphicspath{ {cap-introducere_in_sdn/figures/} }

În zilele noastre, reţelele de comunicaţii au devenit complexe și greu de administrat și configurat. De asemenea, numărul dispozitivelor mobile a crescut considerabil, alături de conţinutul pe care acestea îl accesează. Aceste lucruri au dus la evidenţierea limitărilor pe care reţelele tradiţionale le presupun. Chiar dacă nu toate ideile ce stau la baza ei sunt noi, datorită unui context favorabil, acestea, împreună cu alte noi idei, au dus la apariţia paradigmei \gls{sdn} în industria reţelisticii.

Această nouă tehnologie nu a ajuns încă la maturitate și la adoptarea pe scară largă, în toate aspectele reţelelor, însă eforturile considerabile care se fac în activităţile de standardizare și crearea de ecosisteme \gls{sdn} vor duce la această adoptare. După cum este evidenţiat și în~\cite{nadeau2013sdn}, se tinde către crearea unor reţele care pot fi programate prin software, crescând astfel flexibilitatea și agilitatea lor.

\section{Istoria SDN}

Istoria reţelelor definite prin software.
\section{Standardizarea SDN}

Activitățile de cercetare și standardizare din jurul \gls{sdn} au loc pe două planuri importante. Pe de o parte, există organizaţiile care dezvoltă standarde - \gls{sdo}, care sunt formate din reprezentanţi ai industriei, ai academiei, sau alte entități și au ca scop dezvoltarea de specificaţii sau recomandări tehnice, în contextul \gls{sdn} care să fie folosite de toată industria rețelisticii. Autorii din~\cite{schneider2014standardizations} amintesc astfel de organizații: \gls{onf}, \gls{ietf}, \gls{etsi}, \gls{itu-t}, \gls{ieee}. 

Pe de altă parte, există asociații sau comunități de oameni, în general care fac parte din industrie, ale căror rezultate ale cercetării candidează pentru a deveni standarde. Aceste rezultate sunt, de obicei, implementări cu sursă deschisă ce vor fi folosite ulterior în industrie. Exemple de astfel de comunități sunt prezente în~\cite{halpern2014standards, meyer2013software}: OpenDaylight (activităţi ce se desfăşoară sub patronajul fundaţiei Linux), \gls{mef}, \gls{bbf}, \gls{oif}.

Activitățile de standardizare ale \gls{sdn} sunt foarte importante, asigurându-i acestei tehnologii o evoluţie stabilă și aducând-o la o maturitate care îi va permite adoptarea pe scară largă în industria rețelisticii, mitigând astfel dezavantajele rețelelor tradiționale.

Sunt mai multe planuri pe care se lucrează pentru standardizarea \gls{sdn}. Unul dintre aceste planuri este cel al interfeţelor de tip \textit{southbound}, care fac legătura între echipamentele de dirijare și echipamentele de control \gls{sdn}. Protocolul OpenFlow este un exemplu în acest sens, însă nu este singurul protocol capabil să facă legătura între planurile de date și de control. În ultimul timp se pune foarte mare accent pe \gls{netconf} ca o alternativă pentru a configura echipamentele care dirijează traficul, după cum se poate observa în~\cite{csoma2015escape, felix2014multi, zhou2014research}. Astfel apare nevoia de a dezvolta modele informaţionale care să abstractizeze echipamentele din planul de date și care să poată fi folosite de \gls{netconf} pentru a configura dispozitivele. Și planul interfeţelor de tip \textit{northbound} are un rol important în standardizare, deoarece poate oferi un punct de plecare comun pentru dezvoltatorii de aplicații \gls{sdn}. De exemplu, în cadrul \gls{onf} există un grup care se ocupă cu activităţi de standardizare în această direcţie. Un alt plan este reprezentat de implementările software, cu sursă deschisă (\textit{open-source}), care se crează în acest context. Așa cum evidenţiază și autorii din~\cite{lin2014software, rothenberg2014open}, aceste implementări sunt foarte importante prin ecosistemele care apar ca urmare a activităţii \textit{comunităţilor open-source}.

O importanţă foarte mare în cadrul acestor activităţi o au și demonstraţiile de concept. Acestea au capacitatea de a demonstra avantajele pe care \gls{sdn} le aduce nu doar la un nivel teoretic, ci într-un mod practic, propunând diferite cazuri reale de utilizare a acestei tehnologii și aplicând-o, într-un mod restrâns, pe topologii formate din echipamente reale. Scopul acestora este, pe de o parte, de a adăuga un plus de valoare activităţilor de standardizare și de a testa și proba rezultatele acestor activităţi. Pe de altă parte, aceste demonstraţii de concept au scopul de a atrage atenţia și altor entități din industrie și a duce la înlesnirea adoptării acestei tehnologii pe scară largă.

\subsection{ONF}

\gls{onf} este o organizaţie non-profit formată din peste două sute de membri care fac parte din industrie, academie, sau institute de cercetare, ce are ca obiectiv accelerarea adoptării \gls{sdn} pe scară largă in industria rețelisticii prin dezvoltarea de standarde deschise și de ecosisteme software cu sursă deschisă. \gls{onf} a apărut ca urmare a activităţii de cercetare din jurul protocolului OpenFlow din cadrul Universităţii Stanford. Dintre membrii cei mai importanți amintim operatori de rețele, precum AT\&T, Google, Facebook, Verizon, Deutsche Telekom sau Telefonica, producători de echipamente, cum ar fi Cisco, Ericsson, Huawei, Intel sau NEC și reprezentanţi ai unor universităţi cunoscute, ca Stanford sau Princeton.

Activitățile din cadrul \gls{onf} sunt împărţite în mai multe zone de interes:
\begin{itemize}
	\item \textit{Operatori}. Această zonă se ocupă de mai multe aspecte, precum \gls{sdn} în contextul sistemelor \textit{Carrier Grade}, în centre de date, în întreprinderi sau aspecte legate de migrarea serviciilor dintr-o rețea tradiţională într-o rețea definită prin software.
	\item \textit{Servicii}. Se ocupă de proiecte care permit existența aplicațiilor și serviciilor de rețea care au la bază tehnologia \gls{sdn}. Astfel, există mai multe grupuri de lucru care analizează: arhitectura \gls{sdn} și cum aceste principii se aplică în imaginea de ansamblu a unei rețele de comunicaţii; un model informaţional \textit{de bază}, care să reprezinte piatra de temelie cu ajutorul căreia să se dezvolte alte modele informaţionale, specializate pentru anumite aplicații sau tehnologii; interfeţele de tip \textit{northbound}, pentru a facilita dezvoltarea aplicațiilor \gls{sdn}; probleme de securitate pe care această nouă tehnologie le poate întâmpina.  
	\item \textit{Specificaţii}. Zonă care are ca responsabilitate publicarea tuturor specificaţiilor și recomandărilor tehnice create de \gls{onf}. Acestea includ protocoalele OpenFlow, OF-Config dar și alte interfețe standard care se dezvoltă pentru tehnologii de transport de diferite tipuri (optic, fără fir). Există și un proiect care se ocupă de testare și interoperabilitate, scopul acestuia fiind accelerarea adoptării protocolului OpenFlow prin certificări și promovarea interoperabilităţii între diferiţi producători de echipamente.
	\item \textit{Piaţă}. Această zonă se concentrează pe educarea comunităţii \gls{sdn} în legătură cu valoarea pe care standardele \gls{onf} le oferă rețelelor definite prin software și promovarea adoptării unor astfel de rețele definite prin software cu sursă deschisă. Aceste obiective sunt îndeplinite prin publicaţii, evenimente care se organizează sau demonstraţii care să arate comunităţii cazuri reale de utilizare.
\end{itemize}

Există trei tipuri de publicaţii care reies din activitățile \gls{onf}: (i) \textit{specificaţii}, care includ toate standardele care definesc un protocol, modelul informaţional, funcționalități și documente despre cadrele asociate; publicaţiile normative astfel rezultate se numesc Specificaţii Tehnice - \gls{ts} și sunt supuse unor procese de licenţiere; (ii) \textit{recomandări}, care includ consideraţii arhitecturale, cazuri de utilizare, analiza cerințelor, terminologie; aceste documente referă documente normative \gls{onf}, dar nu necesită licenţiere și pot fi utilizate în mod liber, având rolul de Recomandări Tehnice - \gls{tr}; (iii) \textit{publicaţii}, reprezentând documente care conţin informații ce ajută în procesele de lansare a \gls{sdn} în producție, rezumate ale soluţiilor, studii de caz sau cărţi albe (\textit{white papers}); aceste documente nu au caracter normativ și pot fi folosite în mod liber.

În continuare vor fi enumerate câteva astfel de documente produse de către \gls{onf}: \textit{OpenFlow Switch Specification Ver. 1.5.1} (TS-025), care descrie cerinţele unui comutator logic ce suportă protocolul OpenFlow; \textit{OpenFlow Management	and	Configuration Protocol 1.2 (OF-Config 1.2)} (TS-016), document ce descrie motivația, cerințele, scopul și specificaţiile protocolului OF-Config; \textit{Conformance Test Specification for OpenFlow Switch Specification V1.3.4} (TS-026), definind cerinţele și procedurile de test care determină conformitatea unui comutator cu specificațiile protocolului OpenFlow 1.3.4; \textit{Core Information Model (CoreModel) 1.2} (TR-512), care prezintă modelul informațional de bază, pe care celelalte modele informaționale dezvoltate în \gls{onf} se bazează; \textit{Microwave Information Model} (TR-532), reprezentând modelul informațional ce descrie echipamentele de transport de date fără fir. Aceste ultime două documente vor fi detaliate în următorul capitol, fiind baza dezvoltării și implementării simulatoarelor propuse în această lucrare.

Un rol important în adoptarea \gls{sdn} de către operatori și în diseminarea rezultatelor din cadrul \gls{onf} îl au demonstraţiile de concept. Acestea adună laolaltă operatori, producători de echipamente și integratori de servicii, cu scopul de a demonstra recomandările produse de activitatea de cercetare. În contextul rețelelor de transport de date fără fir, proiectul \gls{wt}, care face parte din grupul \gls{otwg}, a terminat cu succes patru astfel de demonstraţii de concept~\cite{onf2015_poc1, onf2016_poc2, onf2016_poc3}.

\subsection{IETF}

\subsection{ETSI}

\subsection{ITU-T}

\subsection{MEF}
\section{SDN în contextul reţelelor actuale}


\chapter{SDN \^{\i}n contextul reţelelor de transport fără fir\label{ch:sdn_in_contextul_wt}}

\section{ONF TR-532 - Modelul informaţional pentru microunde (\textit{Microwave Information Model})}

Modelul informațional pentru microunde \cite{onftr532} a apărut in decembrie 2016 ca o recomandare formulată de grupul \gls{otwg} din cadrul \gls{onf}. Scopul acestuia este de a modela un echipament de transport de date fără fir, pentru a putea fi folosit de echipamentele de control \gls{sdn}, în încercarea de a asigura o independenţă față de producătorii de echipamente. Chiar dacă este denumit \textit{model informațional pentru microunde}, acesta poate fi aplicat fără probleme nu numai echipamentelor ce funcționează în spectrul microundelor, ci și echipamentelor care funcționează în benzi de frecvenţă mai înalte (lungimi de undă milimetrice), care încep să își facă tot mai mult simţită prezenţa în rețelele actuale de transport.

TR-532 este de fapt o extensie specifică tehnologiei \gls{wt} a modelului informațional de bază, versiunea 1.2 (TR-512.1). Legătura cu acesta se face prin extinderea clasei de obiecte \textit{\gls{lp}}. Astfel, modelul informațional pentru microunde conţine şase pachete condiţionale caracteristice tehnologiilor folosite pentru transport, care au în nume extensia \textit{*\_Pac}: 

\begin{itemize}
	\item \textit{MW\_AirInterface\_Pac};
	\item \textit{MW\_AirInterfaceDiversity\_Pac};
	\item \textit{MW\_PureEthernetStructure\_Pac};
	\item \textit{MW\_HybridMWStructure\_Pac};
	\item \textit{MW\_EthernetContainer\_Pac};
	\item \textit{MW\_TdmContainer\_Pac}
\end{itemize}

O imagine de ansamblu simplificată a acestui model, în limbajul \gls{uml}, care conţine doar obiectele relevante pentru simulatoarele dezvoltate, împreună cu legătura acestuia cu modelul informațional de bază este ilustrată în Figura \ref{fig:microwave_model}.

\begin{figure}[h]
	\centering
	\includegraphics[width=1\textwidth]{microwave_model_overview}
	\caption{Reprezentare UML simplificată a \textit{MicrowaveModel} și legătura acestuia cu \textit{CoreModel} \cite{onftr532}.}
	\label{fig:microwave_model}
\end{figure}

În următoarele paragrafe se vor detalia obiectele acestui model care sunt importante din punctul de vedere al simulatoarelor dezvoltate în această lucrare.

\subsection{Obiectul \textit{MW\_AirInterface\_Pac}}

Obiectul \textit{MW\_AirInterface\_Pac} reprezintă o interfață radio fizică a unui echipament. Este denumit în recomandare ca \textit{punct de terminaţie a traseului secţiunii fizice de microunde} - \gls{mwps-ttp}, astfel că nivelul de transport al obiectului \textit{\gls{lp}} asociat este Secţiunea Fizică de Microunde - \textit{\gls{mwps}} \cite{onftr532}. O reprezentare simplificată în limbajul \gls{uml} a \textit{MW\_AirInterface\_Pac} se poate observa în Figura \ref{fig:airinterface_pac}.

\begin{figure}[h]
	\centering
	\includegraphics[width=1\textwidth]{airinterface_pac}
	\caption{Reprezentare UML simplificată a obiectului \textit{MW\_AirInterface\_Pac} \cite{onftr532}.}
	\label{fig:airinterface_pac}
\end{figure}

Acest obiect conţine alte câteva obiecte care modelează caracteristice unei interfețe radio fizice, cum ar fi: (i) capabilităţi ale modemului și ale transmiţătorului interfeţei radio asociate (de exemplu tipurile de modulaţie suportate pentru transmisie, valorile admisibile ale puterii de transmisie, intervalul de frecvenţe suportate de emiţător sau de receptor, alarmele expuse de interfață, suportul interfeţei pentru modulaţie adaptivă, etc.), (ii) parametrii configurabili ai interfeţei radio (de exemplu numele interfeţei, lărgimea de bandă a canalului de transmisie/de recepţie, frecvenţele folosite pentru transmisie/recepţie, puterea de transmisie, intervalul în care modulaţia poate lua valori, diferite alte caracteristici configurabile ale interfeţei, cum ar fi Anularea Interferenţei dintre Polarizări - \gls{xpic}, Intrări Multiple - Ieşiri multiple - \gls{mimo}, criptarea datelor, etc.), (iii) parametrii care descriu starea interfeţei la un anumit moment de timp (de exemplu frecvenţele actuale de transmisie/recepţie, nivelurile actuale de putere a semnalului de transmisie/recepţie, modulaţia actuală folosită, raportul semnal-zgomot măsurat de către modem, temperatura actuală a unităţii radio, etc.), (iv) problemele actuale ale interfeţei radio (adică alarmele care apar pe interfață la un moment dat), (v) valorile actuale ale parametrilor de performanţă a interfeţei și (vi) valorile istorice ale parametrilor de performanţă a interfeţei \cite{onftr532}.

\subsection{Obiectul \textit{MW\_PureEthernetStructure\_Pac}}

Obiectul \textit{MW\_PureEthernetStructure\_Pac} este o reprezentare logică a unei interfețe radio capabilă să transporte doar trafic Ethernet. Acest obiect este reprezentat într-un mod simplificat, în limbajul \gls{uml}, în Figura \ref{fig:pureethstructure_pac}. Asocierea cu o interfață fizică radio se face la nivelul modelului informațional de bază, printr-o relaţie de tip client-server.

\begin{figure}[h]
	\centering
	\includegraphics[width=1\textwidth]{pureethstructure_pac}
	\caption{Reprezentare UML simplificată a obiectului \textit{MW\_PureEthernetStructure\_Pac} \cite{onftr532}.}
	\label{fig:pureethstructure_pac}
\end{figure}

Acest obiect este denumit în recomandare ca \textit{punct de terminaţie a traseului secţiunii microunde} - \gls{mws-ttp}, astfel că nivelul de transport al obiectului \textit{\gls{lp}} asociat este Secținea de Microunde - \textit{\gls{mws}}. 

Structura obiectelor conţinute de către \textit{MW\_PureEthernetStructure\_Pac} este similară cu cea a obiectului \textit{MW\_AirInterface\_Pac}. Conţine obiecte care reprezintă (i) capabilitățile acestei interfețe logice (de exemplu alarmele aplicabile ei sau identificatorul structurii respective, care poate fi folosit de alte obiecte), (ii) parametrii configurabili ai interfeţei logice (de exemplu gradul de severitate a alarmelor pe care această interfață le expune), (iii) parametrii care descriu starea interfeţei logice la un anumit moment de timp, (iv) problemele actuale ale interfeţei logice, (v) valorile actuale ale parametrilor de performanţă a interfeţei logice și (vi) valorile istorice ale parametrilor de performanţă a interfeţei logice \cite{onftr532}.

\subsection{Obiectul \textit{MW\_EthernetContainer\_Pac}}

Obiectul \textit{MW\_EthernetContainer\_Pac} reprezintă de asemenea o interfață logică și este denumit în recomandare \textit{punct de terminaţie a conexiunii unui client de microunde}, pentru un semnal Ethernet client. Practic, este o interfață logică ce are rol de container pentru traficul Ethernet care este transmis de echipament prin radio. În raport cu obiectul \textit{\gls{lp}} acesta are un nivel de transport denumit Container Ethernet - \textit{\gls{etc}}. O reprezentare grafică simplificată în limbajul \gls{uml} a obiectului \textit{MW\_EthernetContainer\_Pac} poate fi găsită în Figura \ref{fig:ethcontainer_pac}.

\begin{figure}[h]
	\centering
	\includegraphics[width=1\textwidth]{ethcontainer_pac}
	\caption{Reprezentare UML simplificată a obiectului \textit{MW\_EthernetContainer\_Pac} \cite{onftr532}.}
	\label{fig:ethcontainer_pac}
\end{figure}

Și în cazul obiectului \textit{MW\_EthernetContainer\_Pac} se păstrează aceeaşi structura a obiectelor pe care le conţine, ca în cazul celorlalte două obiecte detaliate anterior. Astfel, acesta prezintă obiecte care reprezintă (i) capabilitățile containerului (de exemplu dacă există compresie la diferite niveluri, criptare a datelor sau alarmele pe care această interfață le expune), (ii) parametrii configurabili ai containerului (de exemplu un identificator al containerului, identificatoarele segmentelor folosite pentru a transporta traficul Ethernet asociat acestui container, etc.), (iii) parametrii care descriu starea containerului, (iv) alarmele la momentul actual de timp pe care containerul le raportează, (v) valorile actuale ale parametrilor de performanţă a containerului și (vi) valorile istorice ale parametrilor de performanţă a containerului \cite{onftr532}.
\section{Modelul informaţional de bază - ONF TR-512}

SDN și reţelele actuale..
\section{Protocolul NETCONF și limbajul YANG}

Network Configuration Protocol - \gls{netconf} - este un protocol de administrare a echipamentelor de rețea care oferă posibilitatea de a instala, manipula și şterge configuraţii ale echipamentelor de rețea. A fost prima oară definit în RFC 4741 \cite{rfc4741}, în decembrie 2006, ca mai apoi să fie revizuit în RFC 6241 \cite{rfc6241}, în iunie 2011. A apărut ca o nevoie a industriei rețelisticii, care până atunci utiliza alte modalităţi de administrare, precum \gls{cli}, \gls{snmp} sau \gls{corba} \cite{yu2010empirical}.

Chiar dacă tehnicile folosite până acum pentru administrarea echipamentelor sunt utilizate de mai bine de douăzeci de ani, acestea și-au dovedit limitările. De exemplu, prima abordare, cea folosind interfețe cu linie de comandă, implică o foarte mare dependenţă de producătorii de echipamente \cite{elbadawi2011improving}, în timp ce protocolul \gls{snmp} este folosit îndeosebi pentru colectarea de alarme și de valori ale parametrilor de performanţă de la echipamente și mai puţin pentru configurarea acestora. \gls{netconf} încearcă să diminueze aceste limitări propunând o abordare nouă și inovativă. În acest scop a fost dezvoltat și limbajul \gls{yang}, în RFC 6020 \cite{rfc6020}, care să modeleze datele folosite de acest protocol.

În următoarele secţiuni se vor detalia modul de funcţionare a protocolului \gls{netconf}, punând accent pe aspectele relevante din punctul de vedere al simulatoarelor dezvoltate în această lucrare și limbajul de modelare \gls{yang}.

\subsection{NETCONF}

Protocolul \gls{netconf} defineşte un mecanism simplu de administrare a echipamentelor de rețea, prin care parametrii configurabili pot fi ceruţi dispozitivelor, manipulaţi și apoi retrimiși pentru configurarea echipamentelor. Cu ajutorul acestuia dispozitivele pot expune interfețe de programare care să poată fi utilizate de aplicații software pentru administrare.

Mecanismul propus de \gls{netconf} constă în utilizarea paradigmei apelurilor de proceduri la distanţă - \gls{rpc}. Un server rulează pe fiecare dispozitiv, aşteptând conexiuni de la clienţi prin canale securizate și folosind protocoale orientate spre conexiune, precum \gls{ssh}, \gls{tls}, \gls{beep}, \gls{soap} \cite{ji2009challenges}. Clienţii își codează acel apel de procedură la distanţă cu ajutorul limbajului \gls{xml}, iar serverul răspunde printr-un mesaj codat cu același limbaj. În timpul stabilirii conexiunii între un client și un server \gls{netconf}, acesta din urmă își expune capabilitățile pe care le are, astfel încât clientul va fi informat asupra acestora, având posibilitatea de a-și schimba comportamentul pentru a profita de caracteristicile serverului.

\gls{netconf} propune folosirea a două tipuri de date: (i) date care descriu starea unui dispozitiv, adică valorile parametrilor operaţionali sau valori de contorizare a parametrilor de performanţă, toate acestea fiind valori care pot fi doar citite din echipamente și (ii) date configurabile pe dispozitiv, care pot fi atât scrise cât și citite. Pentru manipularea acestora se oferă diferite operații, cele mai importante fiind:

\begin{itemize}
	\item \textit{get} - această operaţie permite unui client să ceară unui dispozitiv atât date de stare, cât și parametri configurabili. Evident, se pot aplica filtre pentru a întoarce doar anumiţi parametri dintre cei pe care îi expune echipamentul;
	\item \textit{get-config} - această operaţie oferă unui client posibilitatea să ceară unui server doar parametri configurabili, excluzându-i pe cei de stare;
	\item \textit{edit-config} - aceasta este o operaţie prin care un client poate schimba valorile unor parametri configurabili pe un dispozitiv.
\end{itemize}

Pentru stocarea acestor date pe echipament, protocolul \gls{netconf} propune la nivel conceptual un loc în care să se stocheze și prin care să se acceseze informaţia, denumit \textit{datastore}. Reprezentarea acestuia este la alegerea fiecărui dispozitiv, putând fi implementat prin fişiere, baze de date, locaţii de memorie flash, etc. Există trei tipuri de astfel de locuri pentru stocarea datelor:

\begin{itemize}
	\item \textit{startup datastore} - acest tip reprezintă un \textit{datastore} ce poate conţine valori implicite ale parametrilor configurabili ai dispozitivului de rețea și poate fi încărcată în echipament în momentul inițializării acestuia. El este prezent doar în dispozitivele care suportă această separare între configurația curentă, care rulează în echipament și configurația care se încarcă în momentul inițializării;
	\item \textit{running datastore} - acest tip reprezintă toţi parametrii configurabili activi la momentul curent, care sunt prezenţi în dispozitiv. Acest tip există întotdeauna pe un echipament;
	\item \textit{candidate datastore} - acest tip reprezintă o copie a tuturor parametrilor configurabili ai unui dispozitiv. Modificarea acestora nu influenţează configurația curentă a dispozitivului, însă aceasta se poate aplica, prin copierea cu ajutorul unei tranzacţii, înlocuind \textit{running datastore}. Nu toate tipurile de echipamente suportă această capabilitate.
\end{itemize}

\gls{netconf} propune și un mecanism prin care să protejeze accesul concurent la scrierea parametrilor configurabili ai dispozitivelor. Astfel, un client poate bloca o parte sau chiar toată \textit{running datastore} cât timp execută operații prin care schimbă valorile acestor parametri, oferind astfel protecție datelor pe care le modifică. Dispozitivul însă trebuie să se asigure că aceste valori nu pot fi modificate în același timp prin alte căi, cum ar fi \gls{snmp} sau \gls{cli}.

Acest protocol oferă și posibilitatea serverelor \gls{netconf} să trimită informări către clienţi cu privire la anumite evenimente care se petrec în dispozitiv. Spre deosebire de celelalte mesaje, aceste notificări sunt iniţiate de către server și sunt trimise tuturor clienților și-au exprimat dorinţa de a le primi (prin trimiterea către server a unui mesaj de abonare). 

\subsection{YANG} 

Yet Another Next Generation - \gls{yang} este un limbaj de modelare a informaţiei dezvoltat specific pentru protocolul \gls{netconf}. Acesta descrie atât datele de configuraţie și de stare pe care un dispozitiv le poate expune pentru a fi folosite de către protocol, cât și apelurile de proceduri la distanţă sau notificările. A apărut în octombrie 2010 ca RFC 6020, fiind dezvoltat de \gls{ietf}.

În momentul apariției \gls{netconf}, dat fiind faptul că se baza pe limbajul \gls{xml}, soluţia naturală pentru definirea modelului de date folosit de protocol era utilizarea soluţiilor \gls{xml} existente pentru modelarea informaţiei, precum schemele \gls{xml} sau Relax NG \cite{ji2009challenges}. Aceste soluții aveau însă dezavantajele de a fi greu de folosit și de a avea o lizibilitate redusă.

Astfel, un nou limbaj a fost dezvoltat: \gls{yang}. Au fost considerate mai multe aspecte în dezvoltarea acestuia: lizibilitatea, o abordare orientată pe obiecte și o oarecare similaritate cu limbajele de programare. \gls{yang} îndeplineşte aceste condiţii, fiind folosit pentru a descrie ierarhii de noduri, care pot reprezenta notificări, apeluri de proceduri la distanţă sau parametri de stare sau de configuraţie și pot fi folosite de operaţiile \gls{netconf}. Informaţiile sunt stocate în modele \gls{yang} și precum în limbajele de programare, un model poate include date din alt model, oferind astfel posibilitatea de a crea seturi de modele de date generice și reutilizabile\cite{nataf2010end}. Acestea descriu atât nodurile într-un mod concis și clar, cât și interacţiunile dintre ele \cite{cui2009contrast}.

\gls{yang} descrie informaţia într-un mod ierarhic, astfel că un fiecare nod are, pe de o parte, un nume și pe de altă parte o valoare sau un set de noduri copil. Se pot descrie și constrângeri ce pot fi aplicate asupra apariției sau valorii unor noduri, bazându-se pe prezenţa sau valoarea altor noduri ale ierarhiei.

Există mai multe tipuri de noduri definite în limbajul \gls{yang}. Cele relevante din punctul de vedere al simulatoarelor implementate în această lucrare sunt:

\begin{itemize}
	\item \textit{grouping} - acesta este, așa cum o sugerează și numele, un nod care reprezintă o grupare de noduri. După ce este definit el poate fi utilizat în același sau în alte module sau sub-module;
	\item \textit{list} - acest tip de nod \gls{yang} defineşte o listă de noduri, iar intrările în această listă sunt distinse prin noduri care reprezintă cheia intrării respective;
	\item \textit{typedef} - este folosit pentru definirea unui tip de date care poate fi utilizat ulterior de alte noduri;
	\item \textit{rpc} - este folosit pentru modelarea apelurilor de proceduri la distanţă, prin definirea numelui procedurii și a parametrilor de intrare și de ieșire;
	\item \textit{notification} - acest tip de nod se folosește pentru descrierea unei notificări \gls{netconf} pe care un server o poate genera, prin modelarea conținutului acesteia;
	\item \textit{leaf} - nodurile frunză reprezintă nivelul cel mai jos al ierarhiei si descriu un parametru al dispozitivului, care poate fi de stare (poate fi doar citit) sau de configurare (poate fi și scris și citit).
\end{itemize}

Așa cum a fost prezentat anterior, \gls{onf} dezvoltă modelele informaționale pe care le recomandă cu ajutorul limbajului \gls{uml}, care este mult mai general si mai puţin specializat decât \gls{yang}. Însă, pentru a putea fi folosite într-un mod facil, \gls{onf} a dezvoltat și o unealtă software care să transforme modelele din limbajul \gls{uml} în limbajul \gls{yang}, împreună cu o recomandare despre cum această transformare ar trebui făcută \cite{onftr531}.
\section{Alegerea unui cadru software pentru serverul NETCONF}

Există numeroase soluții care lucrează cu protocolul \gls{netconf}, atât pe partea de client, cât și pe cea de server. O listă a acestora este menţinută de către grupul de lucru \gls{netconf} și poate fi găsită online \cite{netconfwiki}. Conţine și soluții software proprietare, dar și soluții cu sursă deschisă. Pentru implementarea simulatoarelor prezentate în această lucrare au fost considerate trei opţiuni de implementare a unui server \gls{netconf}, cu sursă deschisă: \textit{Netopeer}, \textit{OpenYuma} și \textit{\gls{netconf} Test Tool} (unealtă oferită de proiectul \gls{odl}). O comparaţie între acestea va fi prezentată în continuare, justificând astfel alegerea de a folosi una dintre ele în simulatoare \cite{stancu2016comparison}.

\subsection{Netopeer}

\textit{Netopeer} este o soluție ce se bazează pe librăria \textit{libnetconf}, oferind atât o implementare pentru server, cât și una pentru client. Această librărie este una cu sursă deschisă, implementată în limbajul C, ce oferă o implementare a protocolului \gls{netconf} \cite{krejci2013building}. Este o soluție care poate fi personalizată, oferind numeroase posibilităţi pentru implementările de server și de client și suportă toate caracteristicile protocolului \gls{netconf}.

\textit{Netopeer} oferă câteva unelte, de exemplu pentru a facilita integrarea modelelor informaționale \gls{yang} în module ale serverului, denumită \textit{netopeer-manager}, sau pentru a configura caracteristicile serverului \gls{netconf}, \textit{netopeer-configurator}. Orice model de date \gls{yang} poate fi adăugat ca un modul al serverului, însă acesta trebuie prelucrat înainte. Astfel, fişierul \textit{*.yang} este transformat de către această soluție în fişiere pe care serverul le poate recunoaşte, inclusiv un fişier \textit{*.c}, care conţine un schelet de cod C, reprezentând așa-numite funcții de apel invers (\textit{callback functions}) ce pot fi implementate pentru ca serverul să ofere comportamentul dorit, în raport cu modelul \gls{yang} folosit. Apoi, codul C este compilat, rezultând o bibliotecă partajată (\textit{shared library}) care poate fi utilizată de către codul de bază al serverului.

\subsection{OpenYuma}

\textit{OpenYuma} este o soluție software care se bazează pe proiectul \textit{Yuma}, care a devenit proprietar în anul 2011. Propune de asemenea implementări pentru server și client \gls{netconf}, scrise în limbajul C, oferind chiar posibilitatea de a încorpora acest cod în dispozitive al căror software folosește tot limbajul C.

\textit{OpenYuma} are o filosofie asemănătoare cu \textit{Netopeer}, oferind unelte pentru transformarea modelelor \gls{yang} în cod C schelet, care să fie apoi implementat pentru ca serverul \gls{netconf} să ofere facilităţile propuse. Unealta propusă de acest cadru software se numeşte \textit{yangudmp} și transformă fişierele \textit{*.yang} în fişiere \textit{*.h} și \textit{*.c}, conţinând, la fel ca în cazul \textit{Netopeer}, funcții de apel invers ce trebuie rescrise.

Codul C obţinut după transformarea modelelor \gls{yang} se compilează, rezultând tot o bibliotecă partajată care să poată fi folosită de către codul de cază al serverului. Aceasta poate fi încărcată în momentul inițializării serverului sau chiar în mod dinamic, în timp ce acesta rulează.

\subsection{Netconf Test Tool}

\textit{Netconf Test Tool} este un cadru software oferit în cadrul proiectului OpenDaylight. Este o soluție simplă, care nu poate fi personalizată foarte mult, oferind doar o implementare Java pentru un server \gls{netconf}. Aceasta este folosită de proiectul \gls{odl} pentru a-și testa interfața de tip \textit{southbound} care implementează protocolul \gls{netconf}.

Scopul acestei soluții este puțin diferit de al celorlalte, deoarece \textit{Netconf Test Tool} nu își propune oferirea unui cadru software pentru un server \gls{netconf} care apoi să poată fi integrat cu echipamentele de rețea, ci oferirea unei soluții simple și rapide care să încarce un model \gls{yang} specific, cu scopul de a-l testa. Cu toate acestea, acest software a fost considerat pentru comparaţie, deoarece și scopul simulatoarelor este de a testa modelele \gls{yang} și de a crea topologii specifice rețelelor de transport de date fără fir, expunând modelele informaționale descrise anterior.

\subsection{Comparaţie între soluţiile care oferă server NETCONF}

Soluţiile descrise anterior au fost evaluate atât prin compararea documentaţiei relevante pe care acestea o pun la dispoziţie, cât și prin experimente practice care consideră diferite scenarii. Primul criteriu care poate fi considerat este limbajul pe programare în care aceste cadre software sunt implementate: \textit{Netopeer} și \textit{OpenYuma} sunt scrise în limbajul C, pe când \textit{Netconf Test Tool} este o implementare Java.

Un alt criteriu pentru evaluare constă în abilitatea serverului de a încărca în mod direct (dinamic sau în momentul inițializării) modele \gls{yang}. Această posibilitate este oferită doar de implementarea Java. Celelalte două soluții au nevoie de o fază premergătoare de procesare,transformând fişierele \textit{*.yang} în \textit{*.c}. \textit{Netconf Test Tool} poate încărca modelul \gls{yang} doar în momentul inițializării serverului, dintr-un director anume, pe când celelalte cadre software pot încărca acest model, după ce a fost procesat, în mod dinamic.

Un alt subiect pentru comparaţie este dat de tipurile de locuri de stocare a datelor (\textit{datastore}) propuse de protocolul \gls{netconf} suportate de implementările serverelor. \textit{Netopeer} și \textit{OpenYuma} folosesc fişiere \gls{xml} în care stochează informaţiile și suportă toate cele trei tipuri de \textit{datastore} propuse de \gls{netconf}: \textit{startup}, \textit{candidate} și \textit{runnning}. Cealaltă soluție software, \textit{Netconf Test Tool} oferă posibilitatea de a utiliza doar \textit{running datastore} stocând valorile parametrilor în variabile de execuţie, acestea pierzându-se în momentul în care serverul este oprit. O diferenţă importantă apare în acest context, între cele două soluții implementate în C, \textit{OpenYuma} oferind o flexibilitate mai mare. În cazul \textit{Netopeer}, atunci când serverul se iniţializează, încarcă valorile parametrilor din \textit{startup datastore} și \textit{running datastore} în memorie. Apoi, începe să analizeze valorile din \textit{startup datastore}, comparându-le cu valorile corespunzătoare atributelor din \textit{running datastore}. Dacă valorile nu sunt egale, sau valoarea din \textit{running datastore} nu există (însemnând prima utilizare a serverului), atunci serverul copiază valoarea din \textit{startup datastore} și apelează funcţia de apel invers asociată parametrului de configurare. Prin această abordare severul se asigură ca nu există inconsistenţe între \textit{startup datastore} și \textit{running datastore} și, mai mult, dacă acest server este conectat la un echipament real de rețea, prin apelarea funcţiei asociate parametrului, dispozitivul va fi configurat astfel încât valorile din server să reflecte valorile de pe echipament. \textit{OpenYuma} are o abordare mai flexibilă, permiţând dezvoltatorilor să altereze \textit{running datastore} în timpul inițializării modulului, fără a implica \textit{startup datastore}.

O altă caracteristică importantă a serverelor care poate fi comparată este abilitatea de a genera notificări \gls{netconf}. Toate cele trei soluții oferă notificări. În cazul \textit{Netconf Test Tool} este ceva simplist, acestea fiind declanşate printr-o comandă \gls{netconf} și trimise dintr-un fişier \gls{xml} care le conţine. \textit{OpenYuma} oferă câte o funcție cu apel invers pentru fiecare notificare pe care o găseşte în momentul procesării modelului \gls{yang}. \textit{Netopeer} nu are această abilitate în mod implicit, însă suportul este oferit de \textit{libnetconf}.

Din perspectiva funcţiilor cu apel invers generate în momentul procesării modelului \gls{yang} putem compara doar cele două soluții implementate în limbajul C, deoarece \textit{Netconf Test Tool} nu oferă astfel de caracteristici. Soluția \textit{Netopeer} oferă posibilitatea de a alege care dintre parametrii configurabili vor avea o funcție cu apel invers asociată, în timp ce \textit{OpenYuma} va genera o astfel de funcție pentru fiecare din parametrii configurabili pe care îi găseşte în modelul \gls{yang}. Pentru informaţiile de stare \textit{Netopeer} generează o singură funcție cu apel invers care este apelată pentru orice operaţie \gls{netconf} de tip \textit{get} care ajunge la server, iar \textit{OpenYuma} generează câte o astfel de funcție pentru fiecare parametru de stare.

Din punctul de vedere al abilităţii de a configura portul pe care serverul ascultă, toate soluțiile oferă această posibilitate. \textit{Netconf Test Tool} are nevoie de un port de plecare, incrementând apoi numărul portului pentru celelalte instanţe ale serverului. Și celelalte soluții oferă posibilitatea de a schimba portul pe care serverul ascultă. Din perspectiva rulării mai multor instanţe de server pe aceeaşi mașină, atât \textit{Netconf Test Tool} cât și \textit{OpenYuma} oferă această posibilitate, diferitele instanţe folosind porturi diferite. \textit{Netopeer}, pe de altă parte, nu permite rularea mai multor instanţe pe aceeaşi mașină.

Un alt criteriu pentru compararea acestor cadre software este dat de suportul pentru mai multe fire de execuție, adică abilitatea serverului de a permite conectarea mai multor clienţi în același timp. \textit{Netconf Test Tool} oferă această posibilitate. \textit{OpenYuma} oferă de asemenea acest suport, cu menţiunea că accesul concurent la resursele comune trebuie rezolvat în funcțiile cu apel invers care vor fi implementate de către dezvoltatori, nefiind rezolvat de către cadrul software. În cazul \textit{Netopeer}, acest acces concurent este rezolvat din faza de proiectare a serverului și este oferit de \textit{libnetconf}.

Și capabilitățile de depanare oferite ar putea constitui un criteriu de comparaţie, însă toate soluţiile se bazează pe fişiere de tip jurnal, oferind mai multe niveluri de jurnalizare. 

Comparaţia bazată pe lucrurile descrise în documentaţie este rezumată în Tabelul \ref{tab:Table_1}, în timp ce un sumar al comparaţiei bazată pe experimentare se găseşte în Tabelul \ref{tab:Table_2}.

\begin{table}[tp]
	\caption{Comparaţie a caracteristicilor oferite de cadrele software considerate.\label{tab:Table_1}}
	
	\begin{tabular}{|M{0.35\textwidth}|M{0.17\textwidth}|M{0.17\textwidth}|M{0.16\textwidth}|}
			\hline 
			\textbf{Criteriile} & \multicolumn{3}{c|}{\textbf{Soluții servere NETCONF}} \tabularnewline
			\cline{2-4} 
			\textbf{de comparație} & \textbf{\emph{Netopeer}} & \textbf{\emph{OpenYuma}} & \textbf{\emph{Testtool}}\tabularnewline
			\hline 
			Limbajul de programare & C & C & Java\tabularnewline
			\hline 
			Încărcarea modelelor YANG brute & Nu & Nu & Da\tabularnewline
			\hline 
			Încărcarea dinamică a modulelor în server & Da & Da & Nu\tabularnewline
			\hline 
			NETCONF \textit{datastore} & toate & toate & \textit{running} \tabularnewline
			\hline 
			Suport pentru notificări & da & da & da\tabularnewline
			\hline 
			Port configurabil & da & da & da\tabularnewline
			\hline 
			Mai multe instanțe de server & nu & da & da \tabularnewline
			\hline 
			Mai multe conexiuni în același timp & da & da & da\tabularnewline
			\hline 
			Capabilități pentru depanare & jurnalizare & jurnalizare & jurnalizare\tabularnewline
			\hline
		\end{tabular}
\end{table}

\begin{table}[tp]

	\caption{Compararea practică a cadrelor software considerate\label{tab:Table_2}}
	\begin{tabular}{|M{0.35\textwidth}|M{0.17\textwidth}|M{0.17\textwidth}|M{0.16\textwidth}|}
		\hline
		\textbf{Scenariul experimentat} & \textbf{\emph{Netopeer}} & \textbf{\emph{OpenYuma}} & \textbf{\emph{Testtool}} \tabularnewline
		\hline 
		Procesarea \textit{*.yang} în \textit{*.c} & lnctool & yangdump & N/A\tabularnewline
		\hline 
		Reprezentarea \textit{datastore} în cadrul serverului & fişier XML & fișier XML & variabile de execuție \tabularnewline
		\hline 
		Încărcarea datelor în faza de iniţializare a serverului & \textit{startup} & flexibilă, orice se poate suprascrie & N/A \tabularnewline
		\hline 
		Implementarea notificărilor NETCONF & \textit{libnetconf} & funcție cu apel invers oferită & fișier XML \tabularnewline
		\hline 
		Funcții pentru parametrii configurabili & câte una per atribut & câte una per atribut & N/A\tabularnewline
		\hline 
		Funcții pentru parametrii de stare & doar una, globală & câte una per atribut & N/A\tabularnewline
		\hline 
		Conexiuni concurente de la clienți & da & suportă, trebuie implementat & da\tabularnewline
		\hline \end{tabular}
\end{table}

\section{Arhitectura demonstraţiilor de concept WT SDN}

Așa cum a fost amintit anterior, proiectul \textit{Wireless Transport} din grupul Open Transport Working Grup din cadrul \gls{onf} se ocupă cu cercetare legată de \gls{sdn} în contextul rețelelor de transport de date fără fir. După cum este prezentat și în \cite{bercovich2015software, bernardos2014architecture}, acest tip de rețele sunt o parte importantă atât din perspectiva rețelelor curente, cât și din cea a rețelelor viitorului, precum cele 5G.

După efectuarea unei demonstraţii de concept \cite{onf2015_poc1} care folosea protocolul OpenFlow pentru a arăta capabilitățile \gls{sdn} în rețelele de transport de date fără fir, observând dificultăţile pe care le presupune lucrul cu OpenFlow în acest context, grupul a trecut la dezvoltarea unui model informațional care să abstractizeze dispozitivele ce fac parte din aceste rețele, model care a fost descris anterior.

Apoi, alte trei demonstraţii de concept au avut loc, descrise în \cite{onf2016_poc2, onf2016_poc3, onf2017_poc4}, care au urmărit evoluția acestui model informațional. Chiar dacă în cel de-al doilea \gls{poc} s-a folosit un model de date mult simplificat, iar în cel de-al patrulea a fost folosită cea mai bună și matură variantă a sa, arhitectura acestor demonstraţii a fost similară.

\begin{figure}[h]
	\centering
	\includegraphics{poc_architecture}
	\caption{Arhitectura demonstraţiilor de concept WT.}
	\label{fig:poc_architecture}
\end{figure}

După cum se poate observa în Figura \ref{fig:poc_architecture}, în arhitectura acestor demonstraţii apare nevoia unui nivel intermediar, de mediator. Acest mediator este reprezentat de o aplicație software care expune o interfață \gls{netconf} către nord, unde se conectează echipamentul de control \gls{sdn}, folosind modelul informațional descris de TR-532. Rolul mediatorului este de a traduce comenzile \gls{netconf} care vin de la controler în comenzi pe care dispozitivul de rețea le poate înţelege (de exemplu \gls{snmp}, \gls{cli} sau \gls{rest}). Această nevoie apare din cauza faptului că echipamentele din rețelele actuale, care au fost folosite și pentru demonstraţiile de concept, nu suportă încă noul model informațional în mod nativ, deoarece abia a fost dezvoltat. Cel mai probabil, în viitor această nevoie va dispărea, echipamentele putând să încadreze \textit{Microwave Model} în aplicaţia software proprie de control.

\begin{figure}[t]
	\centering
	\includegraphics{poc_architecture_simulator}
	\caption{Poziţionarea simulatoarelor în arhitectura demonstraţiilor de concept WT.}
	\label{fig:poc_architecture_simulator}
\end{figure}


Poziţionarea simulatoarelor în arhitectura acestor demonstraţii de concept este ilustrată în Figura \ref{fig:poc_architecture_simulator}. Astfel, ele pot fi folosite de către dezvoltatorii de aplicații \gls{sdn} pentru testarea aplicațiilor care utilizează interfaţa \gls{netconf} ce implementează modelul informațional TR-532, eliminând nevoia acestora de a deţine dispozitive pentru transportul de date fără fir și mediatorul asociat acestora.

\chapter{Mediatorul cu valori implicite - prima versiune\label{ch:dvm_v01}}

\section{Arhitectura DVM versiunea 1}

Prima versiune a mediatorului cu valori implicite expune către echipamentul de control \gls{sdn} modelul informațional simplificat pentru microunde, care a fost folosit în cea de-a doua demonstraţie de concept efectuată de proiectul \gls{wt} din \gls{onf} \cite{stancu2016default}.

Implementarea acestuia este una simplă și modulară, bazată pe soluţia \textit{OpenYuma}, oferind câte o funcție cu apel invers pentru fiecare atribut din modelele \gls{yang} expuse, indiferent de natura lor (parametri de configurare sau de stare). În aceste funcții se stabileşte valoare atributului respectiv, care este returnată unui client \gls{netconf}. Excepţia de la această regulă este dată de câteva atribute ale căror valori sunt definite într-un fişier de configurare folosit de către mediator. Dezavantajul acestei abordări este că dacă un dezvoltator de aplicații are nevoie de o altă valoare a unui atribut care nu face parte din acest fişier de configurare, va trebui să o modifice în funcţia asociată parametrului respectiv și apoi să recompileze codul asociat modului de server care conţine acel parametru și apoi să încarce din nou acel modul în server.

O imagine de ansamblu a arhitecturii primii versiuni a \gls{dvm} este ilustrată în Figura \ref{fig:dvm_v01_architecture}. Acesta se bazează pe un fişier de configurare ce conţine câțiva parametri importanţi din punctul de vedere al aplicațiilor \gls{sdn}. Aceștia sunt: (i) numele echipamentului de rețea - \textit{Network Element Name} și (ii) un identificator unic folosit pentru fiecare legătură radio - \textit{Radio Signal ID}.

\begin{figure}[h]
	\centering
	\includegraphics{dvm_v01_architecture}
	\caption{Arhitectura primei versiuni a DVM \cite{stancu2016default}.}
	\label{fig:dvm_v01_architecture}
\end{figure}

Aceşti parametri sunt importanţi pentru aplicații deoarece cu ajutorul lor se pot identifica în mod unic dispozitivele din rețea. Prin pregătirea mai multor fişiere de configurare se pot simula mai multe elemente de rețea, fără a fi nevoie de recompilarea simulatorului. Astfel, vor rula mai multe instanţe ale \gls{dvm}, fiecare având propriul fişier de configurare ce conţine un alt nume pentru dispozitiv și alte identificatoare pentru legăturile radio.

În această arhitectură, \gls{dvm} este configurat să trimită și notificări \gls{netconf} fictive către utilizatorii care s-au abonat la primirea acestora. Fişierul de configurare conţine astfel și o valoare numerică reprezentând numărul de secunde dintre două notificări fictive consecutive. Dacă această valoare este mai mare decât zero, se vor trimite notificări fictive la intervalul de timp specificat. Altfel, dacă valoarea este zero, \gls{dvm} nu va trimite astfel de mesaje \gls{netconf}.
\section{Implementarea DVM versiunea 1}

Primul pas al implementării a fost reprezentat de alegerea modelelor \gls{yang} care să fie expuse de serverul \gls{netconf}. În cea de-a doua demonstraţie de concept \gls{wt} a fost agreată folosirea unor modele informaționale reduse, care să permită demonstrarea cazurilor de utilizare alese \cite{onf2016_poc2}. Acestea conţineau aproximativ şaizeci de atribute care făceau parte atât din modelul informațional de bază, cât și din modelul informațional pentru microunde. Astfel, au fost alese trei modele \gls{yang} pentru implementarea în cadrul primei versiuni a \gls{dvm}: \textit{CoreModel-CoreNetworkModule-ObjectClasses}, \textit{MicrowaveModel-ObjectClasses-MwConnection} și \textit{MicrowaveModel-Notifications}. Asta a însemnat, practic, generarea a trei biblioteci partajate reprezentând module ale serverului \gls{netconf}, care să poată fi încărcate în soluţia \textit{OpenYuma} și să ofere capabilitățile dorite.

O organigramă a fazei de dezvoltare și implementare a \gls{dvm} este ilustrată în Figura \ref{fig:dvm_v01_workflow}.

\begin{figure}[h]
	\centering
	\includegraphics{dvm_v01_workflow}
	\caption{Organigramă a dezvoltării și implementării DVM \cite{stancu2016default}.}
	\label{fig:dvm_v01_workflow}
\end{figure}

Cel de-al doilea pas al implementării a constat în procesarea modelelor \gls{yang} alese și generarea codului C schelet al modulelor asociate acestora. Acest lucru a fost realizat cu utilitarul \textit{yangdump} oferit de soluţia \textit{OpenYuma}. Pentru a îmbunătăţi flexibilitatea \gls{dvm} și pentru a avea o mai bună separare între codul folosit de server și codul utilizatorului, care trebuie rescris, utilitarul a fost folosit cu opţiunea \textit{--split}. Astfel, pentru fiecare modul au fost generate patru fişiere: câte unul \textit{.c} și \textit{.h} pentru codul de server, respectiv pentru cel de utilizator.

Următorul pas a fost reprezentat de implementarea funcţiilor cu apel invers generate pentru fiecare atribut al modelului \gls{yang}. Pentru a asocia câte o valoare implicită fiecărui parametru, funcțiile au fost modificate astfel încât să întoarcă valoarea respectivă în momentul apelării.

Cea mai importantă parte a implementării \gls{dvm} a constat în construirea bazei de stocare a datelor de operare, reprezentând de fapt arborele atributelor \gls{yang} pe care serverul \gls{netconf} le va utiliza atunci când va fi interogat de către echipamentul de control \gls{sdn}. Nu a fost posibilă construirea automată a acestui arbore de parametri, astfel că atributele au fost implementate manual, de la rădăcină către frunze. Pentru acest lucru a fost nevoie să se altereze funcţia de iniţializare \textit{init2()} generată automat pentru fiecare model \gls{yang}. A fost necesară adăugarea câte unui nod \textit{OpenYuma} pentru fiecare parametru \gls{yang}, având asociată o funcție cu apel invers în care se setează valoarea implicită a acelui atribut. În cazul parametrilor anterior menţionaţi, care fac parte din fişierul de configurare, implementarea funcţiei constă în citirea valorii respective din acel fişier. Această abordare a permis utilizatorilor \gls{dvm} un acces facil la valorile implicite asociate fiecărui atribut. Dacă un utilizator avea nevoie de schimbarea valorii unui parametru \gls{yang}, o putea face uşor prin funcţia cu apel invers asociată sau prin fişierul de configurare, fără să fie nevoit să ştie detaliile de implementare referitoare la construirea arborelui de valori.

Următorii paşi ai implementării sunt simpli și direcţi: codul rezultat se compilează, rezultând bibliotecile partajate care apoi sunt încărcate în serverul \gls{netconf} (mai exact în procesul \textit{netconfd} asociat acestuia).

Etapele anterior menţionate se aplică în cazul modelului informațional de bază și în cazul modelului informațional pentru microunde, excluzând cazul notificărilor \gls{netconf}, unde abordarea este puţin diferită.

Deoarece modelul asociat notificărilor \gls{netconf}, \textit{MicrowaveModel-Notifications}, are o structură diferită, conţinând obiecte \gls{yang} ce reprezintă notificări în locul atributelor obişnuite, comportamentul soluţiei \textit{OpenYuma} este diferit în acest caz. În loc să se genereze funcții cu apel invers pentru obţinerea și setarea valorilor atributelor, în cazul notificărilor soluţia \textit{OpenYuma} va genera funcții cu apel invers folosite pentru declanşarea acestora (trimiterea lor de către server tuturor utilizatorilor care s-au abonat).

Pentru implementarea notificărilor \gls{netconf} în \gls{dvm}, un nou fir de execuţie s-a creat în funcţia de iniţializare a modulului, \textit{init2()}. Acesta rulează o singură funcție care implementează o buclă infinită în care se folosește funcţia cu apel invers asociată pentru a trimite o notificare fictivă, la un interval de secunde definit în fişierul de configurare. În cazul în care valoarea intervalului este zero, declanşarea notificărilor nu va fi activată. Detaliile conţinute în notificarea \gls{netconf} fictivă se găsesc în interiorul funcţiei care implementează generarea și modificarea acestora nu este banală pentru un utilizator neexperimentat.

Structura fişierului de configurare este foarte simplă și nu oferă prea multă flexibilitate utilizatorilor \gls{dvm}. Aceasta este fixă și poate fi observată în Figura \ref{fig:dvm_v01_config}, într-un exemplu în care un dispozitiv conţine două interfețe radio. Conţine doar trei tipuri de parametri, așa cum a fost menţionat anterior: numele echipamentului de rețea (\textit{NeName}), identificatoarele legăturilor radio (\textit{radioSignalId} - există câte un identificator pentru fiecare interfață radio a dispozitivului) și intervalul de timp, exprimat în secunde, dintre două notificări fictive consecutive (\textit{eventFrequency}).

\begin{figure}[h]
	\centering
	\includegraphics{dvm_v01_config}
	\caption{Structura fişierului de configurare al DVM. Exemplu pentru un echipament cu două interfețe radio.}
	\label{fig:dvm_v01_config}
\end{figure}
\section{Folosirea în contextul demonstraţiilor de concept a DVM versiunea 1}

Prima versiune a mediatorului cu valori implicite a fost o unealtă foarte importantă în contextul celei de-a doua demonstraţii de concept a proiectului rețelelor de transport de date fără fir din cadrul \gls{onf}. Acesta a ajutat la accelerarea implementării aplicațiilor \gls{sdn} care au fost dezvoltate pentru cazurile de utilizare propuse în \cite{onf2016_poc2}.

\gls{dvm} a oferit interfaţa de Sud \gls{netconf} care expune modelele informaționale dezvoltate de \gls{onf}, în același mod în care ar expune-o un mediator real ce se conectează la echipamente de transport de date fără fir. În acest mod, dezvoltatorii aplicațiilor \gls{sdn} au putut utiliza acest simulator pentru implementarea și testarea acestora, fără a avea nevoie să deţină dispozitive de rețea, care au un preţ foarte ridicat.

Această primă versiune de simulator a accelerat activitățile de pregătire a celei de-a doua demonstraţii de concept, permiţând lucrul în paralel la aplicațiile \gls{sdn} și la dezvoltarea mediatoarelor. Interfaţa \gls{netconf} comună a putut fi testată înainte ca producătorii de echipamente să își implementeze mediatoarele, oferind astfel mai mult timp dezvoltatorilor aplicațiilor \gls{sdn} pentru depanarea programelor. De exemplu, generarea unei notificări \gls{netconf} se poate face mult mai facil cu ajutorul simulatorului. Pentru un mediator real, trebuie ca dispozitivul să fie făcut să genereze o notificare către mediator, prin interfaţa proprietară echipamentului respectiv, apoi mediatorul să traducă acel mesaj într-o notificare \gls{netconf}.

Figura \ref{fig:dvmv01_poc_usage} reprezintă elementele de bază ale demonstraţiei de concept, așa cum sunt prezentate în lucrarea apărută după desfăşurarea acestuia \cite{onf2016_poc2}, în care se poate vedea \gls{dvm}.

\begin{figure}[h]
	\centering
	\includegraphics[width=1\textwidth]{dvmv01_poc_usage}
	\caption{Configurarea rețelei de test SDN utilizând mașini virtuale \cite{onf2016_poc2}.}
	\label{fig:dvmv01_poc_usage}
\end{figure}

Înregistrarea la controlerul \gls{sdn} a primei versiuni a simulatoarelor \gls{dvm}, se face la fel ca pentru un mediator real. Astfel, echipamentul de control oferă o interfață de programare a aplicaţiei - \gls{api} - prin care o aplicație \gls{sdn} poate înregistra un astfel de mediator în controlerul \gls{sdn}. Înregistrarea nu este una automată, utilizatorul fiind nevoit să facă această înregistrare manual. În cea de-a doua demonstraţie de concept, acest lucru a fost făcut prin interfaţa grafică a echipamentului de control folosit (\gls{odl}). După înregistrare, controlerul stabileşte conexiunea \gls{netconf} cu mediatorul.

Codul asociat primei versiuni a \gls{dvm} este oferit cu sursă deschisă și se poate găsi în repertoriul asociat \gls{onf} de pe platforma GitHub, denumit CENTENNIAL \cite{dvmv01github}.
%\chapter{Mediatorul cu valori implicite (DVM) - a doua versiune\label{ch:dvm_v02}}

\section{Arhitectura}

SDN și reţelele actuale..
\section{Implementarea}

Istoria reţelelor definite prin software.
\section{Folosirea în contextul demonstraţiilor de concept a DVM versiunea 1}

Prima versiune a mediatorului cu valori implicite a fost o unealtă foarte importantă în contextul celei de-a doua demonstraţii de concept a proiectului rețelelor de transport de date fără fir din cadrul \gls{onf}. Acesta a ajutat la accelerarea implementării aplicațiilor \gls{sdn} care au fost dezvoltate pentru cazurile de utilizare propuse în \cite{onf2016_poc2}.

\gls{dvm} a oferit interfaţa de Sud \gls{netconf} care expune modelele informaționale dezvoltate de \gls{onf}, în același mod în care ar expune-o un mediator real ce se conectează la echipamente de transport de date fără fir. În acest mod, dezvoltatorii aplicațiilor \gls{sdn} au putut utiliza acest simulator pentru implementarea și testarea acestora, fără a avea nevoie să deţină dispozitive de rețea, care au un preţ foarte ridicat.

Această primă versiune de simulator a accelerat activitățile de pregătire a celei de-a doua demonstraţii de concept, permiţând lucrul în paralel la aplicațiile \gls{sdn} și la dezvoltarea mediatoarelor. Interfaţa \gls{netconf} comună a putut fi testată înainte ca producătorii de echipamente să își implementeze mediatoarele, oferind astfel mai mult timp dezvoltatorilor aplicațiilor \gls{sdn} pentru depanarea programelor. De exemplu, generarea unei notificări \gls{netconf} se poate face mult mai facil cu ajutorul simulatorului. Pentru un mediator real, trebuie ca dispozitivul să fie făcut să genereze o notificare către mediator, prin interfaţa proprietară echipamentului respectiv, apoi mediatorul să traducă acel mesaj într-o notificare \gls{netconf}.

Figura \ref{fig:dvmv01_poc_usage} reprezintă elementele de bază ale demonstraţiei de concept, așa cum sunt prezentate în lucrarea apărută după desfăşurarea acestuia \cite{onf2016_poc2}, în care se poate vedea \gls{dvm}.

\begin{figure}[h]
	\centering
	\includegraphics[width=1\textwidth]{dvmv01_poc_usage}
	\caption{Configurarea rețelei de test SDN utilizând mașini virtuale \cite{onf2016_poc2}.}
	\label{fig:dvmv01_poc_usage}
\end{figure}

Înregistrarea la controlerul \gls{sdn} a primei versiuni a simulatoarelor \gls{dvm}, se face la fel ca pentru un mediator real. Astfel, echipamentul de control oferă o interfață de programare a aplicaţiei - \gls{api} - prin care o aplicație \gls{sdn} poate înregistra un astfel de mediator în controlerul \gls{sdn}. Înregistrarea nu este una automată, utilizatorul fiind nevoit să facă această înregistrare manual. În cea de-a doua demonstraţie de concept, acest lucru a fost făcut prin interfaţa grafică a echipamentului de control folosit (\gls{odl}). După înregistrare, controlerul stabileşte conexiunea \gls{netconf} cu mediatorul.

Codul asociat primei versiuni a \gls{dvm} este oferit cu sursă deschisă și se poate găsi în repertoriul asociat \gls{onf} de pe platforma GitHub, denumit CENTENNIAL \cite{dvmv01github}.
\section{Integrarea cu LINC}

Standardizare: ONF, etc..

\chapter{Simulatorul reţelelor de transport de date fără fir (WTE)\label{ch:wte}}

\graphicspath{ {cap-wte/figures/} }

Simulatorul rețelelor de transport de date fără fir (\textit{Wireless Transport Emulator - WTE}) a apărut ca o evoluţie firească a simulatoarelor prezentate anterior, \gls{dvm}. Acestea din urmă aveau dezavantajul de a simula elemente de rețea independente, prin implementarea unor servere \gls{netconf} care să expună modelele informaționale dezvoltate de \gls{onf}, TR-532 și TR-512. Nu era simulată topologia în întregimea ei, adică să cuprindă și legăturile dintre dispozitivele de rețea. \gls{wte} propune o arhitectură care să implementeze și acest aspect, oferind astfel o soluție completă, prin care se poate genera și trafic de date.

Secţiunile următoare vor prezenta arhitectura \gls{wte}, detalii despre implementare, dar și modalitatea în care acesta poate fi folosit în prezentarea cazurilor de utilizare propuse în demonstraţiile de concept \gls{onf}, chiar după ce acestea s-au încheiat.

\section{Arhitectura}

Simulatorul rețelelor de transport de date fără fir a fost proiectat pentru a simula, pe o singură mașină Linux, o topologie de rețea, folosind diferite unelte. Scopul simulatorului a fost să expună în continuare modelele informaționale de bază și pentru microunde, precum versiunea precedentă, \gls{dvm}. Astfel, există un fişier de configurare în care se specifică topologia ce se vrea a fi simulată, în limbajul \textit{Notaţie de Obiecte JavaScript} - \gls{json}. Formatul acestui fişier de topologie este unul fix și este influenţat de către nivelurile de transport ale obiectelor \gls{ltp} definite în modelul informațional de bază. Mai multe detalii despre acest format vor fi date în secţiunea următoare.

Există mai multe unelte care sunt folosite în arhitectura \gls{wte}, care împreună alcătuiesc simulatorul. Fiecare dispozitiv de rețea este simulat printr-un container \textit{docker}, în care rulează o imagine \textit{Linux} și serverul \gls{netconf} reprezentat de \gls{dvm}. Această abordare a fost aleasă pentru a obţine o izolare la nivelul sistemului de fişiere, astfel încât mai mule instanţe ale \gls{dvm} să poată funcţiona fără probleme pe aceeaşi mașină. Interfețele prezente în fiecare echipament sunt reprezentate prin interfețe de rețea în imaginea Linux, iar legăturile dintre dispozitive se fac cu ajutorul acestora. Fiecare element de rețea are o interfață de administrare, prin care se conectează la echipamentul de control \gls{sdn}. Pentru a obţine o izolare a acestor interfețe (mai exact pentru ca traficul de date care ar putea fi transmis prin interfeţele unui dispozitiv să nu tracă prin interfaţa de administrare), astfel încât echipamentele să nu comunice între ele prin aceste interfețe, au fost folosite \textit{rețelele docker}. Toate aceste elemente sunt ilustrare în Figura \ref{fig:wte_architecture} și vor fi detaliate în continuare.

\begin{figure}[h]
	\centering
	\includegraphics{wte_architecture}
	\caption{Arhitectura WTE.}
	\label{fig:wte_architecture}
\end{figure}

\textit{Docker} este o unealtă care permite crearea de containere software în care pot rula aplicații într-un mod izolat față de aplicațiile sistemului de operare gazdă, care lansează aceste containere \cite{merkel2014docker}. Acestea permit împachetarea unei aplicații într-un container \textit{docker}, împreună cu toate aplicațiile software de care ea depinde. Consumul de resurse al unui astfel de container este mult mai redus decât al unei mașini virtuale, deoarece nu se replică tot sistemul de operare, ci doar bibliotecile și procesele necesare aplicaţiei care este virtualizată. După cum este prezentat și în \cite{chamberlain2014using}, \textit{docker} poate fi utilizat și pentru a produce activitate de cercetare reproductibilă. Astfel, această unealtă este folosită și în cazul \gls{wte} pentru a obţine izolarea, la nivelul sistemului de fişiere al maşinii gazdă, a aplicaţiei ce implementează serverul \gls{netconf} ce expune modelele informaționale dorite: \gls{dvm}.

\textit{Rețelele docker} reprezintă o facilitate oferită de soluţia \textit{docker} prin care se poate izola și stiva de rețea asociată unei imagini \textit{docker}. Astfel, un container poate fi asociat unei \textit{rețele docker} și imaginile ce aparţin unor \textit{rețele docker} diferite sunt izolate și din punctul de vedere al comunicației dintre ele. Un utilizator își poate crea diferite \textit{rețele docker}, având adrese de rețea sau spaţii de adrese la alegere.

Din punctul de vedere al legăturilor ce se pot face între aceste containere, mai exact între interfeţele Linux ce fac parte din imaginile \textit{docker} care reprezintă elementele de rețea, au fost considerate două abordări în cadrul acestei lucrări, care vor fi descrise în secţiunea următoare. În prima abordare, s-a încercat folosirea unui comutator software prin intermediul căruia să se facă conexiunile, \textit{Comutatorul Virtual Deschis} - \gls{ovs}. Cea de-a doua abordare constă în crearea unei conexiuni între aceste interfețe printr-o \textit{pereche Ethernet virtuală (virtual ethernet pair - \textbf{veth})}. O comparaţie între cele două abordări va fi prezentată în secţiunea următoare.

%\begin{figure}[h]
%	\centering
%	\includegraphics[width=1\textwidth]{wte_links}
%	\caption{Legăturile între dispozitivele de rețea simulate în WTE: a) prin OVS; b) prin \textit{veth}.}
%	\label{fig:wte_links}
%\end{figure}

În urma proiectării \gls{wte} a rezultat o abordare simplă: în momentul inițializării, simulatorul analizează fişierul care conţine topologia ce trebuie simulată. Apoi, acesta construieşte \textit{rețelele docker} asociate fiecărui dispozitiv de rețea definit în topologie și porneşte imaginile \textit{docker} necesare, crează interfeţele Linux asociate cu diferitele niveluri de transport ale obiectelor \gls{ltp} definite în topologie, după care construieşte legăturile dintre aceste interfețe.

Componentele care alcătuiesc \gls{wte} sunt prezentate în Figura \ref{fig:wte_components} și sunt următoarele: \gls{dvm}, care a fost adaptat pentru a putea funcţiona în mediul propus de \gls{wte}, fişierul \gls{json} care conţine topologia ce trebuie simulată, fişierul care conţine detalii despre cum ar trebui să fie configurat \gls{wte}, comutatorul software \gls{ovs} (doar în cazul primei abordări propuse pentru reprezentarea legăturilor dintre două elemente de rețea) și un cadru software care să pună toate componentele împreună și să implementeze logica simulatorului. Această ultimă componentă este scrisă în limbajul Python și reprezintă nucleul \gls{wte}.

\begin{figure}[h]
	\centering
	\includegraphics{wte_components}
	\caption{Componentele majore ale WTE.}
	\label{fig:wte_components}
\end{figure}

Nucleul \gls{wte}, care este scris în limbajul Python, este responsabil pentru implementarea infrastructurii de care simulatorul are nevoie și este proiectat să fie modular și flexibil. Este dezvoltat într-o manieră orientată pe obiecte, având clase pentru fiecare componentă importantă de care este nevoie: cadrul general al simulatorului, elementele de rețea, legăturile de date, topologia, etc. Aceasta oferă posibilitatea de extindere folosind, de exemplu, altă soluție care să implementeze un server \gls{netconf}.
\section{Implementarea}

Istoria reţelelor definite prin software.
\section{Folosirea în contextul demonstraţiilor de concept}

\subsection{Altă implementare de server NETCONF}

\chapter{Rezultate și discuţii\label{ch:rezultate_discutii}}

\graphicspath{ {cap-rezultate_discutii/figures/} }

Așa cum a fost prezentat anterior, principala contribuţie a acestei lucrări constă în simulatoarele dezvoltate (de la partea de arhitectură, până la cea de implementare) pentru reprezentarea modelelor informaționale propuse de \gls{onf} în contextul rețelelor de transport de date fără fir. S-a plecat de la ideea de a avea o unealtă care să ofere facilităţile unui server \gls{netconf} dezvoltatorilor de aplicații \gls{sdn}, care s-a materializat în prima versiune a \gls{dvm} și s-a ajuns, în final, la implementarea unei soluții complexe, care să faciliteze simularea unei întregi topologii de rețea de transport de date fără fir, reprezentată de \gls{wte}. Acest capitol va face referire doar la această ultimă versiune de simulator, fiind cea mai completă și complexă.

În cele ce urmează, se va evalua \gls{wte} cu privire la resursele pe care acesta le necesită în simularea diferitelor topologii de rețea, dar și dimensiunea pe care aceste topologii o pot avea, după care se va compara această soluție cu un alt simulator de rețele definite prin software care a fost folosit pe scară largă în activitățile de cercetare din cadrul \gls{sdn} până acum, \textit{mininet}, propus de autorii din~\cite{lantz2010network}.

\section{Evaluarea soluţiilor propuse}

SDN și reţelele actuale..
\section{Comparaţie între WTE și alte abordări}

Simulatorul care constituie baza acestei lucrări, \gls{wte}, oferă o interfață \gls{netconf} care implementează modelele informaționale dezvoltate de \gls{onf}, TR-512 și TR-532, în contextul \gls{sdn} în rețelele de transport de date fără fir. Acestea sunt apărute destul de recent, astfel încât nu există încă unelte care să simuleze asemenea topologii și să expună modelele informaționale echipamentelor de control \gls{sdn}.

Majoritatea simulatoarelor folosite pentru crearea de topologii de rețele definite prin software se bazează pe protocolul OpenFlow și sunt reprezentate de: \textit{mininet}, \textit{EstiNet} și \textit{ns-3}~\cite{lantz2010network,wang2013estinet,wang2014comparison,henderson2008network}. În continuare se va compara simulatorul \gls{wte} cu \textit{mininet}, deoarece acesta este cel mai folosit simulator în contextul \gls{sdn}~\cite{brandonheller2013}.

Din punct de vedere arhitectural, \gls{wte} este foarte asemănător cu \textit{mininet}. Ambele se bazează pe un nucleu Python care se ocupă de toate aspectele simulării: crearea de mașini gazdă, de comutatoare sau de legături între acestea, prin perechi Ethernet virtuale. Dacă în cazul \textit{mininet} se simulează mașini gazdă, sub forma unor procese, care apoi se leagă la comutatoarele din rețea, în cazul \gls{wte} se pune accent pe simularea de dispozitive de rețea și pe funcţionalitatea oferită de acestea echipamentului de control \gls{sdn}, prin intermediul protocolului \gls{netconf}.

Atât \textit{mininet}, cât și \gls{wte} folosesc, după iniţializarea topologiei ce se vrea a fi simulată, o interfață prin linie de comandă care aşteaptă comenzi de la utilizator. Aceasta are cunoştinţe cu privire la dispozitivele de rețea și legăturile dintre acestea și poate trimite comenzi echipamentelor sau poate informa utilizatorul cu privire la detalii despre starea acestora.

Chiar dacă oferă funcționalități diferite, \textit{mininet} oferind OpenFlow, iar \gls{wte} bazându-se pe \gls{netconf}, ambele tipuri de simulatoare folosesc conceptul de virtualizare bazată pe containere Linux~\cite{handigol2012reproducible}. În cazul \textit{mininet}, acestea sunt folosite în mod nativ, prin utilizarea lor directă în Linux. \gls{wte} folosește utilitarul \textit{docker} pentru a crea containere ce reprezintă dispozitive de rețea.

Abordările în cadrul celor două simulatoare, din punctul de vedere al alegerii topologiei care trebuie simulată, sunt diferite. \textit{Mininet} oferă posibilitatea de a porni simulatorul din linia de comandă, specificând prin anumiţi parametri tipul topologiei și dimensiunea acesteia. O altă posibilitate este crearea acesteia prin descrierea lor direct în codul Python, prin interfeţele de programare puse la dispoziţie. Simulatorul \gls{wte} prezintă altă abordare: descrierea topologiei într-un fişier de configurare, după un format prestabilit.

Din punct de vedere funcţional, cele două simulatoare folosesc aceeaşi metodă pentru reprezentarea legăturilor de rețea, perechi Ethernet virtuale, indiferent de punctele între care se doreşte legătura de date: între o mașină gazdă și un comutator, sau între două comutatoare, în cazul \textit{mininet}, sau între două dispozitive de rețea, în cazul \gls{wte}.

În cazul \gls{wte}, fiecare dispozitiv de rețea este reprezentat ca un container \textit{docker}, în interiorul căruia este executat serverul \gls{netconf}, folosit pentru a expune modelele informaționale dezvoltate de \gls{onf}. În cazul \textit{mininet}, pentru reprezentarea comutatoarelor OpenFlow se folosește implicit soluţia software \gls{ovs}, care este executată nativ (fără să fie nevoie de virtualizare) în mediul în care este instalat simulatorul, sau utilizatorul poate alege să folosească alt comutator software capabil să ofere protocolul OpenFlow. Din acest motiv, timpul de iniţializare, dar și resursele consumate de către \gls{wte} sunt mai mari decât cele utilizate de \textit{mininet}.

Din punctul de vedere al traficului care se poate transmite în rețelele simulate, în ambele cazuri se poate injecta trafic cu ajutorul utilitarului \textit{iperf3}.
%\section{Demonstrarea cazurilor de utilizare cu ajutorul WTE}

Standardizare: ONF, etc..
\chapter{Concluzii\label{ch:concluzii}}

Scopul acestei lucrări a fost dezvoltarea și implementarea unui mediu de simulare care să permită crearea de rețele de transport de date fără fir, care să poată fi folosite în contextul \gls{sdn} prin oferirea unei interfețe specifice \gls{netconf} ce expune modelele informaționale dezvoltate de \gls{onf} în acest sens. Simulatorul este destinat dezvoltatorilor de aplicații \gls{sdn} pentru acest tip de rețele, cărora le este eliminată astfel nevoia de a deţine echipamente reale de rețea, care sunt scumpe, fiindu-le permisă testarea aplicațiilor implementate într-un mod facil. Și operatorii de rețele de telecomunicaţii ar putea beneficia de un astfel de mediu de simulare, pentru a testa comportamentul unei aplicații \gls{sdn}, sau chiar interacţiunile dintre mai multe astfel de aplicații într-un mod sigur, fără a afecta rețelele de producție.

În acest sens, a fost prezentat domeniul rețelelor definite prin software, începând de la istoria și evoluția acestora, până la munca de cercetare și de standardizare care se face în acest domeniu. Apoi, a fost prezentată paradigma \gls{sdn} în contextul rețelelor din zilele noastre, precum cele din centre de date sau rețele hibride.

În continuare au fost expuse uneltele \gls{sdn} care sunt folosite în contextul rețelelor de transport de date fără fir. Cele mai importante sunt reprezentate de modelele informaționale dezvoltate de \gls{onf}: modelul informațional de bază (TR-512) și modelul informațional pentru microunde (TR-532). Acestea oferă o interfață comună echipamentelor de control \gls{sdn}, ele putând administra dispozitivele de rețea, indiferent de compania care le produce. A fost prezentat și protocolul \gls{netconf}, care este folosit pentru conexiunea dintre elementele de rețea și cele de control \gls{sdn}, dar și mai multe soluții software care implementează servere utilizate de acest protocol.

Se propun apoi două versiuni de simulator, denumite \gls{dvm}. Acestea reprezintă contribuţii originale ale autorului, de la arhitectură până la dezvoltarea și implementarea lor. Se prezintă utilizarea acestor simulatoare în contextul demonstraţiilor de concept ale grupului \gls{wt} din cadrul \gls{onf}, precum și încercarea de a integra \gls{dvm} cu comutatorul software \gls{linc}.

Autorul propune apoi o nouă contribuţie originală, de la arhitectură până la implementare, reprezentată de o nouă versiune de simulator, \gls{wte}. Acesta este capabil să simuleze topologii întregi de rețele de transport de date fără fir, nu doar un singur element, ca versiunea anterioară, \gls{dvm}. Este explicată arhitectura simulatorului, apoi sunt date detalii legate de implementare și se prezintă utilizarea acestuia în cea de-a patra demonstraţie de concept \gls{onf}.

În continuare se descrie procesul de evaluare a \gls{wte}, în raport cu anumite caracteristici pe care le are: timpul de iniţializare a simulatorului, spaţiul pe care acesta îl ocupă pe disc, puterea de procesare de care are nevoie și memoria cu acces aleator folosită. Se prezintă măsurătorile acestor caracteristici, care se fac simulând diferite tipuri de topologii (inel, arbore sau plasă), având diferite dimensiuni. Aceste măsurători sunt executate pe trei sisteme diferite, unde mediul de simulare este instalat: pe o mașină locală, într-un mediu de tip \textit{cloud} care face parte din laboratorul Orbit, pus la dispoziţie de AT\&T și într-un alt mediu de tip \textit{cloud}, ce a fost folosit în cea de-a patra demonstraţie de concept \gls{onf}, pus la dispoziţie de Deutsche Telekom. Aceste rezultate relevă faptul că pot fi simulate topologii ce conţin sute sau chiar mii de interfețe de rețea, depinzând de capabilitățile sistemului care este folosit. Apoi este prezentată o comparaţie sumară între \gls{wte} și un alt tip de simulator folosit în contextul \gls{sdn}, \textit{mininet}.

\section{Rezultate obţinute}

Istoria reţelelor definite prin software.
\section{Contribuţii originale}

SDN și reţelele actuale..
\section{Lista contribuţiilor originale}

Standardizare: ONF, etc..
\section{Perspective de dezvoltare ulterioară}

Simulatorul propus în această lucrare, \gls{wte}, s-a dovedit a fi o unealtă importantă în activitatea de standardizare desfăşurată în cadrul \gls{onf}, în contextul \gls{sdn} în rețelele de transport de date fără fir, permiţând dezvoltatorilor de aplicații testarea acestora fără nevoia de a deţine dispozitive de rețea reale. Deoarece această activitate nu este încă finalizată, \gls{wte} poate fi îmbunătăţit pentru a oferi mai multe facilități utilizatorilor. Chiar și după încheierea procesului de standardizare a \gls{sdn}, simulatorul poate fi folosit de către operatori de rețele de telecomunicaţii pentru a testa diferite aplicații, sau pentru a studia interacţiunile dintre acestea, înainte de a le instala în rețele de producție.

O direcţie interesantă de cercetare ulterioară o constituie implementarea unei interfețe grafice pentru utilizator. Acest lucru ar prezenta un avantaj major, pentru că ar simplifica experienţa de utilizare. În momentul de față, specificarea topologiei ce se doreşte a fi simulată se face prin modificarea fişierului \gls{json} folosit de \gls{wte} în momentul inițializării. Acest lucru presupune o înţelegere prealabilă a modelului informațional de bază, astfel că poate părea dificil pentru un utilizator neexperimentat. O interfață grafică în care să se poată descrie topologia ar însemna că simulatorul ar putea fi folosit mai facil și s-ar putea adresa mai multor utilizatori.

O altă direcţie interesantă de cercetare o constituie implementarea unui mecanism de colectare și stocare a valorilor de monitorizare a performanţei pentru interfeţele fiecărui dispozitiv de rețea. Deoarece acestea sunt reprezentate în containerele \textit{docker} ca fiind interfețe Linux, acestea oferă deja valori pentru indicatori de performanţă, precum numărul de pachete transmise sau recepţionate de interfaţa respectivă. În implementarea curentă, \gls{wte} oferă valori implicite pentru atributele de monitorizare a performanţei. Acest lucru ar putea fi schimbat și simulatorul ar putea oferi aceste valori prin mecanismul de colectare a indicatorilor de performanţă.

Altă perspectivă interesantă o constituie analizarea diferitelor optimizări ce pot fi implementate în simulator. Acestea ar putea viza atât îmbunătăţirea timpului de iniţializare, dar, mai important, ar putea viza scăderea procentului de memorie cu acces aleator folosit de fiecare dispozitiv sau interfață reprezentate. Astfel, dimensiunile topologiilor ce vor putea fi simulate ar putea creşte.

Fiind o unealtă unică în momentul de față, deoarece este singura care oferă printr-o interfață de tip \gls{netconf} modelele informaționale nou-apărute în cadrul \gls{onf}, \gls{wte} permite numeroase alte perspective de dezvoltare, prin oferirea acestui mediu de simulare. Cu ajutorul lui se pot face analize asupra eficienţei unor aplicații \gls{sdn} sau se pot testa interacţiuni dintre acestea.



\appendix % all chapters following will be labeled as appendices
%\section{Implementarea}

Istoria reţelelor definite prin software.
%\include{ch-appendicies/printing}


% Make the bibliography single spaced
\singlespacing
%\bibliographystyle{plain}
\bibliographystyle{unsrt}

% add the Bibliography to the Table of Contents
\cleardoublepage
\ifdefined\phantomsection
  \phantomsection  % makes hyperref recognize this section properly for pdf link
\else
\fi
\addcontentsline{toc}{chapter}{Bibliografie}

% include your .bib file
\bibliography{thesis}

\end{document}

