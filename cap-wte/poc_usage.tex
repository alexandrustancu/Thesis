\section{Folosirea în contextul demonstraţiilor de concept}

Primele două versiuni ale \gls{dvm} au fost unelte foarte importante pentru cea de-a doua și cea de-a treia demonstraţie de concept desfăşurate în cadrul \gls{onf}. În același mod, \gls{wte} a fost o unealtă critică în contextul celei de-a patra demonstraţii de concept a proiectului \gls{wt} din grupul \gls{otwg}, care a dus la accelerarea și o bună desfăşurare a acesteia. A permis dezvoltatorilor de aplicații \gls{sdn} testarea unor implementări mai avansate, ce nu puteau fi acoperite de simulatoarele precedente, prin oferirea unei modalităţi facile de a simula diferite topologii de rețea, prin simpla descriere a acestora într-un fişier.

Pe lângă expunerea modelelor informaționale TR-532 și TR-512.1, \gls{wte} permite și modelarea legăturilor fără fir dintre dispozitivele de rețea simulate. Astfel, se pot testa și aplicații care influenţează dirijarea traficului în rețea. O versiune ulterioară a \gls{wte} va permite chiar simularea unor evenimente care influenţează traficul de date (de exemplu atenuări introduse pe o legătură fără fir de vreme nefavorabilă).

Un alt avantaj față de versiunile anterioare, care a fost folosit în cea de-a patra demonstraţie de concept, a constat în expunerea a încă două modele informaționale, care încă nu au fost lansate oficial de către \gls{onf}: un model informațional pentru Ethernet și un model informațional pentru sincronizare, bazat pe \gls{ptp}.

Modelul informațional pentru Ethernet a fost unul simplificat, având aceeaşi structură ca elemente ce se găsesc în modelul informațional pentru microunde: capabilităţi, configuraţie, stare, problemele curente, valori de performanţă curente și valori de performanţă istorice, legătura cu modelul de bază făcându-se printr-un obiect \gls{lp}. Singurul parametru configurabil în acest model simplificat a fost identificatorul rețelei locale virtuale (\gls{vlan} ID). Acest model reprezintă pachetul condiţional asociat obiectelor \gls{ltp} de pe nivelul de transport \gls{eth}. Identificatorul rețelei locale virtuale a fost reprezentat ca parametru al interfeţelor virtuale de tip \textit{vlan}, în interiorul containerelor \textit{docker}.

Modelul pentru sincronizare a fost o extindere a modelului dezvoltat de \gls{itu-t} pentru protocolul \gls{ptp}. Extinderea a constat în adăugarea unor parametri, dar și integrarea cu modelul informațional de bază, printr-un obiect \gls{lp}. Parametrii asociaţi acestui model nu au avut echivalent în implementarea interfeţelor în sistemul de operare Linux al containerelor \textit{docker}, au avut doar valori implicite ce au fost expuse echipamentului de control \gls{sdn}.

\subsection{Altă implementare de server NETCONF}

Arhitectura modulară și abordarea flexibilă pe care se bazează \gls{wte} facilitează integrarea acestuia cu alte implementări de servere \gls{netconf}, în funcție de nevoile fiecărui utilizator.

În contextul celei de-a patra demonstraţii de concept \gls{onf}, o companie membră a proiectului \gls{wt}, \textit{highstreet technologies} a ales să integreze propriul server \gls{netconf}, o implementare Java bazată tot pe fişiere de configurare \gls{xml}, dar cu mult mai puţine facilități decât \gls{dvm}. Au făcut această alegere deoarece implementarea lor era potrivită doar pentru teste simple și rapide, oferind o soluție de server \gls{netconf} mai uşor de folosit decât \gls{dvm}.

Chiar dacă au renunţat la facilităţile oferite de \gls{wte} în combinație cu \gls{dvm} (precum simularea legăturilor fără fir), soluţia lor a demonstrat utilitatea infrastructurii folosite de simulator și faptul ca aceasta poate fi extinsă într-un mod facil, pentru a îndeplini și alte nevoi ale utilizatorilor. Practic, faptul că \gls{wte} a fost folosit de unul dintre membrii proiectului și pentru altceva în afară de utilizarea obişnuită (adică doar rularea simulatorului, cu scopul testării unor aplicații \gls{sdn}), mai exact pentru extinderea acestuia prin integrarea cu altă soluție de server \gls{netconf}, a validat încă o dată, soluţia \gls{wte}.

Chiar dacă cea de-a patra demonstraţie de concept \gls{onf} s-a terminat, \gls{wte} oferă în continuare posibilitatea executării cazurilor de utilizare propuse și chiar dezvoltarea de noi aplicații care să folosească modelele informaționale propuse de \gls{onf} (de bază, pentru microunde, pentru Ethernet și pentru sincronizare). În acest scop, mediul de simulare pus la dispoziţie de \gls{wte} a fost instalat și încă rulează, până la sfârşitul anului 2017, într-un mediu \textit{Cloud} asigurat de gazda celei de-a patra demonstraţii de concept, \textit{Deutsche Telekom}. Utilizatorilor interesaţi li se poate oferi acces la acel mediu de simulare și la echipamentul de control \gls{sdn}, pentru a putea vizualiza cazurile de utilizare propuse, sau pentru a dezvolta altele.