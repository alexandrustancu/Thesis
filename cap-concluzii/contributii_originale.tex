\section{Contribuţii originale}

Activitatea de cercetare doctorală a avut ca rezultat numeroase contribuţii originale, menţionate în cele ce urmează:
\begin{enumerate}
	\item Studiul sintetic privind domeniul \gls{sdn}, punând accent pe activitățile de cercetare din cadrul rețelelor de transport de date fără fir [\ref{sec:papers}, \ref{item:overview_sdn}];
	
	\item Studiul sintetic despre uneltele folosite în \gls{sdn}, în contextul rețelelor de date fără fir: modelele informaționale dezvoltate de \gls{onf} și soluții software de implementare a unui server \gls{netconf}, echipamente de control \gls{sdn} [\ref{sec:papers}, \ref{item:comparison_sdn}];
	
	\item Evaluarea teoretică și practică, a cadrelor software care oferă un server \gls{netconf} și alegerea celei mai potrivite soluții pentru simulatorul propus în această lucrare [\ref{sec:papers}, \ref{item:comparison_netconf}];
	
	\item Definirea arhitecturii unei primi versiuni de simulator cu valori implicite, \gls{dvm} [\ref{sec:papers}, \ref{item:dvm_v01}];
	
	\item Dezvoltarea și implementarea primei versiuni a simulatorului \gls{dvm} [\ref{sec:papers}, \ref{item:dvm_v01}];
	
	\item Validarea soluţiei implementate (\gls{dvm}, versiunea 1) în cea de-a doua demonstraţie de concept \gls{onf} [\ref{sec:papers}, \ref{item:poc_2}];
	
	\item Definirea arhitecturii celei de-a doua versiuni a simulatorului \gls{dvm} [\ref{sec:papers}, \ref{item:dvm_v02}];
	
	\item Dezvoltarea și implementarea celei de-a doua versiuni a simulatorului cu valori implicite, \gls{dvm} [\ref{sec:papers}, \ref{item:dvm_v02}];
	
	\item Validarea soluţiei implementate (\gls{dvm}, versiunea 2) în cea de-a treia demonstraţie de concept \gls{onf} [\ref{sec:papers}, \ref{item:poc_3}];
	
	\item Integrarea simulatorului \gls{dvm}, versiunea 2, cu comutatorul software \gls{linc} [\ref{sec:papers}, \ref{item:wte_linc}];
	
	\item Definirea arhitecturii simulatorului rețelelor de transport de date fără fir, \gls{wte} [\ref{sec:papers}, \ref{item:wte}];
	
	\item Dezvoltarea și implementarea simulatorului \gls{wte} [\ref{sec:papers}, \ref{item:wte}];
	
	\item Validarea soluţiei propuse, \gls{wte}, în cea de-a patra demonstraţie de concept \gls{onf} [\ref{sec:papers}, \ref{item:poc_4}];
	
	\item Evaluarea simulatorului \gls{wte} cu privire la resursele de care are nevoie pentru execuţie, pentru a putea estima dimensiunea topologiilor ce pot fi simulate cu ajutorul lui.
\end{enumerate}