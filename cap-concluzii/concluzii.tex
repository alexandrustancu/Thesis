\chapter{Concluzii\label{ch:concluzii}}

Scopul acestei lucrări a fost dezvoltarea și implementarea unui mediu de simulare care să permită crearea de rețele de transport de date fără fir, care să poată fi folosite în contextul \gls{sdn} prin oferirea unei interfețe specifice \gls{netconf} ce expune modelele informaționale dezvoltate de \gls{onf} în acest sens. Simulatorul este destinat dezvoltatorilor de aplicații \gls{sdn} pentru acest tip de rețele, cărora le este eliminată astfel nevoia de a deţine echipamente reale de rețea, care sunt scumpe, fiindu-le permisă testarea aplicațiilor implementate într-un mod facil. Și operatorii de rețele de telecomunicaţii ar putea beneficia de un astfel de mediu de simulare, pentru a testa comportamentul unei aplicații \gls{sdn}, sau chiar interacţiunile dintre mai multe astfel de aplicații într-un mod sigur, fără a afecta rețelele de producție.

În acest sens, a fost prezentat domeniul rețelelor definite prin software, începând de la istoria și evoluția acestora, până la munca de cercetare și de standardizare care se face în acest domeniu. Apoi, a fost prezentată paradigma \gls{sdn} în contextul rețelelor din zilele noastre, precum cele din centre de date sau rețele hibride.

În continuare au fost expuse uneltele \gls{sdn} care sunt folosite în contextul rețelelor de transport de date fără fir. Cele mai importante sunt reprezentate de modelele informaționale dezvoltate de \gls{onf}: modelul informațional de bază (TR-512) și modelul informațional pentru microunde (TR-532). Acestea oferă o interfață comună echipamentelor de control \gls{sdn}, ele putând administra dispozitivele de rețea, indiferent de compania care le produce. A fost prezentat și protocolul \gls{netconf}, care este folosit pentru conexiunea dintre elementele de rețea și cele de control \gls{sdn}, dar și mai multe soluții software care implementează servere utilizate de acest protocol.

Se propun apoi două versiuni de simulator, denumite \gls{dvm}. Acestea reprezintă contribuţii originale ale autorului, de la arhitectură până la dezvoltarea și implementarea lor. Se prezintă utilizarea acestor simulatoare în contextul demonstraţiilor de concept ale grupului \gls{wt} din cadrul \gls{onf}, precum și încercarea de a integra \gls{dvm} cu comutatorul software \gls{linc}.

Autorul propune apoi o nouă contribuţie originală, de la arhitectură până la implementare, reprezentată de o nouă versiune de simulator, \gls{wte}. Acesta este capabil să simuleze topologii întregi de rețele de transport de date fără fir, nu doar un singur element, ca versiunea anterioară, \gls{dvm}. Este explicată arhitectura simulatorului, apoi sunt date detalii legate de implementare și se prezintă utilizarea acestuia în cea de-a patra demonstraţie de concept \gls{onf}.

În continuare se descrie procesul de evaluare a \gls{wte}, în raport cu anumite caracteristici pe care le are: timpul de iniţializare a simulatorului, spaţiul pe care acesta îl ocupă pe disc, puterea de procesare de care are nevoie și memoria cu acces aleator folosită. Se prezintă măsurătorile acestor caracteristici, care se fac simulând diferite tipuri de topologii (inel, arbore sau plasă), având diferite dimensiuni. Aceste măsurători sunt executate pe trei sisteme diferite, unde mediul de simulare este instalat: pe o mașină locală, într-un mediu de tip \textit{cloud} care face parte din laboratorul Orbit, pus la dispoziţie de AT\&T și într-un alt mediu de tip \textit{cloud}, ce a fost folosit în cea de-a patra demonstraţie de concept \gls{onf}, pus la dispoziţie de Deutsche Telekom. Aceste rezultate relevă faptul că pot fi simulate topologii ce conţin sute sau chiar mii de interfețe de rețea, depinzând de capabilitățile sistemului care este folosit. Apoi este prezentată o comparaţie sumară între \gls{wte} și un alt tip de simulator folosit în contextul \gls{sdn}, \textit{mininet}.

\section{Rezultate obţinute}

Capitolul \ref{ch:introducere_sdn}, în care autorul a efectuat un studiu sintetic despre \gls{sdn}, a avut ca rezultat o mai bună înţelegere a contextului, reprezentat de domeniul rețelelor definite prin software și a activităţilor de cercetare și de standardizare care se desfăşoară în industrie și academie legat de acest subiect. 

Următorul capitol, în care s-au studiat uneltele \gls{sdn} care se folosesc în contextul rețelelor de transport de date fără fir, a avut ca rezultat înţelegerea modelelor informaționale dezvoltate de \gls{onf}, care au influenţat simulatoarele propuse ulterior, dar și găsirea unor soluții software care pot fi folosite pentru dezvoltarea acestor simulatoare.

Activitatea prezentată în capitolul \ref{ch:dvm_v01} a avut ca rezultat concret dezvoltarea și implementarea de către autor a două versiuni ale simulatorului \gls{dvm}, ce au fost folosite în cea de-a doua, respectiv cea de-a treia demonstraţie de concept \gls{onf}, validând astfel utilitatea acestora în procesul de standardizare și în contextul dezvoltării de aplicații \gls{sdn}.

Următorul capitol a avut ca activitate principală dezvoltarea unui simulator îmbunătăţit, \gls{wte}, care se bazează pe versiunea anterioară, \gls{dvm}. Rezultatele obţinute astfel au constat în definirea arhitecturii noului simulator și dezvoltarea și implementarea acestuia. La fel ca versiunile anterioare, \gls{wte} a fost folosit în cea de-a patra demonstraţie de concept organizată de grupul \gls{wt} din cadrul \gls{onf}, fiind astfel validată utilitatea soluţiei propuse de către autor.

În ultimul capitol, \ref{ch:rezultate_discutii}, autorul realizează o evaluare a simulatorului dezvoltat, folosind diverse topologii și sisteme în care \gls{wte} a fost instalat. Rezultatele acestei activităţi sunt date de către măsurătorile diferitor caracteristici ce au fost considerate, precum timpul de iniţializare a simulatorului, spaţiul ocupat de acesta pe disc, sau puterea de procesare și memoria cu acces aleator utilizate de mediul de simulare în momentul execuţiei. Aceste măsurători au fost apoi folosite pentru a trage concluzii asupra dimensiunii topologiilor ce pot fi simulate cu ajutorul \gls{wte} și pentru a compara această propunere cu alte soluții existente în acest context.
\section{Contribuţii originale}

Activitatea de cercetare doctorală a avut ca rezultat numeroase contribuţii originale, menţionate în cele ce urmează:
\begin{enumerate}
	\item Studiul sintetic privind domeniul \gls{sdn}, punând accent pe activitățile de cercetare din cadrul rețelelor de transport de date fără fir [\ref{sec:papers}, \ref{item:overview_sdn}];
	
	\item Studiul sintetic despre uneltele folosite în \gls{sdn}, în contextul rețelelor de date fără fir: modelele informaționale dezvoltate de \gls{onf} și soluții software de implementare a unui server \gls{netconf}, echipamente de control \gls{sdn} [\ref{sec:papers}, \ref{item:comparison_sdn}];
	
	\item Evaluarea teoretică și practică, a cadrelor software care oferă un server \gls{netconf} și alegerea celei mai potrivite soluții pentru simulatorul propus în această lucrare [\ref{sec:papers}, \ref{item:comparison_netconf}];
	
	\item Definirea arhitecturii unei primi versiuni de simulator cu valori implicite, \gls{dvm} [\ref{sec:papers}, \ref{item:dvm_v01}];
	
	\item Dezvoltarea și implementarea primei versiuni a simulatorului \gls{dvm} [\ref{sec:papers}, \ref{item:dvm_v01}];
	
	\item Validarea soluţiei implementate (\gls{dvm}, versiunea 1) în cea de-a doua demonstraţie de concept \gls{onf} [\ref{sec:papers}, \ref{item:poc_2}];
	
	\item Definirea arhitecturii celei de-a doua versiuni a simulatorului \gls{dvm} [\ref{sec:papers}, \ref{item:dvm_v02}];
	
	\item Dezvoltarea și implementarea celei de-a doua versiuni a simulatorului cu valori implicite, \gls{dvm} [\ref{sec:papers}, \ref{item:dvm_v02}];
	
	\item Validarea soluţiei implementate (\gls{dvm}, versiunea 2) în cea de-a treia demonstraţie de concept \gls{onf} [\ref{sec:papers}, \ref{item:poc_3}];
	
	\item Integrarea simulatorului \gls{dvm}, versiunea 2, cu comutatorul software \gls{linc} [\ref{sec:papers}, \ref{item:wte_linc}];
	
	\item Definirea arhitecturii simulatorului rețelelor de transport de date fără fir, \gls{wte} [\ref{sec:papers}, \ref{item:wte}];
	
	\item Dezvoltarea și implementarea simulatorului \gls{wte} [\ref{sec:papers}, \ref{item:wte}];
	
	\item Optimizarea timpului de inițializare a \gls{wte} [\ref{sec:papers}, \ref{item:wte_init_optimization}];
	
	\item Validarea soluţiei propuse, \gls{wte}, în cea de-a patra demonstraţie de concept \gls{onf} [\ref{sec:papers}, \ref{item:poc_4}];
	
	\item Evaluarea simulatorului \gls{wte} cu privire la resursele de care are nevoie pentru execuţie, pentru a putea estima dimensiunea topologiilor ce pot fi simulate cu ajutorul lui [\ref{sec:papers}, \ref{item:wte_evaluation}].
\end{enumerate}
\section{Lista lucrărilor originale \label{sec:papers}}

Din activitățile asociate muncii de cercetare doctorală au rezultat mai multe lucrări ştiinţifice, care au fost publicate în diferite locuri. Autorul a trimis spre publicare \textbf{19 lucrări ştiinţifice} ce au avut subiecte legate de domeniul tezei de doctorat, dintre care \textbf{8 ca prim autor}. Un articol a fost publicat într-o revistă (cotată \textbf{ISI}), 13 articole fac parte din volumele unor conferinţe internaţionale, iar 5 dintre ele sunt cărţi albe (\textit{white papers}) publicate de către \gls{onf}. Autorul a contribuit și la modelul informațional pentru microunde (TR-532) dezvoltat de către \gls{onf}, care are rolul de recomandare tehnică.

Lista de lucrări este următoarea:
\begin{enumerate}
%	\item \textbf{A. Stancu}, ``Analiza facilităţilor oferite de diferite tipuri de echipamente de control in cadrul rețelelor definite prin programe soft'', Referat de doctorat nr. 2, aprilie 2015
%	\item \textbf{A. Stancu}, ``Analiza facilităţilor oferite de diferite tipuri de echipamente de control in cadrul rețelelor definite prin programe soft'', Referat de doctorat nr. 2, aprilie 2015
	\item \textbf{Stancu, A}.; Vulpe, A.; Fratu, O.; Halunga, S., ``Wireless Transport Emulator Based on LINC Software Switch,'' Wireless Personal Communications, 2017, DOI: 10.1007/s11277-017-4654-9 \textbf{(ISI, IF 2017: 0,951)}\label{item:wte_linc};
	
	\item \textbf{Stancu, A}.; Halunga, S.; Suciu, G.; Vulpe, A., ``An Overview Study of Software Defined Networking,'' \textit{Informatics in Economy (IE 2015), 2015 14th International Conference on,} București, Aprilie 30-Mai 3, 2015, ISSN: 2247 – 1480, Accession Number: WOS:000362796900009 \textbf{(ISI)}\label{item:overview_sdn};
	
	\item Suciu, George; Vulpe, Alexandru; Arseni, Stefan Ciprian; \textit{Stancu, Alexandru}; Butca, Cristina; Suciu, Victor, ``Monitoring a cloud-based speech processing system,'' in \textit{Electronics, Computers and Artificial Intelligence (ECAI), 2015 7th International Conference on}, pp.Y-23-Y-27, București, Romania, 25-27 Iunie 2015, doi: 10.1109/ECAI.2015.7301172 \textbf{(ISI, IEEEXplore)};
	
	\item Suciu, G.; Sticlan, A.M.; Butca, C.; Vulpe, A.; \textit{Stancu, A}.; Halunga, S., ``Cloud Search Based Applications for Big Data - Challenges and Methodologies for Acceleration,'' A\textit{CM Symposium on Principles of Distributed Computing (PODC 2015) - Workshop on Adaptive Resource Management and Scheduling for Cloud Computing (ARMS-CC 2015)}, Donostia-San Sebastián, Spania, Iulie 20, 2015;
	
	\item \textbf{Stancu, A}.; Halunga, S.; Vulpe, A.; Suciu, G.; Fratu, O.; Popovici, E.C., ``A Comparison between several Software Defined Networking Controllers,'' \textit{12th International Conference on Advanced Technologies, Systems and Services in Telecommunications (TELSIKS 2015)}, Niš, Serbia, Octombrie 14-17, 2015, pp. 223-226, ISBN:978-1-4673-7516-0, DOI: 10.1109/TELSKS.2015.7357774, Accession Number: WOS:000380406700043 \textbf{(IEEEXplore)}\label{item:comparison_sdn};
	
	\item G. Suciu, C. Butca, V. Suciu, A. Geaba, \textit{A. Stancu} and S. Arseni, ``Basic Internet Foundation and Cloud Computing,'' \textit{2015 10th International Conference on P2P, Parallel, Grid, Cloud and Internet Computing (3PGCIC)}, Krakow, 2015, pp. 278-284, doi: 10.1109/3PGCIC.2015.146 \textbf{(IEEEXplore)};
	
	\item Suciu, G.; Vulpe A.; \textit{Stancu A}.; Arseni, S.; Butcă, C.; Suciu, V.; Necula, L., ``Dedicated Search Engines for Multimedia Big Data Inedixing: EXALEAD Solutions,'' \textit{Informatics in Economy (IE 2016), 2016 15th International Conference on}, Cluj-Napoca, Iunie 2-5, 2016, ISSN: 2247 – 1480 \textbf{(ISI)};
	
	\item \textbf{Stancu, A}.; Vulpe, A.; Suciu, G.; Popovici, E., ``Comparison between several NETCONF Server open source implementations,'' \textit{11th International Conference on Communications (COMM 2016)}, Bucharest, Romania, Iunie 9-11, 2016, pp. 185-188, ISBN: 978-1-4673-8196-3, DOI: 10.1109/ICComm.2016.7528212, Accession Number: WOS:000383221900036 \textbf{(ISI, IEEEXplore)}\label{item:comparison_netconf};
	
	\item \textbf{Stancu, A}.; Arseni, S.; Vulpe, A.; Fratu, O.; Halunga, S., ``Intrusion Prevention System Evaluation for SDN-enabled IoT Networks,'' \textit{2nd EAI International Conference on Future access enablers of ubiquitous and intelligent infrastructures (FABULOUS 2016)}, Belgrade, Serbia, Octombrie 24–26, 2016 \textbf{(ISI)} \label{item:ips_iot};
	
	\item \textbf{Stancu, A}.; Vulpe, A.; Fratu, O.; Halunga, S., ``Default Values Mediator Used for a Wireless Transport SDN Proof of Concept,'' \textit{2016 IEEE Conference on Standards for Communications and Networking (CSCN'16)}, Berlin, Germany, Octombrie 30 - Noiembrie 2, 2016, ISBN: 978-1-5090-3861-9, DOI: 10.1109/CSCN.2016.7784889, Accession Number: WOS:000391392900007 \textbf{(ISI, IEEEXplore)}\label{item:dvm_v01};
	
	\item Arseni, S.; \textit{Stancu, A}.; Vulpe, A.; Fratu, O.; Halunga, S., ``Primary Evaluation of a Software-defined Security Architecture for an IoT Environment,'' \textit{Global Wireless Summit (GWS) 2016}, Aarhus, Denmark, Noiembrie 27-30, 2016 \textbf{ (ISI, IEEEXplore)};
	
	\item \textbf{Stancu, A}.; Avram, A.; Skorupski, M.; Vulpe, A.; Halunga, S., ``Enabling SDN Application Development Using a NETCONF Mediator Layer Simulator,'' \textit{9th International Conference on Ubiquitous and Future Networks (ICUFN 2017)}, Milano, Italy, Iulie 4-7, 2017, DOI: 10.1109/ICUFN.2017.7993873 \textbf{(IEEEXplore)}\label{item:dvm_v02};
	
	\item \textbf{Stancu, A}.; Vulpe, A.; Halunga, S.; Fratu, O., ``Architecture of a Wireless Transport Network Emulator for SDN Applications Development,'' \textit{3rd EAI International Conference on Future access enablers of ubiquitous and intelligent infrastructures (FABULOUS 2017)}, Bucharest, Romania, Octombrie 12–14, 2017, \textit{acceptat}\label{item:wte};
	
	\item Alexandru Vulpe, Ștefan Arseni, \textit{Alexandru Stancu}, Octavian Fratu, ``A Hybrid Testbed for Secure Internet-of-Things,'' \textit{3rd EAI International Conference on Future access enablers of ubiquitous and intelligent infrastructures (FABULOUS 2017)}, Bucharest, Romania, Octombrie 12–14, 2017, \textit{acceptat};
	
	\item ``Wireless Transport SDN Proof of Concept 2 Detailed Report'', Open Networking Foundation, Iunie, 2016 \label{item:poc_2};
	
	\item ``Microwave Information Model, TR-532'', Open Networking Foundation, Decembrie 2016;
	
	\item ``Third Wireless Transport SDN Proof of Concept White Paper'', Open Networking Foundation, Decembrie 2016;
	
	\item ``Third Wireless Transport SDN Proof of Concept Detailed Report'', Open Networking Foundation, Decembrie 2016 \label{item:poc_3};
	
	\item ``Fourth Wireless Transport SDN Proof of Concept White Paper'', Open Networking Foundation, Iunie 2017;
	
	\item ``Fourth Wireless Transport SDN Proof of Concept Detailed Report'', Open Networking Foundation, Iunie 2017 \label{item:poc_4}.
	
\end{enumerate}
\section{Perspective de dezvoltare ulterioară}

Simulatorul propus în această lucrare, \gls{wte}, s-a dovedit a fi o unealtă importantă în activitatea de standardizare desfăşurată în cadrul \gls{onf}, în contextul \gls{sdn} în rețelele de transport de date fără fir, permiţând dezvoltatorilor de aplicații testarea acestora fără nevoia de a deţine dispozitive reale de rețea. Deoarece această activitate nu este încă finalizată, \gls{wte} poate fi îmbunătăţit pentru a oferi mai multe facilități utilizatorilor. Chiar și după încheierea procesului de standardizare a \gls{sdn}, simulatorul poate fi folosit de către operatori de rețele de telecomunicaţii pentru a testa diferite aplicații, sau pentru a studia interacţiunile dintre acestea, înainte de a le instala în rețele de producție.

O direcţie interesantă de cercetare ulterioară o constituie implementarea unei interfețe grafice pentru utilizator. Acest lucru ar prezenta un avantaj major, pentru că ar simplifica experienţa de utilizare. În momentul de față, specificarea topologiei ce se doreşte a fi simulată se face prin modificarea fişierului \gls{json} folosit de \gls{wte} în momentul inițializării. Acest lucru presupune o înţelegere prealabilă a modelului informațional de bază, astfel că poate părea dificil pentru un utilizator neexperimentat. O interfață grafică în care să se poată descrie topologia ar însemna că simulatorul ar putea fi folosit mai facil și s-ar putea adresa mai multor utilizatori.

O altă direcţie interesantă de cercetare o constituie implementarea unui mecanism de colectare și stocare a valorilor de monitorizare a performanţei pentru interfeţele fiecărui dispozitiv de rețea. Deoarece acestea sunt reprezentate în containerele \textit{docker} ca fiind interfețe Linux, acestea oferă deja valori pentru indicatori de performanţă, precum numărul de pachete transmise sau recepţionate de interfaţa respectivă. În implementarea curentă, \gls{wte} oferă valori implicite pentru atributele de monitorizare a performanţei. Acest lucru ar putea fi schimbat și simulatorul ar putea oferi aceste valori prin mecanismul de colectare a indicatorilor de performanţă.

Altă perspectivă interesantă o constituie analizarea diferitelor optimizări ce pot fi implementate în simulator. Acestea ar putea viza atât îmbunătăţirea timpului de iniţializare, dar, mai important, ar putea viza scăderea procentului de memorie cu acces aleator folosit de fiecare dispozitiv sau interfață reprezentate. Astfel, dimensiunile topologiilor ce vor putea fi simulate ar putea creşte.

Fiind o unealtă unică în momentul de față, deoarece este singura care oferă printr-o interfață de tip \gls{netconf} modelele informaționale nou-apărute în cadrul \gls{onf}, \gls{wte} permite numeroase alte perspective de dezvoltare, prin oferirea acestui mediu de simulare. Cu ajutorul lui se pot face analize asupra eficienţei unor aplicații \gls{sdn} sau se pot testa interacţiuni dintre acestea. Simulatorul poate fi transformat chiar într-un produs comercial, dat fiind interesul manifestat de către operatori în acest sens.
