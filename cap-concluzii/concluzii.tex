\chapter{Concluzii\label{ch:concluzii}}

Scopul acestei lucrări a fost dezvoltarea și implementarea unui mediu de simulare care să permită crearea de rețele de transport de date fără fir, care să poată fi folosite în contextul \gls{sdn} prin oferirea unei interfețe specifice \gls{netconf} ce expune modelele informaționale dezvoltate de \gls{onf} în acest sens. Simulatorul este destinat dezvoltatorilor de aplicații \gls{sdn} pentru acest tip de rețele, cărora le este eliminată astfel nevoia de a deţine echipamente reale de rețea, care sunt scumpe, fiindu-le permisă testarea aplicațiilor implementate într-un mod facil. Și operatorii de rețele de telecomunicaţii ar putea beneficia de un astfel de mediu de simulare, pentru a testa comportamentul unei aplicații \gls{sdn}, sau chiar interacţiunile dintre mai multe astfel de aplicații într-un mod sigur, fără a afecta rețelele de producție.

În acest sens, a fost prezentat domeniul rețelelor definite prin software, începând de la istoria și evoluția acestora, până la munca de cercetare și de standardizare care se face în acest domeniu. Apoi, a fost prezentată paradigma \gls{sdn} în contextul rețelelor din zilele noastre, precum cele din centre de date sau rețele hibride.

În continuare au fost expuse uneltele \gls{sdn} care sunt folosite în contextul rețelelor de transport de date fără fir. Cele mai importante sunt reprezentate de modelele informaționale dezvoltate de \gls{onf}: modelul informațional de bază (TR-512) și modelul informațional pentru microunde (TR-532). Acestea oferă o interfață comună echipamentelor de control \gls{sdn}, ele putând administra dispozitivele de rețea, indiferent de compania care le produce. A fost prezentat și protocolul \gls{netconf}, care este folosit pentru conexiunea dintre elementele de rețea și cele de control \gls{sdn}, dar și mai multe soluții software care implementează servere utilizate de acest protocol.

Se propun apoi două versiuni de simulator, denumite \gls{dvm}. Acestea reprezintă contribuţii originale ale autorului, de la arhitectură până la dezvoltarea și implementarea lor. Se prezintă utilizarea acestor simulatoare în contextul demonstraţiilor de concept ale grupului \gls{wt} din cadrul \gls{onf}, precum și încercarea de a integra \gls{dvm} cu comutatorul software \gls{linc}.

Autorul propune apoi o nouă contribuţie originală, de la arhitectură până la implementare, reprezentată de o nouă versiune de simulator, \gls{wte}. Acesta este capabil să simuleze topologii întregi de rețele de transport de date fără fir, nu doar un singur element, ca versiunea anterioară, \gls{dvm}. Este explicată arhitectura simulatorului, apoi sunt date detalii legate de implementare și se prezintă utilizarea acestuia în cea de-a patra demonstraţie de concept \gls{onf}.

În continuare se descrie procesul de evaluare a \gls{wte}, în raport cu anumite caracteristici pe care le are: timpul de iniţializare a simulatorului, spaţiul pe care acesta îl ocupă pe disc, puterea de procesare de care are nevoie și memoria cu acces aleator folosită. Se prezintă măsurătorile acestor caracteristici, care se fac simulând diferite tipuri de topologii (inel, arbore sau plasă), având diferite dimensiuni. Aceste măsurători sunt executate pe trei sisteme diferite, unde mediul de simulare este instalat: pe o mașină locală, într-un mediu de tip \textit{cloud} care face parte din laboratorul Orbit, pus la dispoziţie de AT\&T și într-un alt mediu de tip \textit{cloud}, ce a fost folosit în cea de-a patra demonstraţie de concept \gls{onf}, pus la dispoziţie de Deutsche Telekom. Aceste rezultate relevă faptul că pot fi simulate topologii ce conţin sute sau chiar mii de interfețe de rețea, depinzând de capabilitățile sistemului care este folosit. Apoi este prezentată o comparaţie sumară între \gls{wte} și un alt tip de simulator folosit în contextul \gls{sdn}, \textit{mininet}.

\section{Rezultate obţinute}

Istoria reţelelor definite prin software.
\section{Contribuţii originale}

SDN și reţelele actuale..
\section{Lista contribuţiilor originale}

Standardizare: ONF, etc..
\section{Perspective de dezvoltare ulterioară}

Simulatorul propus în această lucrare, \gls{wte}, s-a dovedit a fi o unealtă importantă în activitatea de standardizare desfăşurată în cadrul \gls{onf}, în contextul \gls{sdn} în rețelele de transport de date fără fir, permiţând dezvoltatorilor de aplicații testarea acestora fără nevoia de a deţine dispozitive de rețea reale. Deoarece această activitate nu este încă finalizată, \gls{wte} poate fi îmbunătăţit pentru a oferi mai multe facilități utilizatorilor. Chiar și după încheierea procesului de standardizare a \gls{sdn}, simulatorul poate fi folosit de către operatori de rețele de telecomunicaţii pentru a testa diferite aplicații, sau pentru a studia interacţiunile dintre acestea, înainte de a le instala în rețele de producție.

O direcţie interesantă de cercetare ulterioară o constituie implementarea unei interfețe grafice pentru utilizator. Acest lucru ar prezenta un avantaj major, pentru că ar simplifica experienţa de utilizare. În momentul de față, specificarea topologiei ce se doreşte a fi simulată se face prin modificarea fişierului \gls{json} folosit de \gls{wte} în momentul inițializării. Acest lucru presupune o înţelegere prealabilă a modelului informațional de bază, astfel că poate părea dificil pentru un utilizator neexperimentat. O interfață grafică în care să se poată descrie topologia ar însemna că simulatorul ar putea fi folosit mai facil și s-ar putea adresa mai multor utilizatori.

O altă direcţie interesantă de cercetare o constituie implementarea unui mecanism de colectare și stocare a valorilor de monitorizare a performanţei pentru interfeţele fiecărui dispozitiv de rețea. Deoarece acestea sunt reprezentate în containerele \textit{docker} ca fiind interfețe Linux, acestea oferă deja valori pentru indicatori de performanţă, precum numărul de pachete transmise sau recepţionate de interfaţa respectivă. În implementarea curentă, \gls{wte} oferă valori implicite pentru atributele de monitorizare a performanţei. Acest lucru ar putea fi schimbat și simulatorul ar putea oferi aceste valori prin mecanismul de colectare a indicatorilor de performanţă.

Altă perspectivă interesantă o constituie analizarea diferitelor optimizări ce pot fi implementate în simulator. Acestea ar putea viza atât îmbunătăţirea timpului de iniţializare, dar, mai important, ar putea viza scăderea procentului de memorie cu acces aleator folosit de fiecare dispozitiv sau interfață reprezentate. Astfel, dimensiunile topologiilor ce vor putea fi simulate ar putea creşte.

Fiind o unealtă unică în momentul de față, deoarece este singura care oferă printr-o interfață de tip \gls{netconf} modelele informaționale nou-apărute în cadrul \gls{onf}, \gls{wte} permite numeroase alte perspective de dezvoltare, prin oferirea acestui mediu de simulare. Cu ajutorul lui se pot face analize asupra eficienţei unor aplicații \gls{sdn} sau se pot testa interacţiuni dintre acestea.
