\section{Rezultate obţinute}

Capitolul \ref{ch:introducere_sdn}, în care autorul a efectuat un studiu sintetic despre \gls{sdn}, a avut ca rezultat o mai bună înţelegere a contextului, reprezentat de domeniul rețelelor definite prin software și a activităţilor de cercetare și de standardizare care se desfăşoară în industrie și academie legate de acest subiect. 

Următorul capitol, în care s-au studiat uneltele \gls{sdn} care se folosesc în contextul rețelelor de transport de date fără fir, a avut ca rezultat înţelegerea modelelor informaționale dezvoltate de \gls{onf}, care au influenţat simulatoarele propuse ulterior, dar și găsirea unor soluții software care pot fi folosite pentru dezvoltarea acestor simulatoare.

Activitatea prezentată în capitolul \ref{ch:dvm_v01} a avut ca rezultat concret dezvoltarea și implementarea de către autor a două versiuni ale simulatorului \gls{dvm}, ce au fost folosite în cea de-a doua, respectiv cea de-a treia demonstraţie de concept \gls{onf}, validând astfel utilitatea acestora în procesul de standardizare și în contextul dezvoltării de aplicații \gls{sdn}.

Următorul capitol a avut ca activitate principală dezvoltarea unui simulator îmbunătăţit, \gls{wte}, care se bazează pe versiunea anterioară, \gls{dvm}. Rezultatele obţinute astfel au fost definirea arhitecturii noului simulator și dezvoltarea și implementarea acestuia. La fel ca versiunile anterioare, \gls{wte} a fost folosit în cea de-a patra demonstraţie de concept organizată de grupul \gls{wt} din cadrul \gls{onf}, fiind astfel validată utilitatea soluţiei propuse de către autor.

În ultimul capitol, \ref{ch:rezultate_discutii}, autorul realizează o evaluare a simulatorului dezvoltat, folosind diverse topologii și sisteme în care \gls{wte} a fost instalat. Rezultatele acestei activităţi sunt date de către măsurătorile diferitor caracteristici ce au fost considerate, precum timpul de iniţializare a simulatorului, spaţiul ocupat de acesta pe disc, sau puterea de procesare și memoria cu acces aleator utilizate de mediul de simulare în momentul execuţiei. Aceste măsurători au fost apoi folosite pentru a trage concluzii asupra dimensiunii topologiilor ce pot fi simulate cu ajutorul \gls{wte} și pentru a compara această propunere cu alte soluții existente în acest context.