\section{Perspective de dezvoltare ulterioară}

Simulatorul propus în această lucrare, \gls{wte}, s-a dovedit a fi o unealtă importantă în activitatea de standardizare desfăşurată în cadrul \gls{onf}, în contextul \gls{sdn} în rețelele de transport de date fără fir, permiţând dezvoltatorilor de aplicații testarea acestora fără nevoia de a deţine dispozitive de rețea reale. Deoarece această activitate nu este încă finalizată, \gls{wte} poate fi îmbunătăţit pentru a oferi mai multe facilități utilizatorilor. Chiar și după încheierea procesului de standardizare a \gls{sdn}, simulatorul poate fi folosit de către operatori de rețele de telecomunicaţii pentru a testa diferite aplicații, sau pentru a studia interacţiunile dintre acestea, înainte de a le instala în rețele de producție.

O direcţie interesantă de cercetare ulterioară o constituie implementarea unei interfețe grafice pentru utilizator. Acest lucru ar prezenta un avantaj major, pentru că ar simplifica experienţa de utilizare. În momentul de față, specificarea topologiei ce se doreşte a fi simulată se face prin modificarea fişierului \gls{json} folosit de \gls{wte} în momentul inițializării. Acest lucru presupune o înţelegere prealabilă a modelului informațional de bază, astfel că poate părea dificil pentru un utilizator neexperimentat. O interfață grafică în care să se poată descrie topologia ar însemna că simulatorul ar putea fi folosit mai facil și s-ar putea adresa mai multor utilizatori.

O altă direcţie interesantă de cercetare o constituie implementarea unui mecanism de colectare și stocare a valorilor de monitorizare a performanţei pentru interfeţele fiecărui dispozitiv de rețea. Deoarece acestea sunt reprezentate în containerele \textit{docker} ca fiind interfețe Linux, acestea oferă deja valori pentru indicatori de performanţă, precum numărul de pachete transmise sau recepţionate de interfaţa respectivă. În implementarea curentă, \gls{wte} oferă valori implicite pentru atributele de monitorizare a performanţei. Acest lucru ar putea fi schimbat și simulatorul ar putea oferi aceste valori prin mecanismul de colectare a indicatorilor de performanţă.

Altă perspectivă interesantă o constituie analizarea diferitelor optimizări ce pot fi implementate în simulator. Acestea ar putea viza atât îmbunătăţirea timpului de iniţializare, dar, mai important, ar putea viza scăderea procentului de memorie cu acces aleator folosit de fiecare dispozitiv sau interfață reprezentate. Astfel, dimensiunile topologiilor ce vor putea fi simulate ar putea creşte.

Fiind o unealtă unică în momentul de față, deoarece este singura care oferă printr-o interfață de tip \gls{netconf} modelele informaționale nou-apărute în cadrul \gls{onf}, \gls{wte} permite numeroase alte perspective de dezvoltare, prin oferirea acestui mediu de simulare. Cu ajutorul lui se pot face analize asupra eficienţei unor aplicații \gls{sdn} sau se pot testa interacţiuni dintre acestea.