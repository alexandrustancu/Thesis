\chapter{Introducere \^{\i}n reţelele definite prin software\label{ch:introducere_sdn}}

\graphicspath{ {cap-introducere_in_sdn/figures/} }

În zilele noastre, reţelele de comunicaţii au devenit complexe și greu de administrat și configurat. De asemenea, numărul dispozitivelor mobile a crescut considerabil, alături de conţinutul pe care acestea îl accesează. Aceste lucruri au dus la evidenţierea limitărilor pe care reţelele tradiţionale le presupun. Chiar dacă nu toate ideile ce stau la baza ei sunt noi, datorită unui context favorabil, acestea, împreună cu alte noi idei, au dus la apariţia paradigmei \gls{sdn} în industria reţelisticii.

Această nouă tehnologie nu a ajuns încă la maturitate și la adoptarea pe scară largă, în toate aspectele reţelelor, însă eforturile considerabile care se fac în activităţile de standardizare și crearea de ecosisteme \gls{sdn} vor duce la această adoptare. După cum este evidenţiat și în~\cite{nadeau2013sdn}, se tinde către crearea unor reţele care pot fi programate prin software, crescând astfel flexibilitatea și agilitatea lor.

\section{Istoria SDN}

Istoria reţelelor definite prin software.
\section{Standardizarea SDN}

Activitățile de cercetare și standardizare din jurul \gls{sdn} au loc pe două planuri importante. Pe de o parte, există organizaţiile care dezvoltă standarde - \gls{sdo}, care sunt formate din reprezentanţi ai industriei, ai academiei, sau alte entități și au ca scop dezvoltarea de specificaţii sau recomandări tehnice, în contextul \gls{sdn} care să fie folosite de toată industria rețelisticii. Autorii din~\cite{schneider2014standardizations} amintesc astfel de organizații: \gls{onf}, \gls{ietf}, \gls{etsi}, \gls{itu-t}, \gls{ieee}. 

Pe de altă parte, există asociații sau comunități de oameni, în general care fac parte din industrie, ale căror rezultate ale cercetării candidează pentru a deveni standarde. Aceste rezultate sunt, de obicei, implementări cu sursă deschisă ce vor fi folosite ulterior în industrie. Exemple de astfel de comunități sunt prezente în~\cite{halpern2014standards, meyer2013software}: OpenDaylight (activităţi ce se desfăşoară sub patronajul fundaţiei Linux), \gls{mef}, \gls{bbf}, \gls{oif}.

Activitățile de standardizare ale \gls{sdn} sunt foarte importante, asigurându-i acestei tehnologii o evoluţie stabilă și aducând-o la o maturitate care îi va permite adoptarea pe scară largă în industria rețelisticii, mitigând astfel dezavantajele rețelelor tradiționale.

Sunt mai multe planuri pe care se lucrează pentru standardizarea \gls{sdn}. Unul dintre aceste planuri este cel al interfeţelor de tip \textit{southbound}, care fac legătura între echipamentele de dirijare și echipamentele de control \gls{sdn}. Protocolul OpenFlow este un exemplu în acest sens, însă nu este singurul protocol capabil să facă legătura între planurile de date și de control. În ultimul timp se pune foarte mare accent pe \gls{netconf} ca o alternativă pentru a configura echipamentele care dirijează traficul, după cum se poate observa în~\cite{csoma2015escape, felix2014multi, zhou2014research}. Astfel apare nevoia de a dezvolta modele informaţionale care să abstractizeze echipamentele din planul de date și care să poată fi folosite de \gls{netconf} pentru a configura dispozitivele. Și planul interfeţelor de tip \textit{northbound} are un rol important în standardizare, deoarece poate oferi un punct de plecare comun pentru dezvoltatorii de aplicații \gls{sdn}. De exemplu, în cadrul \gls{onf} există un grup care se ocupă cu activităţi de standardizare în această direcţie. Un alt plan este reprezentat de implementările software, cu sursă deschisă (\textit{open-source}), care se crează în acest context. Așa cum evidenţiază și autorii din~\cite{lin2014software, rothenberg2014open}, aceste implementări sunt foarte importante prin ecosistemele care apar ca urmare a activităţii \textit{comunităţilor open-source}.

O importanţă foarte mare în cadrul acestor activităţi o au și demonstraţiile de concept. Acestea au capacitatea de a demonstra avantajele pe care \gls{sdn} le aduce nu doar la un nivel teoretic, ci într-un mod practic, propunând diferite cazuri reale de utilizare a acestei tehnologii și aplicând-o, într-un mod restrâns, pe topologii formate din echipamente reale. Scopul acestora este, pe de o parte, de a adăuga un plus de valoare activităţilor de standardizare și de a testa și proba rezultatele acestor activităţi. Pe de altă parte, aceste demonstraţii de concept au scopul de a atrage atenţia și altor entități din industrie și a duce la înlesnirea adoptării acestei tehnologii pe scară largă.

\subsection{ONF}

\gls{onf} este o organizaţie non-profit formată din peste două sute de membri care fac parte din industrie, academie, sau institute de cercetare, ce are ca obiectiv accelerarea adoptării \gls{sdn} pe scară largă in industria rețelisticii prin dezvoltarea de standarde deschise și de ecosisteme software cu sursă deschisă. \gls{onf} a apărut ca urmare a activităţii de cercetare din jurul protocolului OpenFlow din cadrul Universităţii Stanford. Dintre membrii cei mai importanți amintim operatori de rețele, precum AT\&T, Google, Facebook, Verizon, Deutsche Telekom sau Telefonica, producători de echipamente, cum ar fi Cisco, Ericsson, Huawei, Intel sau NEC și reprezentanţi ai unor universităţi cunoscute, ca Stanford sau Princeton.

Activitățile din cadrul \gls{onf} sunt împărţite în mai multe zone de interes:
\begin{itemize}
	\item \textit{Operatori}. Această zonă se ocupă de mai multe aspecte, precum \gls{sdn} în contextul sistemelor \textit{Carrier Grade}, în centre de date, în întreprinderi sau aspecte legate de migrarea serviciilor dintr-o rețea tradiţională într-o rețea definită prin software.
	\item \textit{Servicii}. Se ocupă de proiecte care permit existența aplicațiilor și serviciilor de rețea care au la bază tehnologia \gls{sdn}. Astfel, există mai multe grupuri de lucru care analizează: arhitectura \gls{sdn} și cum aceste principii se aplică în imaginea de ansamblu a unei rețele de comunicaţii; un model informaţional \textit{de bază}, care să reprezinte piatra de temelie cu ajutorul căreia să se dezvolte alte modele informaţionale, specializate pentru anumite aplicații sau tehnologii; interfeţele de tip \textit{northbound}, pentru a facilita dezvoltarea aplicațiilor \gls{sdn}; probleme de securitate pe care această nouă tehnologie le poate întâmpina.  
	\item \textit{Specificaţii}. Zonă care are ca responsabilitate publicarea tuturor specificaţiilor și recomandărilor tehnice create de \gls{onf}. Acestea includ protocoalele OpenFlow, OF-Config dar și alte interfețe standard care se dezvoltă pentru tehnologii de transport de diferite tipuri (optic, fără fir). Există și un proiect care se ocupă de testare și interoperabilitate, scopul acestuia fiind accelerarea adoptării protocolului OpenFlow prin certificări și promovarea interoperabilităţii între diferiţi producători de echipamente.
	\item \textit{Piaţă}. Această zonă se concentrează pe educarea comunităţii \gls{sdn} în legătură cu valoarea pe care standardele \gls{onf} le oferă rețelelor definite prin software și promovarea adoptării unor astfel de rețele definite prin software cu sursă deschisă. Aceste obiective sunt îndeplinite prin publicaţii, evenimente care se organizează sau demonstraţii care să arate comunităţii cazuri reale de utilizare.
\end{itemize}

Există trei tipuri de publicaţii care reies din activitățile \gls{onf}: (i) \textit{specificaţii}, care includ toate standardele care definesc un protocol, modelul informaţional, funcționalități și documente despre cadrele asociate; publicaţiile normative astfel rezultate se numesc Specificaţii Tehnice - \gls{ts} și sunt supuse unor procese de licenţiere; (ii) \textit{recomandări}, care includ consideraţii arhitecturale, cazuri de utilizare, analiza cerințelor, terminologie; aceste documente referă documente normative \gls{onf}, dar nu necesită licenţiere și pot fi utilizate în mod liber, având rolul de Recomandări Tehnice - \gls{tr}; (iii) \textit{publicaţii}, reprezentând documente care conţin informații ce ajută în procesele de lansare a \gls{sdn} în producție, rezumate ale soluţiilor, studii de caz sau cărţi albe (\textit{white papers}); aceste documente nu au caracter normativ și pot fi folosite în mod liber.

În continuare vor fi enumerate câteva astfel de documente produse de către \gls{onf}: \textit{OpenFlow Switch Specification Ver. 1.5.1} (TS-025), care descrie cerinţele unui comutator logic ce suportă protocolul OpenFlow; \textit{OpenFlow Management	and	Configuration Protocol 1.2 (OF-Config 1.2)} (TS-016), document ce descrie motivația, cerințele, scopul și specificaţiile protocolului OF-Config; \textit{Conformance Test Specification for OpenFlow Switch Specification V1.3.4} (TS-026), definind cerinţele și procedurile de test care determină conformitatea unui comutator cu specificațiile protocolului OpenFlow 1.3.4; \textit{Core Information Model (CoreModel) 1.2} (TR-512), care prezintă modelul informațional de bază, pe care celelalte modele informaționale dezvoltate în \gls{onf} se bazează; \textit{Microwave Information Model} (TR-532), reprezentând modelul informațional ce descrie echipamentele de transport de date fără fir. Aceste ultime două documente vor fi detaliate în următorul capitol, fiind baza dezvoltării și implementării simulatoarelor propuse în această lucrare.

Un rol important în adoptarea \gls{sdn} de către operatori și în diseminarea rezultatelor din cadrul \gls{onf} îl au demonstraţiile de concept. Acestea adună laolaltă operatori, producători de echipamente și integratori de servicii, cu scopul de a demonstra recomandările produse de activitatea de cercetare. În contextul rețelelor de transport de date fără fir, proiectul \gls{wt}, care face parte din grupul \gls{otwg}, a terminat cu succes patru astfel de demonstraţii de concept~\cite{onf2015_poc1, onf2016_poc2, onf2016_poc3}.

\subsection{IETF}

\subsection{ETSI}

\subsection{ITU-T}

\subsection{MEF}
\section{SDN în contextul reţelelor actuale}

