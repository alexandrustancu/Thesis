\section{Standardizarea SDN}

Activitățile de cercetare și standardizare din jurul \gls{sdn} au loc pe două planuri importante. Pe de o parte, există organizaţiile care dezvoltă standarde - \gls{sdo}, care sunt formate din reprezentanţi ai industriei, ai academiei, sau alte entități și au ca scop dezvoltarea de specificaţii sau recomandări tehnice, în contextul \gls{sdn} care să fie folosite de toată industria rețelisticii. Autorii din~\cite{schneider2014standardizations} amintesc astfel de organizații: \gls{onf}, \gls{ietf}, \gls{etsi}, \gls{itu-t}, \gls{ieee}. 

Pe de altă parte, există asociații sau comunități de oameni, în general care fac parte din industrie, ale căror rezultate ale cercetării candidează pentru a deveni standarde. Aceste rezultate sunt, de obicei, implementări cu sursă deschisă ce vor fi folosite ulterior în industrie. Exemple de astfel de comunități sunt prezente în~\cite{halpern2014standards, meyer2013software}: OpenDaylight (activităţi ce se desfăşoară sub patronajul fundaţiei Linux), \gls{mef}, \gls{bbf}, \gls{oif}.

Activitățile de standardizare ale \gls{sdn} sunt foarte importante, asigurându-i acestei tehnologii o evoluţie stabilă și aducând-o la o maturitate care îi va permite adoptarea pe scară largă în industria rețelisticii, mitigând astfel dezavantajele rețelelor tradiționale.

Sunt mai multe planuri pe care se lucrează pentru standardizarea \gls{sdn}. Unul dintre aceste planuri este cel al interfeţelor de Sud, care fac legătura între echipamentele de dirijare și echipamentele de control \gls{sdn}. Protocolul OpenFlow este un exemplu în acest sens, însă nu este singurul protocol capabil să facă legătura între planurile de date și de control. În ultimul timp se pune foarte mare accent pe \gls{netconf} ca o alternativă pentru a configura echipamentele care dirijează traficul, după cum se poate observa în~\cite{csoma2015escape, felix2014multi, zhou2014research}. Astfel apare nevoia de a dezvolta modele informaţionale care să abstractizeze echipamentele din planul de date și care să poată fi folosite de \gls{netconf} pentru a configura dispozitivele. Și planul interfeţelor de Nord are un rol important în standardizare, deoarece poate oferi un punct de plecare comun pentru dezvoltatorii de aplicații \gls{sdn}. De exemplu, în cadrul \gls{onf} există un grup care se ocupă cu activităţi de standardizare în această direcţie. Un alt plan este reprezentat de implementările software, cu sursă deschisă (\textit{open-source}), care se crează în acest context. Așa cum evidenţiază și autorii din~\cite{lin2014software, rothenberg2014open}, aceste implementări sunt foarte importante prin ecosistemele care apar ca urmare a activităţii comunităţilor care dezvoltă software cu sursă deschisă.

O importanţă foarte mare în cadrul acestor activităţi o au și demonstraţiile de concept. Acestea au capacitatea de a demonstra avantajele pe care \gls{sdn} le aduce nu doar la un nivel teoretic, ci într-un mod practic, propunând diferite cazuri reale de utilizare a acestei tehnologii și aplicând-o, într-un mod restrâns, unor topologii formate din echipamente reale. Scopul acestora este, pe de o parte, de a adăuga un plus de valoare activităţilor de standardizare și de a testa și proba rezultatele acestor activităţi. Pe de altă parte, aceste demonstraţii de concept au scopul de a atrage atenţia și altor entități din industrie și a duce la înlesnirea adoptării acestei tehnologii pe scară largă.

\subsection{Open Networking Foundation}

\gls{onf} este o organizaţie non-profit formată din peste două sute de membri care fac parte din industrie, academie, sau institute de cercetare, ce are ca obiectiv accelerarea adoptării \gls{sdn} pe scară largă in industria rețelisticii prin dezvoltarea de standarde deschise și de ecosisteme software cu sursă deschisă. \gls{onf} a apărut ca urmare a activităţii de cercetare din jurul protocolului OpenFlow din cadrul Universităţii Stanford. Dintre membrii cei mai importanţi amintim operatori de rețele, precum AT\&T, Google, Facebook, Verizon, Deutsche Telekom sau Telefonica, producători de echipamente, cum ar fi Cisco, Ericsson, Huawei, Intel sau NEC și reprezentanţi ai unor universităţi cunoscute, ca Stanford sau Princeton.

Activitățile din cadrul \gls{onf} sunt împărţite în mai multe zone de interes, conform \cite{onftech}:
\begin{itemize}
	\item \textit{Operatori}. Această zonă se ocupă de mai multe aspecte, precum \gls{sdn} în contextul sistemelor de tip Purtător - \textit{Carrier Grade} - (rețelele de telecomunicaţii foarte sigure, cu o disponibilitate de peste 99,999\% și recuperare în caz de defectare de sub 50 de milisecunde), în centre de date, în întreprinderi sau aspecte legate de migrarea serviciilor dintr-o rețea tradiţională într-o rețea definită prin software.
	\item \textit{Servicii}. Se ocupă de proiecte care permit existența aplicațiilor și serviciilor de rețea care au la bază tehnologia \gls{sdn}. Astfel, există mai multe grupuri de lucru care analizează arhitectura \gls{sdn} și modul în care aceste principii se aplică în imaginea de ansamblu a unei rețele de comunicaţii, un model informaţional \textit{de bază}, care să reprezinte piatra de temelie cu ajutorul căreia să se dezvolte alte modele informaţionale, specializate pentru anumite aplicații sau tehnologii, interfeţele de Nord, pentru a facilita dezvoltarea aplicațiilor \gls{sdn}, probleme de securitate pe care această nouă tehnologie le poate întâmpina.  
	\item \textit{Specificaţii}. Zonă care are ca responsabilitate publicarea tuturor specificaţiilor și recomandărilor tehnice create de \gls{onf}. Acestea includ protocoalele OpenFlow, OF-Config dar și alte interfețe standard care se dezvoltă pentru tehnologii de transport de diferite tipuri (optic, fără fir). Există și un proiect care se ocupă de testare și interoperabilitate, scopul acestuia fiind accelerarea adoptării protocolului OpenFlow prin certificări și promovarea interoperabilităţii între diferiţi producători de echipamente \cite{onftr539}.
	\item \textit{Piaţă}. Această zonă se concentrează pe educarea comunităţii \gls{sdn} în legătură cu valoarea pe care standardele \gls{onf} le oferă rețelelor definite prin software și promovarea adoptării unor astfel de rețele definite prin software cu sursă deschisă. Aceste obiective sunt îndeplinite prin publicaţii, evenimente care se organizează sau demonstraţii care să arate comunităţii cazuri reale de utilizare.
\end{itemize}

Există trei tipuri de publicaţii care reies din activitățile \gls{onf}: (i) \textit{specificaţii}, care includ toate standardele care definesc un protocol, modelul informaţional, funcționalități și documente despre cadrele asociate; publicaţiile normative astfel rezultate se numesc Specificaţii Tehnice - \gls{ts} și sunt supuse unor procese de licenţiere; (ii) \textit{recomandări}, care includ consideraţii arhitecturale, cazuri de utilizare, analiza cerințelor, terminologie; aceste documente referă documente normative \gls{onf}, dar nu necesită licenţiere și pot fi utilizate în mod liber, având rolul de Recomandări Tehnice - \gls{tr}; (iii) \textit{publicaţii}, reprezentând documente care conţin informații ce ajută în procesele de lansare a \gls{sdn} în producție, rezumate ale soluţiilor, studii de caz sau cărţi albe (\textit{white papers}); aceste documente nu au caracter normativ și pot fi folosite în mod liber.

În continuare vor fi enumerate câteva astfel de documente produse de către \gls{onf}: \textit{OpenFlow Switch Specification Ver. 1.5.1} (TS-025), care descrie cerinţele unui comutator logic ce suportă protocolul OpenFlow; \textit{OpenFlow Management and	Configuration Protocol 1.2 (OF-Config 1.2)} (TS-016), document ce descrie motivația, cerințele, scopul și specificaţiile protocolului OF-Config; \textit{Conformance Test Specification for OpenFlow Switch Specification V1.3.4} (TS-026), definind cerinţele și procedurile de test care determină conformitatea unui comutator cu specificațiile protocolului OpenFlow 1.3.4; \textit{Core Information Model (CoreModel) 1.2} (TR-512), care prezintă modelul informațional de bază, pe care celelalte modele informaționale dezvoltate în \gls{onf} se bazează; \textit{Microwave Information Model} (TR-532), reprezentând modelul informațional ce descrie echipamentele de transport de date fără fir. Aceste ultime două documente vor fi detaliate în următorul capitol, fiind baza dezvoltării și implementării simulatoarelor propuse în această lucrare.

Un rol important în adoptarea \gls{sdn} de către operatori și în diseminarea rezultatelor din cadrul \gls{onf} îl au demonstraţiile de concept. Acestea adună laolaltă operatori, producători de echipamente și integratori de servicii, cu scopul de a demonstra recomandările produse de activitatea de cercetare. În contextul rețelelor de transport de date fără fir, proiectul \gls{wt}, care face parte din grupul \gls{otwg}, a terminat cu succes patru astfel de demonstraţii de concept~\cite{onf2015_poc1, onf2016_poc2, onf2016_poc3, onf2017_poc4}.

Prima astfel de demonstraţie a avut loc în octombrie 2015, în Madrid și a fost organizată de Telefonica, împreună cu Universitatea Carlos III. Scopul acesteia a fost de a extinde protocolul OpenFlow cu atribute specifice echipamentelor de transport de date fără fir. Astfel, interfaţa de Sud folosită a fost OpenFlow, în timp ce echipamentul de control \gls{sdn} a fost \gls{onos}. Au fost prezentate două cazuri de utilizare: (i) pornirea/oprirea unei interfeţe radio în funcţie de nivelul de trafic ce trebuie transmis de către echipament, economisind astfel energie în momentele în care nivelul de trafic în reţea nu este foarte ridicat; (ii) schimbarea căilor de date într-un ruter, în cazul în care se pierd pachete pe legătura radio, ca urmare a unor schimbări meteorologice (simulate prin folosirea unui atenuator variabil pe legătura radio).

A doua demonstraţie de concept a avut loc în aprilie 2016, la München și a fost organizată de Telefonica. Spre deosebire de prima demonstraţie, în cea de-a doua s-a folosit drept interfaţă de Sud protocolul \gls{netconf}, iar echipamentul de control \gls{sdn} a fost \gls{odl}. Ca model informaţional YANG pentru configurarea echipamentelor s-a ales un model pentru microunde simplificat (conţinea un număr limitat de atribute). Acesta a fost dezvoltat de către grup și apoi implementat de către echipamentele de la diferiţi producători. Au fost demonstrate mai multe cazuri de utilizare folosind aplicaţii \gls{sdn} care se bazează pe acel model: detectarea și configurarea de noi echipamente, detectarea și corecţia (printr-o acţiune a operatorului) diferenţelor între configuraţia curentă și cea planificata, detecţia și vizualizarea reţelei de transport configurate, primirea, afişarea și stocarea evenimentelor și alarmelor din reţea.

Cea de-a treia demonstraţie de concept s-a desfăşurat în octombrie 2016 și a avut loc în centrul de cercetare WINLAB de la Universitatea Rutgers, fiind organizat de AT\&T. S-au ales aceeaşi interfaţă de Sud și acelaşi echipament de control \gls{sdn}, scopul acestei demonstraţii fiind de a proba utilitatea întregului model informaţional pentru microunde (conţinea toate atributele pe care le poate avea un echipament de transport de date fără fir). S-a implementat și modelul de bază (\textit{Core Model}), dezvoltat de alt grup din cadrul \gls{onf}. Au fost demonstrate aplicaţiile dezvoltate pentru cea de-a doua demonstraţie, dar și două noi aplicaţii: una care administrează spectrul, prin compararea și configurarea frecvenţelor planificate și cele configurate pe echipamente, realocându-le în caz că nu se potrivesc; cealaltă, denumită automatizarea în bucla închisă, demonstra un răspuns simplist la anumiţi factori declanşatori (interni, externi sau temporali).

A patra demonstraţie de concept a avut loc în iunie 2017, la Bonn și a fost organizată de Deutsche Telekom. S-au folosit mai multe modele informaţionale: modelul pentru microunde, modelul de bază, un model Ethernet simplificat și un model pentru sincronizare - pentru \gls{ptp}, plecând de la un model dezvoltat de ITU-T. Astfel, s-au putut demonstra mai multe cazuri de utilizare: administrarea echipamentelor care transportă date fără fir, administrarea indicatorilor de performanță ai echipamentelor, administrarea puterii echipamentelor, administrarea echipamentelor capabile Ethernet, recalcularea căilor de trafic Ethernet într-o rețea sau administrarea echipamentelor care suportă \gls{ptp}. 

\subsection{IETF}

Un alt \gls{sdo} care prestează activităţi de standardizare în domeniul rețelelor definite prin software este \gls{ietf}. Această organizaţie, în general, oferă standardele de bază folosite în Internet. Obiectivul lor declarat este îmbunătăţirea funcţionării Internetului prin dezvoltarea de documente tehnice relevante și de înaltă calitate care să ghideze proiectarea, utilizarea și administrarea Internetului.

Doar puține din grupurile de lucru din cadrul \gls{ietf} se ocupă cu activităţi care au legătură cu tehnologia \gls{sdn}. Unul dintre acestea a dezvoltat \gls{forces}, iniţiativă ce separa planurile de date și de control ale echipamentelor \cite{doria2010forwarding}, cu câțiva ani înainte ca protocolul OpenFlow, mult mai cunoscut, să propună asta într-un mod care să permită o adoptare mai rapidă în industrie.

Alt grup de lucru se ocupă de \textit{interfaţa către sistemul de rutare} - \gls{i2rs}. Acesta își propune să dezvolte interfețe pentru ca aplicațiile \gls{sdn} să poată accesa sistemul de rutare al rețelei. Se doreşte ca beneficiarii \gls{i2rs} să fie aplicații de administrare, echipamente de control \gls{sdn} sau aplicații de utilizator care au cereri specifice la adresa rețelei. Acestea ar permite informaţiilor, politicilor și parametrilor operaţionali să fie introduşi sau extraşi din sistemul de rutare, fără a afecta consistenţa datelor și coerenţa infrastructurii de rutare. Această abordare poate fi privită ca un rival al protocolului OpenFlow, cu menţiunea că se aplică doar de la nivelul 3 în sus, în stiva OSI.

În interiorul grupului de lucru \gls{spring} se discută un control al rutării care să permită indicarea unei căi de dirijare încă de la sursa de trafic \cite{schneider2014standardizations}, calcularea acesteia putând fi făcută atât centralizat, cât și distribuit. Tot în cadrul \gls{ietf} se dezvoltă și protocolul \gls{netconf}, însă acesta va fi detaliat într-o secţiune viitoare.

Parte din \gls{ietf} este și \gls{irtf}. Aceasta este o organizaţie paralelă, care se ocupă mai mult de partea de cercetare, sprijinind proiectele care se crede că vor fi benefice comunităţii Internetului, dar care nu sunt încă pregătite pentru implementare. Aici există un grup de lucru care investighează tehnologia \gls{sdn} din mai multe perspective, cu scopul de a recunoaşte atât abordări ce pot fi definite și lansate pe termen scurt, dar și provocări viitoare: \gls{sdnrg}. Sunt de interes probleme ca extensibilitatea soluţiei, abstractizări sau limbaje de programare ce pot fi folosite în contextul \gls{sdn}. Se doreşte ca acest grup să ofere un spațiu comun tuturor cercetătorilor cu interes în acest domeniu.

%\subsection{ETSI}
%
%Unul dintre grupurile specificaţiilor pentru industrie din cadrul \gls{etsi} se numeşte grupul \gls{nfv}. Conceptul de virtualizare a funcţiilor de rețea este unul complementar rețelelor definite prin software. Este o iniţiativă a operatorilor de rețea și are ca scop să permită acestora să profite de noile tehnologii ce apar, să le crească abilităţile de a livra servicii, să reducă costurile și să crească viteza de lansare în producție a serviciilor noi sau experimentale.
%
%Chiar dacă accentul este pus pe virtualizarea funcţiilor de rețea, a fost foarte bine înţeles faptul că utilizarea tehnologiei \gls{sdn} reprezintă un avantaj major și că multe dintre cazurile de utilizare propuse pot fi implementate folosind această paradigmă. Astfel, grupul lucrează la standardizare și stabileşte protocoale împreună cu celelalte organizaţii care se ocupă de standardizarea \gls{sdn}.
%
%\subsection{ITU-T}
%
%Unul dintre cele mai vechi organizaţii de standardizare, \gls{itu-t}, a avut o importanţă majoră în multe aspecte critice din domeniul telecomunicaţiilor. În contextul \gls{sdn}, grupul se ocupă de în principal de arhitectura și cerinţe pentru utilizarea acestei tehnologii în rețelele de transport de date. Deoarece acest tip de rețele au cerinţe importante, diferite de celelalte rețele, evidenţierea acestora de către \gls{itu-t} ajută la orientarea eforturilor de standardizare în direcţia corectă. 
%
%Zonele de interes pentru grupurile care fac parte din această organizaţie sunt: aspecte arhitecturale ale securităţii în \gls{sdn} și servicii pentru securitate prin \gls{sdn}, specificaţii pentru arhitectura planului de control în rețele de transport de date, cerinţe de semnalizare folosind tehnologia \gls{sdn} sau cerinţe funcţionale și arhitecturale pentru \gls{sdn} și rețelele viitorului.
