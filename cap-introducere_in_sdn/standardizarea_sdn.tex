\section{Standardizarea SDN}

Activitățile de cercetare și standardizare din jurul \gls{sdn} au loc pe două planuri importante. Pe de o parte, există organizaţiile care dezvoltă standarde - \gls{sdo}, care sunt formate din reprezentanţi ai industriei, ai academiei, sau alte entități și au ca scop dezvoltarea de specificaţii sau recomandări tehnice, în contextul \gls{sdn} care să fie folosite de toată industria rețelisticii. Autorii din~\cite{schneider2014standardizations} amintesc astfel de organizații: \gls{onf}, \gls{ietf}, \gls{etsi}, \gls{itu-t}, \gls{ieee}. 

Pe de altă parte, există asociații sau comunități de oameni, în general care fac parte din industrie, ale căror rezultate ale cercetării candidează pentru a deveni standarde. Aceste rezultate sunt, de obicei, implementări cu sursă deschisă ce vor fi folosite ulterior în industrie. Exemple de astfel de comunități sunt prezente în~\cite{halpern2014standards, meyer2013software}: OpenDaylight (activităţi ce se desfăşoară sub patronajul fundaţiei Linux), \gls{mef}, \gls{bbf}, \gls{oif}.

Activitățile de standardizare ale \gls{sdn} sunt foarte importante, asigurându-i acestei tehnologii o evoluţie stabilă și aducând-o la o maturitate care îi va permite adoptarea pe scară largă în industria rețelisticii, mitigând astfel dezavantajele rețelelor tradiționale.

Sunt mai multe planuri pe care se lucrează pentru standardizarea \gls{sdn}. Unul dintre aceste planuri este cel al interfeţelor de tip \textit{southbound}, care fac legătura între echipamentele de dirijare și echipamentele de control \gls{sdn}. Protocolul OpenFlow este un exemplu în acest sens, însă nu este singurul protocol capabil să facă legătura între planurile de date și de control. În ultimul timp se pune foarte mare accent pe \gls{netconf} ca o alternativă pentru a configura echipamentele care dirijează traficul, după cum se poate observa în~\cite{csoma2015escape, felix2014multi, zhou2014research}. Astfel apare nevoia de a dezvolta modele informaţionale care să abstractizeze echipamentele din planul de date și care să poată fi folosite de \gls{netconf}
pentru a configura dispozitivele. Și planul interfeţelor de tip \textit{northbound} are un rol important în standardizare, deoarece poate oferi un punct de plecare comun pentru dezvoltatorii de aplicații \gls{sdn}. De exemplu, în cadrul \gls{onf} există un grup care se ocupă cu activităţi de standardizare în această direcţie. Un alt plan este reprezentat de implementările software, cu sursă deschisă (\textit{open-source}), care se crează în acest context. Așa cum evidenţiază și autorii din~\cite{lin2014software, rothenberg2014open}, aceste implementări sunt foarte importante prin ecosistemele care apar ca urmare a activităţii \textit{comunităţilor open-source}.

\subsection{ONF}

